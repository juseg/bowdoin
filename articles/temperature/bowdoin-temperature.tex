\documentclass[utf8]{article}

\usepackage{authblk}
\usepackage[T1]{fontenc}
\usepackage[utf8]{inputenc}
\usepackage[pdftex]{xcolor}
\usepackage[pdftex]{graphicx}
%\usepackage[authoryear,round]{natbib}

% review mode
\usepackage{geometry}
\usepackage{lineno}
\linenumbers
\linespread{1.5}

\graphicspath{{../../figures/}}

\definecolor{darkblue}{cmyk}{0.9,0.3,0.0,0.0}
\definecolor{darkgreen}{cmyk}{0.8,0.0,1.0,0.0}
\definecolor{darkred}{cmyk}{0.1,0.9,0.8,0.0}

\newcommand{\idea}[1]{\textcolor{darkgreen}{\emph{[\textbf{IDEA:} #1]}}}
\newcommand{\note}[1]{\textcolor{darkblue}{\emph{[\textbf{NOTE:} #1]}}}
\newcommand{\todo}[1]{\textcolor{darkred}{\emph{[\textbf{TODO:} #1]}}}
\newcommand{\aref}[0]{\textcolor{darkblue}{\textbf{[REF.]}}}

\title{Englacial warming indicates deep crevassing in Bowdoin tidewater
       glacier, northwest Greenland}

\author[1]{Julien Seguinot
           \thanks{Correspondence to seguinot@vaw.baug.ethz.ch}}
\author[1]{Martin Funk}
\author[2]{Shin Sugiyama}

\affil[1]{Laboratory of Hydraulics, Hydrology and Glaciology,
          ETH Zürich, Switzerland}
\affil[2]{Institute of Low Temperature Science,
          Hokkaido University, Sapporo, Japan}

% ======================================================================
\begin{document}
% ======================================================================

\maketitle

\begin{abstract}

    \note{Preliminary abstract. The two first paragraphs will be shortened and
          reworked to better reflect the introduction.}

    The observed rapid retreat of ocean-terminating glaciers in southern
    Greenland in the last two decades has now propagated to the northwest.
    Hence, tidewater glaciers in this area, some of which have remain stable
    for decades, have started retreating rapidly through iceberg calving in
    recent years, thus allowing a monitoring and investigation of ice dynamical
    changes starting the early stages of retreat.

    Bowdoin Glacier, a small and relatively accessible calving glacier in
    Northwest Greenland, appeared to be very stable since its frontal position
    had been first documented by explorers in the late 19th century. However,
    following a two-fold increase of surface velocity in the early 2000s, the
    calving front of Bowdoin Glacier has experienced a rapid retreat of ca.
    2\,km between 2007 and 2013. Since 2013, the ice front is once again
    stable, yet the glacier surface continues to experience lowering at an
    alarming rate of ca. 6\,m/a.

    Here, we present a two-year record of englacial temperatures measured in
    the tidewater glacier tongue. The measured temperature profiles indicate
    that the glacier's ice is below pressure melting point except for its
    temperate base. More surprisingly, a temperature difference of up to 2
    degrees was measured between two boreholes that are located only 250 meter
    apart along a flowline. In addition, the warmer profile shows a slow
    warming trend of up to ca.~0.5\,$^\circ$C\,a$^{-1}$\todo{Refine.}. This
    patterns can be explained by latent heat release from meltwater freezing in
    transverse crevasses that reach to or near the glacier bed. These crevasses
    may be related to the tide-modulated extensional strain regime of the
    tidewater glacier tongue.

\end{abstract}

% ----------------------------------------------------------------------
\section{Introduction}
% ----------------------------------------------------------------------

    In many parts of the world, glaciers are currently slowing down as they
    become thinner. Yet this behaviour is not ubiquitous. Around the margins of
    the Greenland Ice Sheet, ocean-terminating outlet glaciers also thin, but
    accelerate and retreat faster than any other parts of the ice sheet.

    These so-called tidewater glaciers are partly submerged in sea water, such
    that when they thin due to surface melt, more and more of the glaciers'
    weight is balanced by buoyancy forces from the ocean. Basal friction is
    reduced and the glaciers flow faster, thus thinning even more. When the
    glaciers come close to floatation, basal friction may be so low that the
    only factor limiting flow velocities are the internal (longitudinal)
    viscous forces in the ice.

    However, ice viscosity depends critically on temperature. Between
    $-15\,^\circ$C and the pressure-melting point, ice softness varies by as
    much as an order of magnitude. Numerical models show that such differences
    in ice viscosity propagate to influence the flow and shape of entire
    glaciers and ice sheets.

    Previous measurements show that glacier ice temperatures can be below
    the pressure-melting point (cold ice), at the pressure-melting point
    (temperate ice), or both (polythermal glacier). Glacier temperatures are
    primarily controlled by air temperatures, geothermal heat flux, and
    internal heat advection and diffusion, but can also be affected by latent
    heating from meltwater refreezing in surface crevasses.

    Much of this knowledge comes from measurements conducted on mountain
    glaciers and the interior of ice sheets. However, ice temperature
    measurement from the terminal tongues of tidewater glaciers are presently
    lacking. In fact, tidewater glacier tongues flow at hundreds of metres to
    kilometres per year, and are characterised by strong longitudinal
    extension. Heavy crevassing on the glacier surface hinders accessibility
    and poses practical and technical challenges to instrumentation.

    Here, we present a new, three-year, continuous record of englacial
    temperature from Bowdoin Glacier, a relatively accessible tidewater outlet
    glacier of the northwestern Greenland ice sheet. Despite successive data
    losses to ice deformation and harsh climatic conditions, the record reveals
    new mechanisms controlling temperature variations in tidewater glacier.


% ----------------------------------------------------------------------
\section{Methods}
% ----------------------------------------------------------------------

% -- -- -- -- -- -- -- -- -- -- -- -- -- -- -- -- -- -- -- -- -- -- -- -
\subsection{Bowdoin Glacier}
% -- -- -- -- -- -- -- -- -- -- -- -- -- -- -- -- -- -- -- -- -- -- -- -

    Bowdoin Glacier is a relatively accessible tidewater glacier located in
    northwest Greeland. It is a medium-size outlet glacier of the Greenland
    Ice Sheet with a catchment of about 60x60\,km, and drains into Bowdoin
    Fjord across a 3\,km calving front. The glacier front is located 30\,km
    from the village of Qaanaaq and northernmost airport in Greenland.

    The glacier was choosen for fieldwork due to its accessibility, and
    because at the time of field campaign planning it appeared that the
    Greenland Ice Sheet mass loss was propagating northwest. However, the more
    recent satellite gravimetry data show that virtually all margins of the
    Greenland Ice Sheet are now loosing mass. The surface of Bowdoin Glacier
    is heavily crevassed but less so than that of many other greenlandic
    tidewater glaciers. Besides, a ca.~10\,m-wide medial moraine greatly
    facilitate displacements on the glacier surface. The moraine is generally
    uninterrupted by moraines and can be walked down to the glacier front.

    Bowdoin Glacier was first explored (and so named) by western explorers in
    the late-nineteeth century and described in the following terms:
    \begin{quote}
        Beyond that, an isolated mountain of striking boldness and sharpness of
        outline jutted into the air apparently some two thousand feet, and
        then, from its base, the crystal wall of a great glacier stretched
        clear across the opposite side of the bay head. This glacier I named,
        in honour of my Alma Mater, Bowdoin Glacier, and the bay I called
        Bowdoin Bay (Robert E. Peary, \emph{Northward over the Great Ice},
        p.~393--394).
    \end{quote}
    Photographs indicate that the glacier was thicker at that time, but its
    frontal position was only a few kilometers downstream from the present
    calving front, and has been relatively stable since then. However, a
    two-fold increase of surface velocity in the early 2000s preceded a rapid
    frontal retreat of Bowdoin Glacier by ca. 2\,km between 2007 and 2013.
    Since 2013, the ice front is once again stable, yet the tidewater glacier
    tongue continues to experience thinning at an alarming rate of ca. 6\,m/a.
    Bowdoin Glacier longitudinal strain and seismicity vary in response to
    tides and major icebergs calve according to a relatively recurring pattern.


% -- -- -- -- -- -- -- -- -- -- -- -- -- -- -- -- -- -- -- -- -- -- -- -
\subsection{Drilling sites}
% -- -- -- -- -- -- -- -- -- -- -- -- -- -- -- -- -- -- -- -- -- -- -- -

    Fig.~\ref{fig:boreholes}a -- Location map.

    Table.~\ref{tab:drilling} -- Borehole drilling sites.

    During the field campaign in July 2014, three boreholes BH1, BH2, and BH3
    were drilled with hot water in the tidewater tongue of Bowdoin Glacier
    using meltwater from the crevasses.  After the sucessful drilling of BH1
    and BH2 at the first (upper) drilling site ca.~2\,km from the calving
    front, it was planned that two other boreholes would be drilled ca.~x\,km
    ustream in the thicker part of the glacier. However, due to weather the
    hotwater drilling equipment could not be shipped upstream. Thus, a third
    and last hole was drilled ca.~x\,m downstream of the first drilling site
    to be equipped with all instruments left.

    Although the experiment was originally planned for one year, some of the
    instruments were let on the glacier for up to three years as more funding
    was obtained and additional field campaigns were planned for parallel
    experiments. From the drilling in July 2017 to the last field campaign in
    July 2017, the boreholes were displaced by xxx (BHx) to xxx\,m (BHx), and
    the distance between the lower (BH3) and uppermost (BH1) boreholes
    increased from xxx to xxx\,m. New crevasses appeared on the glacier
    surface, sometimes causing damage to the instruments.


% -- -- -- -- -- -- -- -- -- -- -- -- -- -- -- -- -- -- -- -- -- -- -- -
\subsection{Instrumentation}
% -- -- -- -- -- -- -- -- -- -- -- -- -- -- -- -- -- -- -- -- -- -- -- -

    Fig.~\ref{fig:boreholes}b -- Borehole sensors map.

    The boreholes were equipped with three types of sensors: strings of simple,
    regularly-spaced thermistors arranged to span the entire depth of the
    glacier, two piezometers near the base of the glacier, and digital
    inclinometer units at different depths. Besides their primary sensors, the
    piezometers and digital inclinometers were equipped with additional
    thermistors, so that ice temperature could be measured at multiple depths
    in the glacier. This manuscript focuses on the temperature data, and other
    data will be presented elsewhere.

    At the first (upper) drilling site, one borehole (BH1) was equipped with
    seven digital inclinometers and the other (BH2) with one basal piezometer
    and two thermistor strings. At the second (lower) drilling site, the unique
    borehole (BH3) was equipped with five digital inclinometers, one basal
    piezometer and two thermistor strings.

    The basal piezometers (GeoKon xxx) were connected to Campbell (model xxx)
    data loggers. The thermistor strings (model xxx) were connected to
    Campbell CX1000 data loggers. Piezometers and thermistor strings data
    loggers each were powered by a xx\,V lead battery, and mounted in glass
    fiber (?) cases on quadripods (photo). These batteries were carried out and
    back on the glacier during each field campaign to be recharged using a
    petrol generator at the camp.

    The digital inclinometres (DIBOSS, see Ph.D. thesis Ryser) were connected
    to CR1000 data loggers. Each data logger was set-up in a large plastic
    Plelicase (?) including three xx\,V lead batteries and a solar panel on the
    outside, and anchored to aluminum poles drilled into the ice.
    In the case of digital inclinometers, the observed resistance from the
    thermistors were converted to temperature values in-situ by the englacial
    digitizers, thus avoiding to record the resistance of the borehole cables
    and their potential variations due to cable deformation.


% -- -- -- -- -- -- -- -- -- -- -- -- -- -- -- -- -- -- -- -- -- -- -- -
\subsection{Temperature calibration}
% -- -- -- -- -- -- -- -- -- -- -- -- -- -- -- -- -- -- -- -- -- -- -- -

    Table~\ref{tab:calibration} -- Temperature calibration.

    All thermistors were calibrated prior to fieldwork in the lab (Martin?).
    For digital inclinometres the conversion from resistance to temperature
    values is performed in-situ by the englacial digitizers. For basal
    piezometers this conversion is performed by the data logger installed on
    the ice surface (?). For the thermistor strings the conversion is performed
    during post-processing following the formula:
    %
    \begin{equation}
      T = 1 / (a_1 + a_2 \log(R) + a_3 \log(R)^3) - 273.15
    \end{equation}
    %
    where $R$ is the measured resistance, $T$ the ice temperature, and $a_1$,
    $a_2$, and $a_3$ are coefficients calibrated individually for each sensor.

    Despite the pre-field calibration it was observed that initial temperatures
    observed in the borehole immediately after drilling were not at the
    pressure-melting point. Thus for senors where initial data is available,
    a recalibration is applied in post-processing by correcting for the initial
    temperature offset, $\Delta_T$. Unfortunately some of these initial
    temperature data have been lost so that the recalibration was possible for
    some sensors only. The pressure-melting point was computed as
    %
    \begin{equation}
      T_m = -\beta \rho g z
    \end{equation}
    %
    where $\beta$ is the empirical Clapeyron constant for glacier ice
    (Luethi et al., 2002), $\rho$ is the ice density, $g$ is the standard
    acceleration due to gravity, and $z$ is the depth below the ice surface.


% ----------------------------------------------------------------------
\section{Results}
% ----------------------------------------------------------------------

% -- -- -- -- -- -- -- -- -- -- -- -- -- -- -- -- -- -- -- -- -- -- -- -
\subsection{Length of the record}
% -- -- -- -- -- -- -- -- -- -- -- -- -- -- -- -- -- -- -- -- -- -- -- -

    Table~\ref{tab:timeline} -- Events timeline.

    From the two drilling sites on the tidewater tongue of Bowdoin Glacier,
    continuous ice temperature time series could be obtained with a lenght of
    up to three years. Although the experiment was initially planned for one
    year, following initial success some of the instruments were let installed
    for up to three years. However, sensors were successively lost to technical
    failures and ice deformation.

    Of the total 44~sensors installed, 41~worked after the installation, 38
    recorded data for one year, 35 were still functional after two years (of
    which 16 were not connected to a data logger), and only two were left
    working after three years. Contact to sensors was progressively lost to a
    battery issue (2014 July~28 and again 2015 July~26, one sensor) cable
    damage during installation (2014 July~16 and 23, three sensors) cable
    damage at depth (2014 Oct.~25, 2017 Jan.~28, Feb.~03, and June~21, five
    sensors) cable damage at the surface (2015 Nov.~12 and unknown dates,
    33~sensors).


% -- -- -- -- -- -- -- -- -- -- -- -- -- -- -- -- -- -- -- -- -- -- -- -
\subsection{Temperature time series}
% -- -- -- -- -- -- -- -- -- -- -- -- -- -- -- -- -- -- -- -- -- -- -- -

    Fig.~\ref{fig:timeseries} -- Temperature time series.

    After the drillings, most sensors are immerged and temperatures are
    measured at or near the pressure-melting point. In BH2 some sensors record
    strong temperature variations as they are temporarily emmerged. Temperature
    then drop off the pressure-melting point and follow an S-curve before
    stabilizing below freezing. This initial phase lasts for hours to months
    depending on the sensor and relating to the equilibrium temperature. While
    most sensors reach a thermal equilibrium within three months, some refreeze
    only after more than a year, and stabilize at temperatures below but near
    the pressure-melting point.

    The sensors nearest to the ice surface exhibit a seasonal temperature
    cycle, which is out-of-phase with the atmospheric temperature cycle. The
    amplitude increase with time as the thermistor strings progressively
    melt-out of the glacier. Temperature records for sensors installed deeper
    down in the ice do not show a seasonal cycle, but many exhibit a slow
    warming trend.


% -- -- -- -- -- -- -- -- -- -- -- -- -- -- -- -- -- -- -- -- -- -- -- -
\subsection{Temperature profiles}
% -- -- -- -- -- -- -- -- -- -- -- -- -- -- -- -- -- -- -- -- -- -- -- -

    Fig.~\ref{fig:profiles}a -- Temperature profiles.

    After the refreezing of the entire borehole, vertical profiles depict
    temperatures below the pressure-melting point except for the base of the
    glacier where temperatures are at the pressure-melting point. The coldest
    temperatures are reached in the middle part of the ice column, above which
    is found a layer of relatively high temperatures, and finally a layer with
    seasonal temperature variations.

    The upper boreholles BH1 and BH2, which are separated by only ten metres,
    show compatible temperature profiles. For the lower borehole, BH3, the
    deepest thermistor string depict temperatures incompatible with those
    recorded by digital inclinometers in the same hole. Because the depths of
    digital inclinometers was calibrated from independent pressure sensors, we
    interpret that sensors on the thermistor string were misplaced due to a
    manipulation error and mark these data as erratic (ERR) (maybe this should
    go to the instrumentation section).

    Finally, the upper (BH1, BH2) and lower (BH3, ERR) boreholes show
    significant temperature differences despite being located on the same
    flowline and separated by only xxx\,m. Remarkedly, temperature differences
    around 2$^\circ$C prevail over the entire glacier depth.


% -- -- -- -- -- -- -- -- -- -- -- -- -- -- -- -- -- -- -- -- -- -- -- -
\subsection{Englacial warming}
% -- -- -- -- -- -- -- -- -- -- -- -- -- -- -- -- -- -- -- -- -- -- -- -

    Fig.~\ref{fig:profiles}b -- Warming profiles.

    All three temperature profiles show a general warming trend, except for
    their base were some sensors take several months to refreeze after the
    boreholes was drilled.  However, the warming trend is much more pronounced
    for the lower (BH3) than for the upper (BH1, BH2) boreholes. Part of this
    englacial warming can be explained by heat diffusion from the base and
    near-surface of the glacier to its coldest depths.

    Theoretical warming due to heat diffusion was approximated by central
    differences on the one-dimensional diffusion equation,
    %
    \begin{equation}
      \frac{\partial T}{\partial T} =
        \frac{k}{\rho c} \frac{\partial^2 T}{\partial z^2}
    \end{equation}
    %
    where $k$ is the thermal conductivity of ice, and $c$ its specific heat
    capacity. However, the observed warming trend is up to an order of
    magnitude higher than the theoretical warming by diffusion.


% ----------------------------------------------------------------------
\section{Discussion}
% ----------------------------------------------------------------------

\subsection{Deep crevassing}

    Fig.~\ref{fig:arcticdem} -- Topography around boreholes.

    Such local temperature variations have previously been observed in
    Greenlandic glaciers, but were restricted to the first xx metres below
    the glacier surface. Such differences have been explain by meltwater
    refreezing in crevasses. In summer surface meltwater penetrates in open
    crevasses. In winter it refreezes generating heat.

    However the temperature profiles at Bowdoin glacier show temperature
    differences over the entire depth of the glacier. Besides englacial warming
    an order of magnitude stronger than the theoretical warming due to heat
    diffusion occurs over much of the depth of the glacier. The different
    profiles and warming trend can be explained my refreezing of meltwater
    in deep crevasses reaching near or to the glacier bed.

    Although such variations could also be explained by the proximity of a
    moulin, no large moulins have been observed in the vicinity of the borehole
    sites or elsewhere on Bowdoin Glacier during the field campaigns. Small
    moulins have sometimes been observed, but such moulins would most likely
    refreeze over the winter and thus could not explained the observed
    continuous warming trend.

    Because it is a tidewater glacier, Bowdoin Glacier is subject to continuous
    extension yielding to the formation of transverse crevasses. The
    longitudinal extension is stronger at lowering tide and associated with
    seismic activity which is most likely indicating crevasse opening.

    The boreholes were drilled in a highly crevassed area. However, there is
    no clear surface expression of a higher crevassing in the vicinity of the
    lower drilling site.

\subsection{Link to ogives}

    The surface topography of Bowdoin Glacier exhibit bands, which are advected
    together with the flow of the glacier. They have a wavelength of around
    xxx\,m and an amplitude of xx\,m. These bands can also be visualized on
    satellite images. Topographic high appear as whiter and less crevassed
    areas than topographic lows. They originate from a steeper part of the
    glacier about xx\,km from the calving front and thus could be considered as
    some kind of ogives.

    The upper borehole site (BH1, BH2) is located on a topographic high and the
    lower borehole site (BH3) is located on a topographic low. This topography
    is not so obvious for an observer located on the surface of the glacier,
    because it is overprinted by the smaller-wavelength and similar-amplitude
    topography of penitents and crevasses. But the view from the upper drilling
    site was better than from the lower drilling site.

    We speculate that areas of lower topography may tend to localise crevassing
    as the glacier is thinner and thus weaker there, and meltwater has more
    chance to penetrate into the glacier there.

% ----------------------------------------------------------------------
\section{Conclusions}
% ----------------------------------------------------------------------

    We present the first continuous temperature record from the terminal part
    of a tidewater glacier.

    It can take more than a year for a borehole to entirely refreeze, so it is
    interesting to have continuous measurements.

    Temperature profiles separated by less than 200\,m and differing by up to
    2\,$^\circ$C indicating strong local temperature variations over the
    entire ice column.

    The warming trend of up to 0.6$^\circ$C in two years can not be explained
    by temperature diffusion and indicates deep refreezing.

    This is most likely the expression of deep crevassing. The localisation of
    such deep crevasses could be controlled by ogive banding.


% ----------------------------------------------------------------------
% Figures
\clearpage
% ----------------------------------------------------------------------

    \begin{figure}
      \centerline{\includegraphics{bowtem_boreholes}}
      \caption{\textbf{(a)} Bowdoin borehole locations from drilling in July
               2014 to dismantling in July 2017 and background satellite
               image from 8 Aug. 2016, 17:59:15 UTC. Contains modified
               Copernicus Sentinel data, processed with Sentinelflow.
               \textbf{(b)} Initial ice thickness and sensor depths.
               Inclinometers and piezometers are also equipped with
               thermistors.}
      \label{fig:boreholes}
    \end{figure}

    \begin{figure}
      \centerline{\includegraphics{bowtem_timeseries}}
      \caption{Ice temperature time series from drilling in July 2014 to
               dismantling in July 2017. Field campaigns (orange spans) and
               dates selected for temperature profiles (dashed lines;
               Fig.~\ref{fig:profiles}) are indicated.}
      \label{fig:timeseries}
    \end{figure}

    \begin{figure}
      \centerline{\includegraphics{bowtem_profiles}}
      \caption{\textbf{(a)} Daily mean temperature profiles for selected dates
               (Fig.~\ref{fig:timeseries}) three months after the drilling
               (solid lines) and long after the refreezing of the entire
               borehole (dashed lines).
               \textbf{(b)} Theoretical temperature diffusion (solid lines) and
               observed temperature changes (dashed lines) for the
               corresponding period.}
      \label{fig:profiles}
    \end{figure}

    \begin{figure}
      \centerline{\includegraphics{bowtem_arcticdem}}
      \caption{\textbf{(a)} Glacier surface topography in the vicinity of the
               boreholes sites in the early phase of the experiment (Arctic
               DEM, 6 Sep. 2014) and corresponding locations of the boreholes.
               The locations of BH2 and BH3 are estimated from their initial
               locations in July 2014 (color crosses) and the observed
               displacement of BH1 during the corresponding time intervals.
               \textbf{(b)} Glacier surface topographic profile along flow
               passing through BH1 and the estimated location BH3.}
      \label{fig:arcticdem}
    \end{figure}


% ----------------------------------------------------------------------
% Tables
\clearpage
% ----------------------------------------------------------------------

    \begin{table}[t]
      \caption{%
        Bowdoin Glacier drilling sites.}
      \label{tab:drilling}
      {\begin{tabular}{cccccc}
        \hline
        Hole & Depth & Date       & Latitude  & Longitude  & Elev. \\
        \hline
        BH1  & 272   & 2014-07-16 & 77.691244 & -68.555749 & 88.7 \\ % UI
        BH2  & 262   & 2014-07-17 & 77.691307 & -68.555685 & 87.7 \\ % UPT
        BH3  & 252   & 2014-07-23 & 77.689995 & -68.558857 & 83.4 \\ % LIPT
        \hline
      \end{tabular}}
    \end{table}

    \begin{table}
      \caption{%
        Timeline of events including borehole drilling, instrumental failures,
        and final removal.}
      \label{tab:timeline}
      {\begin{tabular}{lccp{95mm}}
        \hline
        Date       & Hole & Type      & Event \\
        \hline
        2014-07-15 & ---  & Campaign & \emph{2014 summer fieldwork begins} \\
        2014-07-16 & BH1  & Drilling & Contact to the deepest inclinometer
                                       (UI01) is lost. \\
        2014-07-17 & BH2  & Drilling & -- \\
        2014-07-20 & BH2  & Incident & Earlier upper thermistor (UT*) data are
                                       lost accidentally. \\
        2014-07-23 & BH3  & Drilling & Contact to the deepest inclinometers
                                       (LI01, LI02) is lost. \\
        2014-07-28 & BH2  & Battery  & The upper piezometer data logger battery
                                       drains prematurely without notice from
                                       the fieldwork participants. \\
        2014-07-29 & ---  & Campaign & \emph{2014 summer fieldwork ends} \\
        \hline
        2014-10-25 & BH1  & Incident & Contact to the deepest inclinometers
                                       (UI02, UI03) is lost. \\
        \hline
        2015-07-06 & ---  & Campaign & \emph{2015 summer fieldwork begins} \\
        2015-07-09 & BH2  & Upgrade  & The upper piezometer (UP) data logger
                                       gets a new battery. \\
        2015-07-14 & BH2  & Removal  & The upper thermistor strings (UT*) data
                                       logger is removed. \\
        2015-07-20 & ---  & Campaign & \emph{2015 summer fieldwork ends} \\
        \hline
        2015-07-26 & BH2  & Battery  & The new battery for the upper piezometer
                                       (UP) drains again prematurely, perhaps
                                       indicating faulty instruments. \\
        2015-11-12 & BH3  & Incident & Contact to all remaining inclinometers
                                       (LI03, LI04, LI05) is lost due to
                                       a crevasse opening on the surface. \\
        \hline
        2016-07-04 & ---  & Campaign & \emph{2016 summer fieldwork begins} \\
        2016-07-19 & BH3  & Removal  & The lower thermistor strings (LT*) data
                                       logger is removed. \\
        2016-07-21 & ---  & Campaign & \emph{2016 summer fieldwork ends} \\
        \hline
        2017-01-28 & BH1  & Incident & Contact to the deepest inclinometer
                                       (UI06) is lost. \\
        2017-02-03 & BH3  & Incident & The lower piezometer (LP) starts
                                       recording nonsense. \\
        2017-06-21 & BH1  & Incident & Contact to the deepest inclimoneter
                                       (UI07) is lost. \\
        ?          & BH2  & Incident & Contact to the thermistor strings (UT*)
                                       is lost. \\
        ?          & BH3  & Incident & Contact to the thermistor strings (LT*)
                                       is lost. \\
        \hline
        2017-07-04 & ---  & Campaign & \emph{2017 summer fieldwork begins} \\
        2017-07-12 & BH1  & Removal  & The upper inclinometer (UI*) data logger
                                       is removed. \\
        2017-07-17 & ---  & Campaign & \emph{2017 summer fieldwork ends} \\
        \hline
      \end{tabular}}
    \end{table}

    \begin{table}
      \label{tab:calibration}
      \begin{minipage}[b]{85mm}
        \begin{tabular}{ccccc}
          \hline
          Sensor & $a_1\times10^{3}$ & $a_2\times10^{4}$
                 & $a_3\times10^{7}$ & $\Delta T$ (K) \\
          \hline
          UI02 &   ---   &   ---   &   ---   &     ---   \\
          UI03 &   ---   &   ---   &   ---   &     ---   \\
          UI04 &   ---   &   ---   &   ---   &     ---   \\
          UI05 &   ---   &   ---   &   ---   &     ---   \\
          UI06 &   ---   &   ---   &   ---   &     ---   \\
          UI07 &   ---   &   ---   &   ---   &     ---   \\
          UP   &   ---   &   ---   &   ---   &     ---   \\
          UT01 & 2.62948 & 3.15882 &-2.11729 &     ---   \\
          UT02 & 2.70630 & 2.83255 & 4.85636 &     ---   \\
          UT03 & 2.71054 & 2.85092 & 4.36997 &     ---   \\
          UT04 & 2.70327 & 2.85200 & 4.38303 &     ---   \\
          UT05 & 2.70416 & 2.83808 & 4.40717 &     ---   \\
          UT06 & 2.67961 & 2.96215 & 1.95166 &     ---   \\
          UT07 & 2.71620 & 2.79459 & 5.60670 &     ---   \\
          UT08 & 2.70382 & 2.83626 & 4.52513 &     ---   \\
          UT09 & 2.72032 & 2.78078 & 5.86472 &     ---   \\
          UT10 & 2.71264 & 2.82098 & 4.98119 &     ---   \\
          UT11 & 2.70714 & 2.83545 & 4.61922 &     ---   \\
          UT12 & 2.69855 & 2.87227 & 3.95361 &     ---   \\
          UT13 & 2.70676 & 2.84831 & 4.31297 &     ---   \\
          UT14 & 2.73315 & 2.73929 & 7.08468 &     ---   \\
          UT15 & 2.70659 & 2.83499 & 4.69136 &     ---   \\
          UT16 & 2.71143 & 2.82174 & 5.01209 &     ---   \\
          \hline
        \end{tabular}
      \end{minipage}
      \begin{minipage}{85mm}
        \caption{%
          Temperature calibration coefficients and melt offset corrections.}
        \begin{tabular}{ccccc} \\
          \hline
          Sensor & $a_1\times10^{3}$ & $a_2\times10^{4}$
                 & $a_3\times10^{7}$ & $\Delta T$ (K) \\
          \hline
          LI03 &   ---   &   ---   &   ---   & -0.366179 \\
          LI04 &   ---   &   ---   &   ---   &  0.153631 \\
          LI05 &   ---   &   ---   &   ---   &  0.077992 \\
          LP   &   ---   &   ---   &   ---   &     ---   \\
          LT01 & 2.68849 & 2.91312 & 2.90792 & -0.146871 \\
          LT02 & 2.72120 & 2.77064 & 6.36107 & -0.113681 \\
          LT03 & 2.72538 & 2.75376 & 6.65425 & -0.155103 \\
          LT04 & 2.72203 & 2.75468 & 6.66331 & -0.157251 \\
          LT05 & 2.72629 & 2.74648 & 6.70740 & -0.099604 \\
          LT06 & 2.71837 & 2.77063 & 6.77034 & -0.088528 \\
          LT07 & 2.71239 & 2.80048 & 5.56011 & -0.114487 \\
          LT08 & 2.71464 & 2.79042 & 5.78066 & -0.106990 \\
          LT09 & 2.71831 & 2.78497 & 5.80214 & -0.117413 \\
          LT10 & 2.73324 & 2.72923 & 7.50070 & -0.120982 \\
          LT11 & 2.72182 & 2.77915 & 5.88847 & -0.078674 \\
          LT12 & 2.71306 & 2.80286 & 5.33636 & -0.085439 \\
          LT13 & 2.75088 & 2.66886 & 8.65715 & -0.098273 \\
          LT14 & 2.71922 & 2.79321 & 5.78862 & -0.126086 \\
          LT15 & 2.71990 & 2.78311 & 5.88938 & -4.615325 \\
          LT16 & 2.71911 & 2.78555 & 5.89390 & -2.064276 \\
          \hline
        \end{tabular}
      \end{minipage}
    \end{table}

    \begin{table}
      \caption{%
        Parameter values used to compute the pressure melting-point and
        theoretical temperature diffusion.}
      \label{tab:parameters}
      {\begin{tabular}{llrll}
        \hline
        Not.    & Name & Value & Unit \\
        \hline
        $\beta$ & Clapeyron constant
                & $7.9\times10^{-8}$    & K\,Pa$^{-1}$                      \\
        $c$     & Ice specific heat capacity
                & 2009                  & J\,kg$^{-1}$\,K$^{-1}$            \\
        $g$     & Standard gravity
                & 9.80665               & m\,s$^{-2}$                       \\
        $k$     & Ice thermal conductivity
                & 2.1                   & J\,m$^{-1}$\,K$^{-1}$\,s$^{-1}$   \\
        %$L$    & Water latent heat of fusion
        %       & $3.34\times10^5$      & J\,kg$^{-1}$\,K$^{-1}$            \\
        $\rho$  & Ice density
                & 910                   & kg\,m$^{-3}$                      \\
        \hline
      \end{tabular}}
    \end{table}

% ======================================================================
\end{document}
% ======================================================================

\documentclass{article}

\usepackage{authblk}
\usepackage[T1]{fontenc}
\usepackage[utf8]{inputenc}
\usepackage[pdftex]{xcolor}
\usepackage[pdftex]{graphicx}

\graphicspath{{../../figures/}}

\definecolor{darkblue}{cmyk}{0.9,0.3,0.0,0.0}
\definecolor{darkgreen}{cmyk}{0.8,0.0,1.0,0.0}
\definecolor{darkred}{cmyk}{0.1,0.9,0.8,0.0}

\newcommand{\idea}[1]{\textcolor{darkgreen}{\emph{[\textbf{IDEA:} #1]}}}
\newcommand{\note}[1]{\textcolor{darkblue}{\emph{[\textbf{NOTE:} #1]}}}
\newcommand{\todo}[1]{\textcolor{darkred}{\emph{[\textbf{TODO:} #1]}}}
\newcommand{\aref}[0]{\textcolor{darkblue}{\textbf{[REF.]}}}

\title{Englacial warming indicates deep crevassing in Bowdoin tidewater
       glacier, northwest Greenland}

\author[1]{Julien Seguinot
           \thanks{Correspondence to seguinot@vaw.baug.ethz.ch}}
\author[1]{Martin Funk}
\author[2]{Shin Sugiyama}

\affil[1]{Laboratory of Hydraulics, Hydrology and Glaciology,
          ETH Zürich, Switzerland}
\affil[2]{Institute of Low Temperature Science,
          Hokkaido University, Sapporo, Japan}

% ======================================================================
\begin{document}
% ======================================================================

\maketitle

\begin{abstract}

    \note{Preliminary abstract. The two first paragraphs will be shortened and
          reworked to better reflect the introduction.}

    The observed rapid retreat of ocean-terminating glaciers in southern
    Greenland in the last two decades has now propagated to the northwest.
    Hence, tidewater glaciers in this area, some of which have remain stable
    for decades, have started retreating rapidly through iceberg calving in
    recent years, thus allowing a monitoring and investigation of ice dynamical
    changes starting the early stages of retreat.

    Bowdoin Glacier, a small and relatively accessible calving glacier in
    Northwest Greenland, appeared to be very stable since its frontal position
    had been first documented by explorers in the late 19th century. However,
    following a two-fold increase of surface velocity in the early 2000s, the
    calving front of Bowdoin Glacier has experienced a rapid retreat of ca.
    2\,km between 2007 and 2013. Since 2013, the ice front is once again
    stable, yet the glacier surface continues to experience lowering at an
    alarming rate of ca. 6\,m/a.

    Here, we present a two-year record of englacial temperatures measured in
    the tidewater glacier tongue. The measured temperature profiles indicate
    that the glacier's ice is below pressure melting point except for its
    temperate base. More surprisingly, a temperature difference of up to 2
    degrees was measured between two boreholes that are located only 250 meter
    apart along a flowline. In addition, the warmer profile shows a slow
    warming trend of up to ca.~0.5\,$^\circ$C\,a$^{-1}$\todo{Refine.}. This
    patterns can be explained by latent heat release from meltwater freezing in
    transverse crevasses that reach to or near the glacier bed. These crevasses
    may be related to the tide-modulated extensional strain regime of the
    tidewater glacier tongue.

\end{abstract}

% ----------------------------------------------------------------------
\section{Introduction}
% ----------------------------------------------------------------------

    Ice rheology depends on temperature. But temperature was never measured on
    tidewater glaciers. Yet Bowdoin Glacier allows to do it.


% ----------------------------------------------------------------------
\section{Methods}
% ----------------------------------------------------------------------

\subsection{Drilling sites}
    Fig.~\ref{fig:boreholes}a -- Location map.

\subsection{Instrumentation}
    Fig.~\ref{fig:boreholes}b -- Borehole sensors map.

\subsection{Temperature calibration}

\subsection{Depth correction}
    Fig.~\ref{fig:closure}a -- Closure time and temperature.\\
    Fig.~\ref{fig:closure}b -- Closure time and depth.

% ----------------------------------------------------------------------
\section{Results}
% ----------------------------------------------------------------------

\subsection{Temperature time series}
    Fig.~\ref{fig:timeseries} -- Temperature time series.

\subsection{Temperature profiles}
    Fig.~\ref{fig:profiles}a -- Temperature profiles.

\subsection{Englacial warming}
    Fig.~\ref{fig:profiles}b -- Warming profiles.


% ----------------------------------------------------------------------
\section{Discussion}
% ----------------------------------------------------------------------

\subsection{Deep crevassing}
    Fig.~\ref{fig:arcticdem} -- Topography around boreholes.

\subsection{Tidal effects}

% ----------------------------------------------------------------------
\section{Conclusions}
% ----------------------------------------------------------------------


% ----------------------------------------------------------------------
% Figures
\clearpage
% ----------------------------------------------------------------------

    \begin{figure}
      \centerline{\includegraphics{bowtem_boreholes}}
      \caption{\textbf{(a)} Bowdoin borehole locations from drilling in July
               2014 to dismantling in July 2017 and background satellite
               image from 8 Aug. 2016, 17:59:15 UTC. Contains modified
               Copernicus Sentinel data, processed with Sentinelflow.
               \textbf{(b)} Initial ice thickness and sensor depths.
               Inclinometers and piezometers are also equipped with
               thermistors.}
      \label{fig:boreholes}
    \end{figure}

    \begin{figure}
      \centerline{\includegraphics{bowtem_closure}}
      \caption{\textbf{(a) Measured average temperature on }.\textbf{(b)}.
               \todo{Mask BH2 missing data, fit BH1 depths.}}
      \label{fig:closure}
    \end{figure}

    \begin{figure}
      \centerline{\includegraphics{bowtem_timeseries}}
      \caption{\textbf{(a)}.\textbf{(b)}.
               \todo{Add campaigns and profile dates, mask low values.}}
      \label{fig:timeseries}
    \end{figure}

    \begin{figure}
      \centerline{\includegraphics{bowtem_profiles}}
      \caption{\textbf{(a)}.\textbf{(b)}.
               \todo{Compute theoretical temperature diffusion.}}
      \label{fig:profiles}
    \end{figure}

    \begin{figure}
      \centerline{\includegraphics{bowtem_arcticdem}}
      \caption{\textbf{(a)}.\textbf{(b)}.
               \todo{Maybe enlarge the plotting domain.}}
      \label{fig:arcticdem}
    \end{figure}

% ======================================================================
\end{document}
% ======================================================================

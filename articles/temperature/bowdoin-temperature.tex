\documentclass[utf8]{article}

\usepackage{bm}
\usepackage{doi}
\usepackage{authblk}
\usepackage[T1]{fontenc}
\usepackage[utf8]{inputenc}
\usepackage[pdftex]{xcolor}
\usepackage[pdftex]{graphicx}
\usepackage[authoryear,round]{natbib}

% review mode
\usepackage{geometry}
\usepackage{lineno}
\linenumbers
\linespread{1.5}

\graphicspath{{../../figures/}}

\definecolor{c0}{HTML}{1f77b4}
\definecolor{c1}{HTML}{ff7f0e}
\definecolor{c2}{HTML}{2ca02c}
\definecolor{c3}{HTML}{d62728}
\definecolor{c4}{HTML}{9467bd}
\definecolor{c5}{HTML}{8c564b}
\definecolor{c6}{HTML}{e377c2}
\definecolor{c7}{HTML}{7f7f7f}
\definecolor{c8}{HTML}{bcbd22}
\definecolor{c9}{HTML}{17becf}

\newcommand{\idea}[1]{\textcolor{c2}{\emph{[\textbf{IDEA:} #1]}}}
\newcommand{\note}[1]{\textcolor{c0}{\emph{[\textbf{NOTE:} #1]}}}
\newcommand{\todo}[1]{\textcolor{c3}{\emph{[\textbf{TODO:} #1]}}}

\hypersetup{colorlinks, citecolor=c0, linkcolor=c1, urlcolor=c6}

\title{Englacial warming indicates deep crevassing\\
       in Bowdoin Glacier, Greenland}

\author[1]{Julien Seguinot
           \thanks{Correspondence to seguinot@vaw.baug.ethz.ch}}
\author[1]{Martin Funk}
\author[1]{Andreas Bauder}
\author[1]{Thomas Wyder}
\author[2]{Cornelius Senn}
\author[3]{Shin Sugiyama}

\affil[1]{Laboratory of Hydraulics, Hydrology and Glaciology,
          ETH Zürich, Switzerland}
\affil[2]{Department of Civil, Environmental and Geomatic Engineering,
          ETH Zürich, Switzerland}
\affil[3]{Institute of Low Temperature Science, Hokkaido University,
          Sapporo, Japan}


% ======================================================================
\begin{document}
% ======================================================================

\maketitle

\begin{abstract}

    \todo{R1 - use past tense, e.g. in first sentence.}

    All around the margins of the Greenland Ice Sheet, marine-terminating
    glaciers currently thin and accelerate. The reduced basal friction yields
    increased flow velocity, while the rate of longitudinal stretching is
    limited by ice viscosity, which itself critically depends
    on temperature. However, ice temperature has rarely
    been measured on such fast-flowing and heavily crevassed glaciers.
    %
    Here, we present a three-year record of englacial temperatures obtained
    2 (in 2014) to 1\,km (in 2017) from the calving front of Bowdoin Glacier, a
    tidewater glacier in northwestern Greenland. Two boreholes separated by 165
    (2014) to 197\,m (2017) show significant temperature differences averaging
    2.07\,$^\circ$C on their entire depth. Englacial warming of up to
    0.39$^\circ$C\,a$^{-1}$, an order of magnitude above the theoretical rate
    of heat diffusion and viscous dissipation, indicates a deep and local heat
    source within the tidewater glacier.
    %
    We interpret the heat source as latent heat from meltwater refreezing in
    crevasses reaching to, or near to, the bed of the glacier, whose localisation
    may be controlled by preferential mechanical damage and meltwater ponding
    in topographic dips between ogives.

\end{abstract}

% ----------------------------------------------------------------------
\section{Introduction}
% ----------------------------------------------------------------------

    \todo{R1 - use past tense, e.g. outlet glacier thinned.}

    In many parts of the world, glaciers are currently slowing down as they
    become thinner \citep{Heid.Kaab.2012, Dehecq.etal.2018}. Yet this behaviour
    is not ubiquitous. Around the margins of
    the Greenland Ice Sheet, marine-terminating outlet glaciers also thin, but
    they accelerate and retreat faster than any of its other parts
    \citep[e.g.,][]{Krabill.etal.2000, Rignot.Kanagaratnam.2006,
    Pritchard.etal.2009, Hill.etal.2017}, significantly impacting the total
    mass loss of the ice sheet \citep[e.g.,][]{Enderlin.etal.2014,
    Khan.etal.2015, McMillan.etal.2016}.

    These tidewater glaciers are partly submerged in sea water, so that, as the
    glaciers thin, the gravitational force acting on the ice is increasingly
    counterbalanced by buoyancy forces from the ocean. Basal friction is
    reduced and the glaciers flow faster, thus thinning even more
    \citep{Meier.Post.1987}. When the glaciers come close to floatation,
    basal drag is drastically reduced \citep{Shapero.etal.2016}.
    Longitudinal stress coupling then becomes apparent through the upstream
    propagation of tidal velocity variations \citep{Walters.1989,
    Walter.etal.2012, Sugiyama.etal.2014, Podolskiy.etal.2017,
    Seddik.etal.2019}. In such conditions, the increased flow velocities and
    longitudinal stretching are largely controlled by the ice viscosity.
    However, ice viscosity depends critically on temperature. Between
    $-15\,^\circ$C and the pressure-melting point, ice softness varies by as
    much as an order of magnitude \citep[p.~72]{Cuffey.Paterson.2010}.
    Numerical models show that such differences
    in ice viscosity influence the flow and shape of entire glaciers and ice
    sheets \citep[e.g. Figs.~2 and~7 of][]{Seguinot.etal.2016}.

    Previous measurements show that glacier ice temperatures can be below
    the pressure-melting point (cold ice), at the pressure-melting point
    (temperate ice), or both (polythermal glacier)
    \citep[p.~399]{Ahlmann.1935, Cuffey.Paterson.2010}. Glacier temperatures
    are primarily controlled by air temperatures, geothermal heat flux, viscous
    dissipation, and internal heat advection and diffusion \citep{Q-Robin.1955},
    but can also be affected by latent heat released by meltwater refreezing in
    crevasses \citep{Phillips.etal.2010, Phillips.etal.2013, Luthi.etal.2015}.

    Much of this knowledge comes from measurements conducted on mountain
    glaciers and the interior of ice sheets. Ice temperature measurements from
    tidewater glaciers are presently available near the calving front for
    Svalbard \citep{Jania.etal.1996} but limited to upstream areas in Greenland
    \citep{Iken.etal.1993, Luthi.etal.2002, Luthi.etal.2015, Doyle.etal.2018},
    as heavy crevassing typically hinders accessibility and poses practical and
    technical challenges to instrumentation near the glacier front.

    Here, we present a new, three-year, continuous record of englacial
    temperature from Bowdoin Glacier, a tidewater outlet
    glacier of the northwestern Greenland Ice Sheet. Despite some data gaps,
    the record reveals new mechanisms controlling temperature variations in a
    tidewater glacier.


% ----------------------------------------------------------------------
\section{Methods}
% ----------------------------------------------------------------------

% -- -- -- -- -- -- -- -- -- -- -- -- -- -- -- -- -- -- -- -- -- -- -- -
\subsection{Bowdoin Glacier}
% -- -- -- -- -- -- -- -- -- -- -- -- -- -- -- -- -- -- -- -- -- -- -- -

    Bowdoin Glacier is a tidewater glacier located in northwestern Greenland.
    It is a medium-size outlet glacier of the Greenland Ice Sheet with a
    catchment of about 60$\times$60\,km, and drains into Bowdoin Fjord across a
    3\,km-wide calving front (Figs.~\ref{fig:images}a
    and~\ref{fig:boreholes}a). The glacier front is located ca.~30\,km from
    the settlement of Qaanaaq
    \citep[see Fig.~1 of][for a map]{Sugiyama.etal.2015}.

    The glacier was chosen for fieldwork due to its accessibility, and the
    previously observed propagation of the mass loss of the Greenland Ice Sheet
    from the south to the northwest \citep{Khan.etal.2010}, although the more
    recent satellite gravimetry
    data shows that virtually all margins of the Greenland Ice Sheet are now
    loosing mass \citep{Groh.Horwath.2016}. The surface of Bowdoin Glacier
    is heavily crevassed but parts of it are accessible on feet
    (Fig.~\ref{fig:images}b and c). Crevasses are
    particularly few along a ca.~20\,m-wide medial moraine
    \citep[Fig.~68 of][]{Chamberlin.1897}, which can generally be walked down
    to the glacier front (Fig.~\ref{fig:images}a and b).

    \todo{Use Bowdoin Glacier's Greenlandic name?}

    Bowdoin Glacier was first explored (and so named) by western explorers in
    the late-nineteenth century, who
    reported a ``daily movement'' of 0.85\,m for July 1893 ``at the
    fastest point'' of the glacier \citep{Chamberlin.1894}.
    Photographs indicate that Bowdoin Glacier was thicker at that time, but its
    frontal position was only a few kilometers downstream from the present
    calving front (\citealp[p.~668]{Chamberlin.1895}; \citealp[Figs.~64 and~65
    of][]{Chamberlin.1897}; \citealp[Fig.~1 of][]{Podolskiy.etal.2016}) and has
    been relatively stable since then.

    However, a two-fold increase of surface velocity in the early 2000s was
    followed by a rapid frontal retreat of Bowdoin Glacier by ca.~2\,km between
    2007 and 2013 \citep[Fig.~2 of][]{Sugiyama.etal.2015}, a behaviour
    synchronous to that of other tidewater glaciers in the area
    \citep{Sakakibara.Sugiyama.2018}. Since 2013, the ice front appears once
    again stable, yet the terminal tongue continues to experience surface
    lowering at an alarming rate of ca.~4.1\,m\,a$^{-1}$, which is primarily
    the expression of continued dynamic thinning \citep{Tsutaki.etal.2016}.

    Bowdoin Glacier longitudinal strain and seismicity vary in response to
    tides~\citep{Podolskiy.etal.2016, Podolskiy.etal.2017} indicating a low
    basal drag and near plug-flow conditions \citep{Seddik.etal.2019}. Major
    icebergs calve according to a recurring pattern in which fractures
    propagate nearly parallel the to ice front \citep{Jouvet.etal.2017}.
    The emergence of subglacial meltwater forms several submarine plumes of
    highly
    variable surface imprint \citep{Jouvet.etal.2018} and entrains nutrients
    from the bottom to the surface of Bowdoin Fjord \citep{Kanna.etal.2018}.


% -- -- -- -- -- -- -- -- -- -- -- -- -- -- -- -- -- -- -- -- -- -- -- -
\subsection{Drilling sites}
% -- -- -- -- -- -- -- -- -- -- -- -- -- -- -- -- -- -- -- -- -- -- -- -

    During the field campaign in July 2014, three boreholes BH1 (272\,m deep),
    BH2 (262\,m), and BH3 (252\,m) were drilled with hot water in the marine
    (glacier bed below sea level) part of Bowdoin Glacier
    using meltwater from the crevasses. After the successful drilling of BH1
    and BH2, 7\,m apart, at the first (upper) drilling site, 2\,km from the
    calving front, it was planned that two other boreholes would be drilled
    upstream in the thicker part of the glacier. Due to unfavourable weather the
    hot water drilling equipment could not be transported upstream.
    Thus, a third
    and last hole was drilled 158\,m downstream of the first borehole site
    to be equipped with all instruments left (Figs.~\ref{fig:images}b and
    \ref{fig:boreholes}a).

    Although the experiment was originally planned for one year, some of the
    instruments were let on the glacier for up to three years as more funding
    was obtained and additional field campaigns were planned for parallel
    experiments. From the drilling in July 2014 to the last data retrieval in
    July 2017, the boreholes were displaced by 997 (BH1) to 1191\,m (BH3),
    their surface lowered from 89 (BH1, 2014) to 54\,m~above sea level
    (BH3, 2017) and
    the distance between the lower (BH3) and upper (BH1) drilling sites
    increased from 158 to 191\,m (Fig.~\ref{fig:boreholes}a). New crevasses
    appeared on the glacier surface, some causing damage to the instruments
    (Fig.~\ref{fig:images}c).


% -- -- -- -- -- -- -- -- -- -- -- -- -- -- -- -- -- -- -- -- -- -- -- -
\subsection{Instrumentation}
% -- -- -- -- -- -- -- -- -- -- -- -- -- -- -- -- -- -- -- -- -- -- -- -

    The boreholes were equipped with three types of sensors: strings of simple,
    regularly-spaced thermistors arranged to span the entire depth of the
    glacier, two piezometers near the base of the glacier, and digital
    inclinometer units at different depths. Besides their primary sensors, the
    piezometers and digital inclinometers were equipped with additional
    thermistors, so that ice temperature could be measured at multiple depths
    in the glacier. This manuscript focuses on the temperature data. The tilt
    data are used to estimate strain heating.

    At the first (upper) drilling site, one borehole (BH1) was equipped with
    seven digital inclinometers (Fig.~\ref{fig:boreholes}b, blue triangles) and
    the other (BH2) with one basal piezometer (Fig.~\ref{fig:boreholes}b,
    orange square) and two thermistor strings (Fig.~\ref{fig:boreholes}b,
    orange circles). At the second (lower) drilling site, the
    borehole (BH3) was equipped with five digital inclinometers
    (Fig.~\ref{fig:boreholes}b, green triangles), one basal piezometer
    (Fig.~\ref{fig:boreholes}b, green square) and two thermistor strings
    (Fig.~\ref{fig:boreholes}b, green and grey circles).

    Due to the relocation of the second drilling site the thermistor strings
    were re-arranged to fit a smaller ice thickness. However, the deeper
    thermistor string depict temperatures incompatible with those recorded by
    digital inclinometers in the same hole. The depths of digital inclinometers
    were calibrated from independent pressure sensors and are thus more robust
    than those of sensors on the thermistor string. Thus, we think that sensors
    on the thermistor string were misplaced due to a
    manipulation error and mark their positions and data as erratic (ERR,
    Fig.~\ref{fig:boreholes}b, grey circles). The error most likely occured
    when cables were re-arranged and taped together for the installation of
    all remaining instruments in a single (BH3) borehole
    instead of the two originally planned. This decision was made in light of
    the observed diminishing availability of surface meltwater for drilling in
    this part of the glacier.

    The basal piezometers (Geokon 4500) were connected to Campbell CR10X data
    loggers via Campbell AVW1 vibrating wire interfaces. The thermistor strings
    \citep[NTC~Fenwal 135-103FAG-J01,][]{Ryser.2014} were connected to Campbell
    CR1000 data loggers via Campbell AM416 relay multiplexers. The piezometers
    and thermistor strings data loggers were each powered by a 12\,V, 24\,Ah
    lead battery, and mounted in polyester cases on tetrapods
    (Fig.~\ref{fig:images}c). These batteries were recharged during the 2015
    and 2016 field campaigns using
    a petrol generator at the camp. For the thermistor strings, in addition to
    automatic measurements, manual readings were performed in 2015, 2016 and
    2017 using a hand-held ohmmeter.

    The digital inclinometers units \citep[DIBOSS,][]{Ryser.2014,
    Ryser.etal.2014, Ryser.etal.2014a} are equipped with VTI Technologies
    SCA103T inclinometers and iST~TSic~716 temperature sensors and
    were connected to CR1000 data loggers. Each data logger
    was set-up in a hard plastic case including three 12\,V, 65\,Ah lead
    batteries and a solar panel on the outside, and anchored to an aluminum
    pole drilled into the ice.
    In the case of digital inclinometers, the observed resistance from the
    thermistors were converted to temperature values in situ by the englacial
    digitizers, thus avoiding to record the resistance of the borehole cables
    and their potential variations due to cable deformation. The deployment and
    maintenance of the borehole installations was eased by the medial moraine
    (Fig.~\ref{fig:images}b).

    Finally, a dual frequency Global Positioning System (GPS) receiver was
    installed near the first borehole (BH1). A GPS antenna (JAVAD GrAnt-G3T)
    was mounted on an aluminium stake re-drilled every summer in the ice to
    accommodate melt. It was connected to a GPS receiver (GNSS Technology
    Inc. GEM-1) set-up in a hard plastic case and powered by an external ca
    30\,Ah lead battery and a solar panel. The same GPS instruments were
    installed near the camp site as a reference station for post-processing
    (Fig.~\ref{fig:boreholes}a, black triangle).


% -- -- -- -- -- -- -- -- -- -- -- -- -- -- -- -- -- -- -- -- -- -- -- -
\subsection{Experiment duration}
% -- -- -- -- -- -- -- -- -- -- -- -- -- -- -- -- -- -- -- -- -- -- -- -

    Of the total 44~sensors installed, 41~worked after the installation, 38
    recorded data for one year, 35 were still functional after two years (of
    which 16 were not connected to a data logger any longer but were used for a
    manual temperature reading), and only two were left
    working after three years. Contact to sensors was progressively lost due to a
    battery issue (2014 July~28 and again 2015 July~26, one sensor) cable
    damage during the installation (2014 July~16 and 23, three sensors) cable
    damage at depth (2014 Oct.~25, 2017 Jan.~28, Feb.~03, and June~21, five
    sensors), and most importantly, cable damage at the surface (2015 Nov.~12
    and unknown dates, 33~sensors, Table~\ref{tab:timeline},
    Fig.~\ref{fig:images}c).


% -- -- -- -- -- -- -- -- -- -- -- -- -- -- -- -- -- -- -- -- -- -- -- -
\subsection{Temperature calibration}
% -- -- -- -- -- -- -- -- -- -- -- -- -- -- -- -- -- -- -- -- -- -- -- -


    For the thermistor strings, the conversion from resistance to temperature
    is performed during post-processing following the Steinhart-Hart equation:
    %
    \begin{equation}
      \frac{1}{T} = a_0 + a_1 \log(R) + a_3 \log(R)^3,
    \end{equation}
    %
    where $R$ is the measured resistance, $T$ the ice temperature, and $a_0$,
    $a_1$, and $a_3$ are coefficients calibrated individually for each sensor
    prior to fieldwork in the lab for temperatures of -15, -12, -9, -6, -3 and
    0\,$^\circ$C (Table~\ref{tab:calibration}). The basal piezometers use
    temperature calibration coefficients published by the manufacturer
    (Table~\ref{tab:calibration}). The digital inclinometers use fully
    calibrated sensors delivering direct temperature measurements.

    \todo{R1 - more details on the digital temperature sensor are required
          (e.g. model, accuracy, resolution, calibration procedure).}

    Despite the pre-field calibration, the temperatures observed in the
    meltwater-filled borehole immediately after drilling were not at the
    pressure-melting point but instead up to 0.16\,$^\circ$C colder.
    As we can not exclude that the initial calibration was affected by the
    transport of the instruments to Greenland, an in-situ recalibration
    recalibration is applied in post-processing by correcting for the initial
    temperature offset to the pressure melting-point, $\Delta
    T$ (Table~\ref{tab:calibration}). Unfortunately some of these initial
    temperature data have been lost so that the recalibration was possible for
    some sensors only. The pressure-melting point was computed as
    %
    \begin{equation}
      T_m = -\beta \rho g z,
    \end{equation}
    %
    where $\beta$ is the Clausius-Clapeyron constant,
    $\rho$ is the ice density, $g$ is the standard
    acceleration due to gravity, and $z$ is the depth below the ice surface
    (parameter values given in Table~\ref{tab:parameters}).
    Because the Clausius-Clapeyron constant may be affected by the meltwater
    content and impurities, we use a value determined in similar condition for
    another water-filled borehole in Greenland \citep[][]{Luthi.etal.2002}.


% ----------------------------------------------------------------------
\section{Results}
% ----------------------------------------------------------------------

% -- -- -- -- -- -- -- -- -- -- -- -- -- -- -- -- -- -- -- -- -- -- -- -
\subsection{Temperature time series}
% -- -- -- -- -- -- -- -- -- -- -- -- -- -- -- -- -- -- -- -- -- -- -- -

    After the drillings, some sensors in BH2 record hourly air temperature
    variations as they are temporarily located above the borehole water level.
    However, most sensors are immersed and temperatures are measured at or near
    the pressure-melting point (Table~\ref{tab:calibration}). For most sensors
    temperatures then drop off the pressure-melting point and follow an S-curve
    before stabilizing at -0.25 to -6.04\,$^\circ$C
    (Fig.~\ref{fig:timeseries}; \citealp[cf. Fig.~3.6 of][]{Ryser.2014}). This
    initial phase lasts for hours to months
    depending on the sensor and relating to the equilibrium temperature. While
    most records reach a thermal equilibrium within three months, sensors LT03
    and LT02 recorded freezing after 11 and 15 months, reaching respective
    minimum temperatures of -0.35 and -0.25\,$^\circ$C
    (Fig.~\ref{fig:timeseries}).

    The sensors nearest to the ice surface exhibit a seasonal temperature
    cycle, which is out-of-phase with the atmospheric temperature cycle
    (Fig.~\ref{fig:timeseries}, lowermost curves). The
    amplitude increases with time as the thermistor strings progressively
    melt-out towards the glacier surface, and may be affected by sunlight
    penetrating the ice. Temperature records for sensors installed deeper
    down in the ice do not show a seasonal cycle, but many exhibit a slow
    warming trend of 0.1 to 0.4\,$^\circ$C\,a$^{-1}$
    (Fig.~\ref{fig:timeseries}).

    Manual temperature readings are generally compatible with the automatic
    records (Fig.~\ref{fig:timeseries}, filled circles), but some of these
    measurement are off by a few degrees, perhaps due to wet connectors
    (Fig.~\ref{fig:timeseries}, empty circles). Manual readings were also
    performed in 2017 but yielded values well-off the expected range, which
    is certainly due to surface cable damage visibly caused by opening crevasses
    (Fig.~\ref{fig:images}b).


% -- -- -- -- -- -- -- -- -- -- -- -- -- -- -- -- -- -- -- -- -- -- -- -
\subsection{Temperature profiles}
% -- -- -- -- -- -- -- -- -- -- -- -- -- -- -- -- -- -- -- -- -- -- -- -

    After the refreezing of the entire boreholes, vertical profiles depict
    temperatures below the pressure-melting point except for the base of the
    glacier where temperatures reach the pressure-melting point
    (Fig.~\ref{fig:profiles}a, Table~\ref{tab:temperature}). A thin layer of
    temperate ice possibly exists near the base
    but it is not revealed by the resolution of our measurements. The coldest
    temperatures of -6.03 (BH1), -5.98 (BH2) and -4.10\,$^\circ$C (BH3)
    are reached in the middle part of the ice column. Both profiles exhibit a
    subsurface layer with warmer temperatures up to -3.74 (BH2) and -1.35\,$^\circ$C
    (BH3), and a surface layer with
    seasonal temperature variations (Fig.~\ref{fig:profiles}a).

    The upper boreholes, BH1 and BH2, which are only separated by seven metres,
    show compatible temperature profiles with differences within 0.52\,$^\circ$C
    (Fig.~\ref{fig:profiles}a, blue and
    orange lines). For the lower borehole, the deepest thermistor string (ERR,
    Fig.~\ref{fig:profiles}a, grey lines) depict temperatures incompatible
    with those recorded by digital inclinometers in the same hole
    (BH3, Fig.~\ref{fig:profiles}a, green lines), whose depths were calibrated
    using independent pressure sensors.

    Remarkably, the upper (BH1, BH2) and lower (BH3, ERR) drilling sites are
    located on the same flow line and separated by only 158 (2014) to 191\,m
    (2017) but show temperature differences around 2\,$^\circ$C prevailing over
    the entire glacier depth.  BH1 and BH3 are separated by 158 (2014) to
    191\,m and show an average temperature difference of 1.81\,$^\circ$C and a
    maximum of 2.11\,$^\circ$C after interpolation over the lower part of the
    ice column where BH1 data are available (Fig.~\ref{fig:profiles}a, blue and
    green lines). BH2 and BH3 are separated by 165 (2014) to 197\,m (2017) and
    show an average temperature difference of 2.07\,$^\circ$C and a maximum of
    2.86\,$^\circ$C after interpolation (Fig.~\ref{fig:profiles}a, orange and
    green lines).

    All three temperature profiles show a general warming trend, except for
    their base (Figs.~\ref{fig:profiles}a and~\ref{fig:profiles}b, solid
    lines). The warming trend of the upper drilling site (BH1, BH2) is
    within 0.1\,$^\circ$C\,a$^{-1}$ except for the upper part where a maximum
    warming of 0.21\,$^\circ$C\,a$^{-1}$ is observed (Fig.~\ref{fig:profiles},
    blue and orange lines). On the other hand, the lower (BH2, ERR) drilling
    site experiences significant englacial
    warming over much of the ice column with a maximum rate of
    0.39\,$^\circ$C\,a$^{-1}$ (Fig.~\ref{fig:profiles}, green and grey lines).


% ----------------------------------------------------------------------
\section{Discussion}
% ----------------------------------------------------------------------

% -- -- -- -- -- -- -- -- -- -- -- -- -- -- -- -- -- -- -- -- -- -- -- -
\subsection{Theoretical warming}
% -- -- -- -- -- -- -- -- -- -- -- -- -- -- -- -- -- -- -- -- -- -- -- -

    The theoretical englacial warming due to vertical heat diffusion and
    viscous dissipation can be expressed by the temperature evolution equation,
    %
    \begin{equation}
      \rho c \frac{\partial T}{\partial t}
        = k \frac{\partial^2 T}{\partial z^2} + H,
    \end{equation}
    %
    where $T$ is the ice temperature, $\rho$ the ice density, $k$ the
    thermal conductivity of ice, and $c$ its specific heat capacity
    (Table~\ref{tab:parameters}). The source term, $H$, corresponds to the
    energy dissipation due to strain heating, which can be expressed as
    %
    \begin{equation}
      H = \mathrm{tr}(\bm{\tau\dot\epsilon}),
    \end{equation}
    %
    where $\bm\tau$ is the deviatoric stress tensor, and $\bm{\dot\epsilon}$ the
    strain-rate tensor \citep[p.~417]{Clarke.etal.1977, Cuffey.Paterson.2010}.
    Stresses and strains can be related by the constitutive law for ice
    \citep{Glen.1952, Nye.1953},
    %
    \begin{equation}
        \bm{\dot\epsilon} = A\,\tau_\mathrm{e}^{n-1}\,\bm{\tau} \,,
    \end{equation}
    %
    where the effective stress, $\tau_\mathrm{e}$, is defined by
        ${\tau_\mathrm{e}}^2 = \frac{1}{2} \mathrm{tr}(\bm\tau^2)$.
    The ice softness coefficient, $A$, depends on the ice temperature, $T$, and
    depth below the surface, $z$, through an Arrhenius-type law,
        ${A = A_0 \exp[
            \frac{-Q}{R}(\frac{1}{T_\mathrm{pa}}-\frac{1}{T_\mathrm{th}})]}$,
    where $T_\mathrm{pa}$ is the pressure-adjusted ice temperature calculated
    using the Clapeyron relation,
        ${T_\mathrm{pa} = T + \beta \rho g z}$,
    and $T_\mathrm{th}$ a temperature threshold defined by
        ${T_\mathrm{th} = 263 + \beta \rho g z}$
    \citep[p.~72]{Cuffey.Paterson.2010}. The strain heating can then be
    rewritten as a function of either the deviatoric stresses or the strain
    rates only,
    %
    \begin{equation}
        H = 2 A \tau_\mathrm{e}^{n+1}
          = 2 A^{-1/n} \dot\epsilon_\mathrm{e}^{1+1/n}.
    \end{equation}

    Assuming deformation within a two-dimensional cross-section in the
    vertical, $z$, and horizontal along-flow, $x$, dimensions, the effective
    strain rate,
        ${\dot\epsilon_\mathrm{e}^2
          = \frac{1}{2}\mathrm{tr}(\bm{\dot\epsilon}^2)}$,
    can be expressed in terms of its Cartesian components,
        ${\dot\epsilon_\mathrm{e}^2
          = \frac{1}{2}(\dot\epsilon_{xx}^2 + \dot\epsilon_{zz}^2)
          + \dot\epsilon_{xz}^2}$.
    Incompressibility yields
        ${\mathrm{div}(\bm{\dot\epsilon})
          = \dot\epsilon_{xx} + \dot\epsilon_{zz}
          = 0}$
    so that the effective strain can be simplified to a function of its
    longitudinal and shear components,
    %
    \begin{equation}
        \dot\epsilon_\mathrm{e}^2 = \dot\epsilon_{xx}^2 + \dot\epsilon_{xz}^2.
    \end{equation}

    The longitudinal strain rate can be estimated from the observed borehole
    positions. The distance between BH1 and BH3
    has increased from 158 to 191\,m between 2014 July 20 (midpoint
    between the two observation dates 17 and 23) and 2017 July 17. This
    corresponds to a longitudinal strain rate, $\dot\epsilon_{xx}$, of
    $2.01\times10^{-9}$\,s$^{-1}$. Measured horizontal shear strain rates,
    $\dot\epsilon_{xz}$, average to $3.66\times10^{-9}$ (BH1, also used for
    nearby BH2) and $2.97\times10^{-9}$\,s$^{-1}$ (BH3), yielding an effective
    strain rate, $\dot\epsilon_\mathrm{e}$, of respectively $4.17\times10^{-9}$ and
    $3.59\times10^{-9}$\,s$^{-1}$.

    Heat diffusion was approximated by $z$-central, time-explicit differences
    on the first temperature profile. At the base of the profile, where
    some sensors take several months to refreeze after the boreholes were
    drilled, cooling is observed (Fig.~\ref{fig:profiles}b, solid lines) which
    exceeds the theoretical temperature evolution (Fig.~\ref{fig:profiles}b,
    dash-dotted lines). This discrepancy could be explained by the late
    refreezing and continued slow cooling of some sensors towards an equilibrium
    with the surrounding ice, enhanced vertical diffusion due to basal melt
    affecting the distance between sensors, or cable stretching stretching
    yielding increased cable lenght and reduced diameter. Both contribute to an
    increased cable resistance, yielding decreased apparent temperature.

    Over much of the temoperature profile, the observed warming trend
    (Fig.~\ref{fig:profiles}b, solid lines) is up to an order of magnitude
    higher than the theoretical warming resulting from viscous
    dissipation and heat diffusion (Fig.~\ref{fig:profiles}b, dash-dotted
    lines), indicating the presence of another heat source within the glacier
    during the measurements period.


% -- -- -- -- -- -- -- -- -- -- -- -- -- -- -- -- -- -- -- -- -- -- -- -
\subsection{Latent heat}
% -- -- -- -- -- -- -- -- -- -- -- -- -- -- -- -- -- -- -- -- -- -- -- -

    The ice temperatures recorded in Bowdoin Glacier are generally higher than
    for other outlet glaciers of the Greenland ice sheet \citep{Iken.etal.1993,
    Luthi.etal.2002, Luthi.etal.2015, Harrington.etal.2015}. This is most
    likely related to the smaller catchment size and relatively low elevation
    of the Bowdoin Glacier accumulation area than for the outlet glaciers
    of central western Greenland where measurements have been made so far.
    Nevertheless, the main annual air temperature at Qaanaaq airport,
    ca.~30\,km south-west and seawards from the drilling sites, from 2005 to
    2015, is -8.5\,$^\circ$C \citep{Sugiyama.etal.2014, Tsutaki.etal.2017}. The
    ice temperatures of -3.74 (BH2) and -1.35\,$^\circ$C measured in Bowdoin
    Glacier below the penetration depth of seasonal variations
    (Fig.~\ref{fig:profiles}a) are significantly higher than this value.
    Similarly on Hansbreen, a tidewater glacier in northern Spitsbergen, ice
    temperatures recorded 10\,m below the ice surface and 2 to 3\,$^\circ$C
    higher than the mean annual air temperature have been explained by latent
    meltwater refreezing \citep{Jania.etal.1996}.

    Longitudinal variations in ice temperature have previously been observed in
    Arctic tidewater glaciers. In Sermeq Avannarleq, a tidewater glacier in
    western Greenland, temperature differences up to 5\,$^\circ$C were measured
    between two boreholes only 86\,m apart down to a depth of ca.~300\,m
    \citep{Luthi.etal.2015}.
    Such temperature differences have been explained by the release of latent
    heat from meltwater refreezing in crevasses, a process sometimes called
    cryo-hydrologic warming \citep{Phillips.etal.2010}. Year after year,
    surface meltwater penetrates in crevasses during summer and refreeze
    throughout the year, generating latent heat that diffuses in the
    glacier, potentially reducing ice viscosity and enhancing ice flow
    \citep{Phillips.etal.2013}. However, such temperature variations have so
    far never been reported to extend to the full depth of a glacier.

    Nevertheless, the temperature profiles at the upper and lower drilling
    sites of Bowdoin Glacier show significant differences over the entire depth
    of the glacier (Fig.~\ref{fig:profiles}a). If these differences were a
    relict advected from upstream areas of Bowdoin Glacier, the longitudinal
    diffusion of temperature should cause both profiles to evolve towards more
    similar temperatures. However, the temperature difference between the two
    profiles is actually increasing over time (Fig.~\ref{fig:profiles}b).
    Thus, the different temperature profiles and warming trend can only be
    explained by involving a spatially localized source of latent heat,
    extending over the entire or nearly entire depth of the glacier.


% -- -- -- -- -- -- -- -- -- -- -- -- -- -- -- -- -- -- -- -- -- -- -- -
\subsection{Deep crevassing}
% -- -- -- -- -- -- -- -- -- -- -- -- -- -- -- -- -- -- -- -- -- -- -- -

    The penetration of meltwater to the bed of Bowdoin Glacier is evident from
    the subglacial discharge of ice-dammed lakes, the formation of sediment
    plumes at the calving front \citep{Jouvet.etal.2018, Kanna.etal.2018}, and
    diurnal speed variations \citep{Sugiyama.etal.2014, Podolskiy.etal.2016}.
    Although locally warmer ice temperatures could be explained by the
    proximity of a moulin, no large moulins have been observed in the vicinity
    of the borehole sites or elsewhere on Bowdoin Glacier during the field
    campaigns. Smaller moulins with apparent diameters of ca.~1\,m have
    occasionally been observed on the glacier. However, small englacial
    conduits would be unsustainable in the middle part of the ice column where
    ice temperatures are well below the melting point and the
    ca.~10\,cm-diameter borehole was observed to refreeze within a few days.
    (Fig.~\ref{fig:timeseries}). They would
    most likely refreeze over the winter and thus could not explain the
    observed continuous warming trend nor the 2\,$^\circ$C difference between
    the two drilling sites, a result of several years of englacial warming
    according to the observed rate of up to 0.39\,$^\circ$C\,a$^{-1}$.

    As a tidewater glacier, Bowdoin Glacier is subject to low basal friction
    \citep{Seddik.etal.2019} and
    thus continuous longitudinal extension yielding to the formation of
    numerous surface transverse crevasses (Figs.~\ref{fig:images} and
    \ref{fig:boreholes}). Surface GPS records indicate that the longitudinal
    extension is most important at lowering tide and correlated with intense
    seismic activity most likely symptomatic of crevasse opening
    \citep{Podolskiy.etal.2016, Podolskiy.etal.2017}.
    The Bowdoin Glacier boreholes were drilled in a highly crevassed area in
    2014 and newly opened crevasses could be observed as the drilling sites
    were
    advected downstream and drifted further apart over the subsequent field
    seasons (Fig.~\ref{fig:images}b).

    Full-depth crevasse propagation through kilometre-thick ice has already
    been observed in Greenland in association with the drainage of a
    supraglacial
    lake \citep{Das.etal.2008}. After crevasse initiation, fracture propagation
    to the bed of the glacier is theoretically possible if enough water is
    supplied \citep{Veen.2007}. This condition is met on a tidewater glacier,
    where crevasses can be expected to remain water filled at least up to sea
    level even after connecting to the subglacial drainage system. If
    surrounded by cold ice, meltwater filling the crevasses would progressively
    refreeze generating latent heat. Therefore, we interpret the englacial
    warming observed at Bowdoin Glacier to relate to meltwater refreezing in
    deep crevasses reaching to or near the glacier bed.


% -- -- -- -- -- -- -- -- -- -- -- -- -- -- -- -- -- -- -- -- -- -- -- -
\subsection{Ogive banding}
% -- -- -- -- -- -- -- -- -- -- -- -- -- -- -- -- -- -- -- -- -- -- -- -

    Overprinted on the heavy crevassing pattern (Fig.~\ref{fig:ogives}a), the
    topography of Bowdoin Glacier exhibits surface undulations transverse to
    the flow direction. This banding is best visualized on low-solar angle
    satellite images (Fig.~\ref{fig:boreholes}a) or shaded relief images
    (Fig.~\ref{fig:ogives}b). The undulations are advected by the movement of
    the glacier, and thus appear to be related to spatial variations in
    ice thickness rather than reflecting a pattern in the bed topography
    \citep[Fig.~\ref{fig:ogives}c; Fig.~3 of][]{Tsutaki.etal.2016}. They have a
    wavelength of ca.~350\,m and an amplitude of ca.~10\,m
    (Fig.~\ref{fig:ogives}d). The undulations seem to originate from a
    steeper part of the glacier ca.~8\,km upstream from the calving front
    immediately above the confluence zone of Bowdoin and Obelisk Glaciers
    \citep[Fig.~\ref{fig:images}a; Fig.~3 of ][]{Tsutaki.etal.2016}, and
    could thus be considered as some kind of ogives.

    The upper borehole site (BH1, BH2) is located on a topographic high and the
    lower borehole site (BH3) is located in a topographic low
    (Fig.~\ref{fig:ogives}a). Although this
    was not at all foreseen, the distance between the two boreholes is roughly
    equal to half the ogive wavelength. Because the surface topography of
    Bowdoin Glacier is overprinted by a dense crevasse pattern and abundant
    smaller-scale topography (Figs.~\ref{fig:images} and~\ref{fig:ogives}a),
    the undulations are not obvious in the field. But in fact it
    could be observed during field seasons that the upper drilling site offered
    a more extensive view than the lower drilling site.

    The lower borehole (BH3), located in a topographic dip, exhibits a warmer
    and faster-warming temperature profile than the upper boreholes (BH1, BH2),
    located on a topographic high. In light of the above observations we
    speculate that topographic dips may tend to localise deep
    crevassing. This localisation of crevassing
    could be the result of mechanical damage due to thinner ice, increased
    meltwater infiltration ponding between ogives, or a combination of these
    two processes.


% ----------------------------------------------------------------------
\section{Conclusions}
% ----------------------------------------------------------------------

    Following hot water drilling of the highly crevassed terminal tongue of
    Bowdoin Glacier in northwestern Greenland, we present the first continuous,
    multi-year temperature record from one of many marine-terminating outlet
    glaciers of the Greenland Ice Sheet, which currently experience a
    generalised thinning and accelerated retreat. From this record we identify
    potential new mechanisms that govern the englacial temperature of tidewater
    glaciers and thereby their viscosity.

    \begin{itemize}

      \item Temperature profiles from two drilling sites separated by 158
        (2014) to 191\,m (2017) differ by up to ca.~2\,$^\circ$C indicating
        strong, full-depth longitudinal temperature variations in the glacier.

      \item Englacial warming up to 0.39$^\circ$C\,a$^{-1}$, an order of
        magnitude above the theoretical warming from heat diffusion and viscous
        dissipation, indicates a deep and local heat source within the
        tidewater glacier.

      \item In the absence of visible moulins on the glacier surface, we interpret
        these results as the expression of latent heat released from meltwater
        refreezing in crevasses reaching to, or near to, the base of the glacier.

      \item We speculate that the localisation of such deep crevasses may be
        controlled by preferential mechanical damage and meltwater infiltration
        in thinner parts of the glacier associated with ogive banding.
        \todo{R2 - I am not sure... Check vs Nye crevasse theory.}

    \end{itemize}

    These results are a somewhat fortuitous conclusion of a last-minute relocation
    of the Bowdoin Glacier second drilling site due to unfavourable weather.
    They are limited by the number of sampling points (two), and need to
    be validated through a more systematical experiment. However, our
    measurements indicate
    a potential new mechanism for full-depth englacial warming and local
    ice softening that may contribute to the presently observed dynamic
    thinning of Greenlandic tidewater glaciers.


% ----------------------------------------------------------------------
% Acknowledgements
% ----------------------------------------------------------------------

\paragraph{Acknowledgements}

    We would like to thank Toku Oshima and Kim Petersen for their warm welcome
    in Qaanaaq, for the shooting lessons and for assistance with field
    preparations. Many thanks to Takanobu Sawagaki, Naoki Katayama, Jun Saito,
    and Shun Tsutaki for their participation in the drilling and
    to Evgeniy Podolskiy and Lukas Preiswerk for
    their precious help retrieving data from instruments and instruments from
    crevasses. We thank Martin Lüthi for his constructive comments and great
    help to interpret the data, and Eef van Dongen for insightful discussions
    and her help proofreading this manuscript.
    The current work was supported by the Swiss National Science Foundation
    grants no.~200020-169558 and 200021-153179/1 to M.~Funk and the by Japanese
    Ministry of Education, Culture, Sports, Science of Technology through
    the GRENE Arctic Climate Change Research Project and Arctic Challenge for
    Sustainability (ArCS) project. Publication fees were paid by ETH Zurich.

\paragraph{Author contributions}

    MF, SS and AB set-up the Bowdoin Glacier project and first field campaign.
    TW and CS assembled and calibrated the borehole instruments. MF, SS, AB and
    TW organized the first fieldwork and drilled the boreholes. SS processed
    the DGPS data. JS maintained the stations, processed the borehole data and
    wrote most of the manuscript.

\paragraph{Conflict of interest}

    The authors declare that the research was conducted in the absence of any
    commercial or financial relationships that could be construed as a
    potential conflict of interest.

\paragraph{Contribution to the field}

    All around Greenland, glaciers flowing into the ocean react more strongly
    to climate change than the rest of the ice sheet. Because ice is lighter
    than sea water, when these glaciers thin, more and more of their weight is
    taken up by the ocean, allowing them to flow faster. A critical factor in
    the physics of fast glacier flow is the ice temperature. Similarly to honey
    or melting plastic, the viscosity of ice depends strongly on its
    temperature. Warm ice flows faster than cold ice.
    For the first time, we present continuous measurements of ice
    temperature from the fast-flowing and highly crevassed terminal part of a
    Greenlandic marine-terminating glacier. The measurements indicate the
    presence of a deep, local heat source within the glacier. Our
    interpretation is that every summer, meltwater penetrates deep into the
    glacier through crevasses that extend from the surface to the bottom or
    nearly the bottom of the glacier. The zero-degree water then refreezes in
    winter, warming up the surrounding ice and thus making it softer.
    Our results shed light on a new mechanism contributing to the observed
    acceleration of marine-terminating glaciers in Greenland.

\paragraph{Data availability}

    The Bowdoin Glacier temperature data will be made available in a public
    repository upon publication of this manuscript.


% ----------------------------------------------------------------------
% References
% ----------------------------------------------------------------------

\bibliographystyle{frontiersinSCNS_ENG_HUMS}
\bibliography{../../references/references}


% ----------------------------------------------------------------------
% Figures
\clearpage
% ----------------------------------------------------------------------

    \begin{figure}
      \centerline{\includegraphics{bowtem_images}}
      \caption{%
        \textbf{(a)} Bowdoin Glacier panoramic view on 2015 July 17. The
          calving front is ca.~3\,km wide and 50\,m high. Note the steeper
          section of the glacier above the confluence of Obelisk Glacier
        \textbf{(b)} Bowdoin Glacier lower drilling site and BH3 data loggers
          for thermistor strings and the basal piezometers on 2016 July~19. The
          data logger for the digital inclinometers was removed after cable
          damage related to the opening of the crevasse pictured.
        \textbf{(c)} Aerial view of the drilling sites on 2016 July~21. The
          location of BH1 and BH3 could be inferred from visible field
          installations.}
      \label{fig:images}
    \end{figure}

    \begin{figure}
      \centerline{\includegraphics{bowtem_boreholes}}
      \caption{%
        \textbf{(a)} Bowdoin borehole locations from drilling in July 2014 to
          dismantling in July 2017 and background satellite image from 2017
          Mars 10, 17:41:29 UTC. Contains modified Copernicus Sentinel data,
          processed with Sentinelflow.
        \textbf{(b)} Initial ice thickness and sensor depths. Inclinometers
          and piezometers are also equipped with thermistors.}
      \label{fig:boreholes}
    \end{figure}

    \begin{figure}
      \centerline{\includegraphics{bowtem_timeseries}}
      \caption{%
        Ice temperature time series from the three boreholes (BH1, BH2, and
        BH3) including BH3 erratic data (ERR, see text) from the drilling in
        July 2014 to the last data retrieval
        in July 2017 (solid lines), and manual temperature readings (circles,
        with outliers marked as empty circles, and 2017 values off the graph).
        Field campaigns (orange spans) and dates selected for
        temperature profiles (dashed lines; Fig.~\ref{fig:profiles}) are
        indicated.}
      \label{fig:timeseries}
    \end{figure}

    \begin{figure}
      \centerline{\includegraphics{bowtem_profiles}}
      \caption{%
        \textbf{(a)} Temperature profiles for selected dates
          (see Fig.~\ref{fig:timeseries}) ca.~2 to 5 months after the drilling
          (solid lines) and towards the end of the available record (dashed
          lines), for the three boreholes (BH1, BH2, and BH3) including BH3
          erratic data (ERR, see text). The second profile from BH2 (orange
          dashed lines) corresponds to manual readings. Other values are daily
          means from the automated record. Borehole initial basal depths (thin
          horizontal lines) and minimum observed non-seasonal temperatures are
          indicated.
        \textbf{(b)} Observed rate of change between the two first profiles
          (solid lines) and theoretical change due to heat diffusion and
          viscous dissipation (dash-dotted lines) for the corresponding time
          period. The maximum observed warming is indicated. All curves were
          interpolated using cubic splines.}
      \label{fig:profiles}
    \end{figure}

    \begin{figure}
      \centerline{\includegraphics{bowtem_ogives}}
      \caption{%
        \textbf{(a)} Glacier surface topography in the vicinity of the
          boreholes sites in the early phase of the experiment (Arctic DEM,
          2014 Sep.~5) and corresponding locations of the boreholes. The
          locations of BH2 and BH3 are estimated from their initial locations
          in July 2014 (color crosses) and the displacement of BH1 measured by
          continuous GPS during the corresponding time intervals.
        \textbf{(b)} Multiple illumination shaded relief map of the terminal
          tongue of Bowdoin Glacier from the same data.
        \textbf{(c)} Elevation change between 2014 Sep.~5 and 2016 Apr.~24
          (Arctic DEM) showing generalised thinning, local thinning near the
          calving front, and the advection of glacier surface undulations.
        \textbf{(d)} Glacier surface topographic profile along a flow line
          passing near BH1 and the estimated location of BH3.}
      \label{fig:ogives}
    \end{figure}


% ----------------------------------------------------------------------
% Tables
\clearpage
\linespread{1.25}  % allows two tables per page
% ----------------------------------------------------------------------

    \begin{table}
      \caption{%
        Timeline of events including borehole drilling locations and initial
        depths, instrumental failures, and final removal (dates as
        YYYY-MM-DD).}
      \label{tab:timeline}
      \centerline{\begin{tabular}{lccp{95mm}}
        \hline
        Date       & Hole & Type      & Event \\
        \hline
        2014-07-15 & ---  & Campaign & \emph{2014 summer fieldwork begins} \\
        2014-07-16 & BH1  & Drilling & (77.691244$^\circ$N, 68.555749$^\circ$W,
                                       88.7\,m elevation, 272\,m depth)\newline
                                       Contact to the deepest inclinometer
                                       (UI01) is lost. \\
        2014-07-17 & BH2  & Drilling & (77.691307$^\circ$N, 68.555685$^\circ$W,
                                       87.7\,m elevation, 262\,m depth)\\
        2014-07-20 & BH2  & Incident & Earlier upper thermistor (UT*) data are
                                       lost accidentally. \\
        2014-07-23 & BH3  & Drilling & (77.689995$^\circ$N, 68.558857$^\circ$W,
                                       83.4\,m elevation, 252\,m depth)\newline
                                       Contact to the deepest inclinometers
                                       (LI01, LI02) is lost. \\
        2014-07-28 & BH2  & Battery  & The upper piezometer data logger battery
                                       drains prematurely without notice from
                                       the fieldwork participants. \\
        2014-07-29 & ---  & Campaign & \emph{2014 summer fieldwork ends} \\
        \hline
        2014-10-25 & BH1  & Incident & Contact to the deepest inclinometers
                                       (UI02, UI03) is lost. \\
        \hline
        2015-07-06 & ---  & Campaign & \emph{2015 summer fieldwork begins} \\
        2015-07-09 & BH2  & Upgrade  & The upper piezometer (UP) data logger
                                       gets a new battery. \\
        2015-07-14 & BH2  & Removal  & The upper thermistor strings (UT*) data
                                       logger is removed. \\
        2015-07-20 & ---  & Campaign & \emph{2015 summer fieldwork ends} \\
        \hline
        2015-07-26 & BH2  & Battery  & The new battery for the upper piezometer
                                       (UP) drains again prematurely, perhaps
                                       indicating faulty instruments. \\
        2015-11-12 & BH3  & Incident & Contact to all remaining inclinometers
                                       (LI03, LI04, LI05) is lost due to
                                       a crevasse opening on the surface. \\
        \hline
        2016-07-04 & ---  & Campaign & \emph{2016 summer fieldwork begins} \\
        2016-07-19 & BH3  & Removal  & The lower thermistor strings (LT*) data
                                       logger is removed. \\
        2016-07-21 & ---  & Campaign & \emph{2016 summer fieldwork ends} \\
        \hline
        2017-01-28 & BH1  & Incident & Contact to the deepest inclinometer
                                       (UI06) is lost. \\
        2017-02-03 & BH3  & Incident & The lower piezometer (LP) starts
                                       recording nonsense. \\
        2017-06-21 & BH1  & Incident & Contact to the deepest inclinometer
                                       (UI07) is lost. \\
        ?          & BH2  & Incident & Contact to the thermistor strings (UT*)
                                       is lost. \\
        ?          & BH3  & Incident & Contact to the thermistor strings (LT*)
                                       is lost. \\
        \hline
        2017-07-04 & ---  & Campaign & \emph{2017 summer fieldwork begins} \\
        2017-07-12 & BH1  & Removal  & The upper inclinometer (UI*) data logger
                                       is removed. \\
        2017-07-17 & ---  & Campaign & \emph{2017 summer fieldwork ends} \\
        \hline
      \end{tabular}}
    \end{table}

    \begin{table}
      \centerline{\begin{minipage}{85mm}
        \caption{%
          Temperature calibration coefficients and melt offset corrections.}
        \label{tab:calibration}
        \begin{tabular}{ccccc} \\\\\\
          \hline
          Sensor & $a_0\times10^{3}$ & $a_1\times10^{4}$
                 & $a_3\times10^{7}$ & $\Delta T$ (K) \\
          \hline
          LI03 &   ---   &   ---   &   ---   & -0.37 \\
          LI04 &   ---   &   ---   &   ---   &  0.15 \\
          LI05 &   ---   &   ---   &   ---   &  0.08 \\
          LP   & 1.4051  & 2.369   & 1.019   &  0.01 \\
          LT01 & 2.68849 & 2.91312 & 2.90792 & -0.19 \\
          LT02 & 2.72120 & 2.77064 & 6.36107 & -0.11 \\
          LT03 & 2.72538 & 2.75376 & 6.65425 & -0.17 \\
          LT04 & 2.72203 & 2.75468 & 6.66331 & -0.16 \\
          LT05 & 2.72629 & 2.74648 & 6.70740 & -0.10 \\
          LT06 & 2.71837 & 2.77063 & 6.77034 & -0.09 \\
          LT07 & 2.71239 & 2.80048 & 5.56011 & -0.11 \\
          LT08 & 2.71464 & 2.79042 & 5.78066 & -0.11 \\
          LT09 & 2.71831 & 2.78497 & 5.80214 & -0.12 \\
          LT10 & 2.73324 & 2.72923 & 7.50070 & -0.12 \\
          LT11 & 2.72182 & 2.77915 & 5.88847 & -0.08 \\
          LT12 & 2.71306 & 2.80286 & 5.33636 & -0.09 \\
          LT13 & 2.75088 & 2.66886 & 8.65715 & -0.10 \\
          LT14 & 2.71922 & 2.79321 & 5.78862 & -0.13 \\
          \hline
        \end{tabular}
      \end{minipage}
      \begin{minipage}{85mm}
        \begin{tabular}{ccccc}
          \hline
          Sensor & $a_1\times10^{3}$ & $a_2\times10^{4}$
                 & $a_3\times10^{7}$ & $\Delta T$ (K) \\
          \hline
          UI02 &   ---   &   ---   &   ---   & -0.34 \\
          UI03 &   ---   &   ---   &   ---   &  ---  \\
          UI04 &   ---   &   ---   &   ---   & -0.16 \\
          UI05 &   ---   &   ---   &   ---   &  ---  \\
          UI06 &   ---   &   ---   &   ---   &  ---  \\
          UI07 &   ---   &   ---   &   ---   &  ---  \\
          UP   & 1.4051  & 2.369   & 1.019   &  0.06 \\
          UT01 & 2.62948 & 3.15882 &-2.11729 &  0.01 \\
          UT02 & 2.70630 & 2.83255 & 4.85636 &  0.07 \\
          UT03 & 2.71054 & 2.85092 & 4.36997 &  0.02 \\
          UT04 & 2.70327 & 2.85200 & 4.38303 &  ---  \\
          UT05 & 2.70416 & 2.83808 & 4.40717 &  ---  \\
          UT06 & 2.67961 & 2.96215 & 1.95166 &  ---  \\
          UT07 & 2.71620 & 2.79459 & 5.60670 &  ---  \\
          UT08 & 2.70382 & 2.83626 & 4.52513 &  ---  \\
          UT09 & 2.72032 & 2.78078 & 5.86472 &  ---  \\
          UT10 & 2.71264 & 2.82098 & 4.98119 &  ---  \\
          UT11 & 2.70714 & 2.83545 & 4.61922 &  ---  \\
          UT12 & 2.69855 & 2.87227 & 3.95361 &  ---  \\
          UT13 & 2.70676 & 2.84831 & 4.31297 &  ---  \\
          UT14 & 2.73315 & 2.73929 & 7.08468 &  ---  \\
          UT15 & 2.70659 & 2.83499 & 4.69136 &  ---  \\
          UT16 & 2.71143 & 2.82174 & 5.01209 &  ---  \\
          \hline
        \end{tabular}
      \end{minipage}}
    \end{table}

    \begin{table}
      \caption{%
        Parameter values used to compute the theoretical englacial warming.}
      \label{tab:parameters}
      \centerline{\begin{tabular}{llrlll}
        \hline
        Not.    & Name & Value & Unit & Source \\
        \hline
        $A_0$   & Ice hardness coefficient
                & $3.5\times10^{-25}$
                & Pa$^{-3}$\,s$^{-1}$
                & \citet[p.~74]{Cuffey.Paterson.2010} \\
        $\beta$ & Clausius-Clapeyron constant
                & $7.9\times10^{-8}$
                & K\,Pa$^{-1}$
                & \citet{Luthi.etal.2002} \\
        $c$     & Ice specific heat capacity
                & 2097
                & J\,kg$^{-1}$\,K$^{-1}$
                & \citet[p.~400]{Cuffey.Paterson.2010} \\
        $g$     & Standard gravity
                & 9.80665
                & m\,s$^{-2}$
                & -- \\
        $k$     & Ice thermal conductivity
                & 2.10
                & J\,m$^{-1}$\,K$^{-1}$\,s$^{-1}$
                & \citet[p.~400]{Cuffey.Paterson.2010} \\
        %$L$     & Water latent heat of fusion
        %        & $3.35\times10^5$
        %        & J\,kg$^{-1}$\,K$^{-1}$
        %        & \citet[p.~400]{Cuffey.Paterson.2010} \\
        $Q$     & Flow law activation energy
                & $115\times10^3$
                & J\,mol$^{-1}$
                & \citet[p.~74]{Cuffey.Paterson.2010} \\
        $R$     & Ideal gas constant
                & 8.314
                & J\,mol$^{-1}$\,K$^{-1}$
                & \citet[p.~72]{Cuffey.Paterson.2010} \\
        $\rho$  & Ice density
                & 917
                & kg\,m$^{-3}$
                & \citet[p.~12]{Cuffey.Paterson.2010} \\
        \hline
      \end{tabular}}
    \end{table}

    \begin{table}
      \centerline{\begin{minipage}{85mm}
        \caption{%
          Selected daily mean temperature values corresponding to profiles shown
          on Fig.~\ref{fig:profiles}a (dates as YYMMDD).}
        \label{tab:bh1}
        \begin{tabular}{ccccc}
          \hline
          Unit &  Depth & \multicolumn{3}{c}{Temperature ($^\circ$C)}\\
          (BH3)&   (m)  & 150101 & 151112 & 160719 \\
          \hline
          LT14 &  10.71 & -4.57 & -5.66 & -3.94 \\
          LT13 &  30.71 & -1.26 & -1.49 & -1.71 \\
          LT12 &  50.71 & -1.33 & -1.19 & -1.10 \\
          LT11 &  70.71 & -2.35 & -2.08 & -1.87 \\
          LT10 &  90.71 & -3.36 & -3.03 & -2.77 \\
          LI05 & 143.46 & -4.10 & -3.86 &   --  \\
          LI04 & 192.57 & -2.49 & -2.50 &   --  \\
          LI03 & 232.26 & -0.58 & -0.68 &   --  \\
          LP   & 246.89 & -0.19 & -0.22 & -0.26 \\
          \hline
          (ERR)&   --   & 150101 & 160719 & --  \\
          \hline
          LT09 & 110.71 & -4.01 & -3.42 &   --  \\
          LT08 & 130.71 & -3.53 & -3.29 &   --  \\
          LT07 & 150.71 & -2.87 & -2.79 &   --  \\
          LT06 & 170.71 & -1.67 & -1.69 &   --  \\
          LT05 & 190.71 & -0.96 & -1.05 &   --  \\
          LT04 & 210.71 & -0.17 & -0.41 &   --  \\
          LT03 & 230.71 & -0.18 & -0.31 &   --  \\
          LT02 & 250.71 & -0.18 & -0.26 &   --  \\
          LT01 & 252.00 & -0.21 & -0.25 &   --  \\
          \hline
        \end{tabular}
      \end{minipage}
      \begin{minipage}{85mm}
        \begin{tabular}{cccc}
          \hline
          Unit &  Depth & \multicolumn{2}{c}{Temperature ($^\circ$C)}\\
          (BH1)&   (m)  & 141001 & 170128 \\
          \hline
          UI07 & 122.83 & -6.03 & -5.90 \\
          UI06 & 171.66 & -5.35 & -5.26 \\
          UI05 & 208.42 & -3.71 & -3.58 \\
          UI04 & 231.52 & -2.06 & -2.01 \\
          UI03 & 252.37 & -1.04 &   --  \\
          UI02 & 265.08 & -0.67 &   --  \\
          \hline
          (BH2)&   --   & 141001 & 160712 \\
          \hline
          UT16 &   8.56 & -7.22 & -7.39 \\
          UT15 &  28.56 & -2.47 & -3.35 \\
          UT13 &  48.56 & -3.74 & -3.37 \\
          UT12 &  68.56 & -5.08 & -4.77 \\
          UT11 &  88.56 & -5.76 & -5.54 \\
          UT10 & 108.56 & -5.98 & -5.87 \\
          UT09 & 118.56 & -5.92 & -5.74 \\
          UT08 & 138.56 & -5.94 & -5.84 \\
          UT07 & 158.56 & -5.82 & -5.78 \\
          UT06 & 178.56 & -5.26 &   --  \\
          UT05 & 198.56 & -4.66 & -4.52 \\
          UT04 & 218.56 & -3.43 & -3.32 \\
          UT03 & 238.56 & -2.03 & -2.03 \\
          UT02 & 258.56 & -0.72 & -0.82 \\
          UT01 & 262.00 & -0.23 & -0.31 \\
          \hline
        \end{tabular}
      \end{minipage}}
    \end{table}


% ======================================================================
\end{document}
% ======================================================================

\documentclass{article}

\usepackage{authblk}
\usepackage[T1]{fontenc}
\usepackage[utf8]{inputenc}
\usepackage[pdftex]{xcolor}
\usepackage[pdftex]{graphicx}

\graphicspath{{../../figures/}}

\definecolor{darkblue}{cmyk}{0.9,0.3,0.0,0.0}
\definecolor{darkgreen}{cmyk}{0.8,0.0,1.0,0.0}
\definecolor{darkred}{cmyk}{0.1,0.9,0.8,0.0}

\newcommand{\idea}[1]{\textcolor{darkgreen}{\emph{[\textbf{IDEA:} #1]}}}
\newcommand{\note}[1]{\textcolor{darkblue}{\emph{[\textbf{NOTE:} #1]}}}
\newcommand{\todo}[1]{\textcolor{darkred}{\emph{[\textbf{TODO:} #1]}}}
\newcommand{\aref}[0]{\textcolor{darkblue}{\textbf{[REF.]}}}

\title{Englacial warming indicates deep crevassing in Bowdoin tidewater
       glacier, northwest Greenland}

\author[1]{Julien Seguinot
           \thanks{Correspondence to seguinot@vaw.baug.ethz.ch}}
\author[1]{Martin Funk}
\author[2]{Shin Sugiyama}

\affil[1]{Laboratory of Hydraulics, Hydrology and Glaciology,
          ETH Zürich, Switzerland}
\affil[2]{Institute of Low Temperature Science,
          Hokkaido University, Sapporo, Japan}

% ======================================================================
\begin{document}
% ======================================================================

\maketitle

\begin{abstract}

    \note{Preliminary abstract. The two first paragraphs will be shortened and
          reworked to better reflect the introduction.}

    The observed rapid retreat of ocean-terminating glaciers in southern
    Greenland in the last two decades has now propagated to the northwest.
    Hence, tidewater glaciers in this area, some of which have remain stable
    for decades, have started retreating rapidly through iceberg calving in
    recent years, thus allowing a monitoring and investigation of ice dynamical
    changes starting the early stages of retreat.

    Bowdoin Glacier, a small and relatively accessible calving glacier in
    Northwest Greenland, appeared to be very stable since its frontal position
    had been first documented by explorers in the late 19th century. However,
    following a two-fold increase of surface velocity in the early 2000s, the
    calving front of Bowdoin Glacier has experienced a rapid retreat of ca.
    2\,km between 2007 and 2013. Since 2013, the ice front is once again
    stable, yet the glacier surface continues to experience lowering at an
    alarming rate of ca. 6\,m/a.

    Here, we present a two-year record of englacial temperatures measured in
    the tidewater glacier tongue. The measured temperature profiles indicate
    that the glacier's ice is below pressure melting point except for its
    temperate base. More surprisingly, a temperature difference of up to 2
    degrees was measured between two boreholes that are located only 250 meter
    apart along a flowline. In addition, the warmer profile shows a slow
    warming trend of up to ca.~0.5\,$^\circ$C\,a$^{-1}$\todo{Refine.}. This
    patterns can be explained by latent heat release from meltwater freezing in
    transverse crevasses that reach to or near the glacier bed. These crevasses
    may be related to the tide-modulated extensional strain regime of the
    tidewater glacier tongue.

\end{abstract}

% ----------------------------------------------------------------------
\section{Introduction}
% ----------------------------------------------------------------------

    All around Greenland, ocean-terminating glaciers are accelerating and
    retreating faster than other parts of the ice sheet.

    As buoyance forces weaken and glaciers come close to floatation, basal
    friction diminishes and the only limiting factor left against flow is the 
    ice's internal viscosity.

    However, ice rheology depends very much on temperature. The ice softness
    varies by an order of magnitude between -15 degrees and the
    pressure-melting point. Numerical models show that such differences in
    rheology affect the shape and flow velocity of glaciers and ice sheets even
    in regions dominated by sliding and longitudinal stretching (check).

    But temperature was never measured on the terminal tongue of tidewater
    glaciers. This is because tidewater glacier tongues are fast-flowing,
    and often full of crevasses causing multiple practical and technical
    challenges to their instrumentation.

    Here, we present ice temperature measurements from Bowdoin Glacier, a
    tidewater glacier which is relatively little crevassed and located in the
    vicinity of the northernmost civilean airport of Greenland in Qaanaaq. The
    glacier has a beautiful medial moraine which allowed to walk down to its
    calving front and install instruments in the fastest-moving areas.

    The installations could be maintained for up to three years, during which
    some of the instruments were successively lost to ice deformation and harsh
    climatic conditions. This three-year record of englacial temperature in
    previously unexplored type of environment reveals new mechanisms by which
    tidewater glacier temperatures fluctuate.


% ----------------------------------------------------------------------
\section{Methods}
% ----------------------------------------------------------------------

\subsection{Drilling sites}

    Fig.~\ref{fig:boreholes}a -- Location map.

    Table.~\ref{tab:drilling} -- Borehole drilling sites.

    Bowdoin Glacier is a minor outlet glacier of the Greeland Ice Sheet located
    in its northwestern parth. The site was choosen for its accessibility, and
    because at the time of field campaign planning it appeared that the
    Greenland Ice Sheet mass loss was propagating northwest, although recent
    satellite gravimetry data show that now nearly all margins of the Greenland
    Ice Sheet loose mass.

    The glacier was first explored by western explorers in the late-nineteeth
    century. Photographs show that the glacier was thicker then, but its
    frontal position was only a few kilometers downstream from the present
    calving front. Since then it has been relatively stable. However, following
    a two-fold increase of surface velocity in the early 2000s, the calving
    front of Bowdoin Glacier has experienced a rapid retreat of ca.  2\,km
    between 2007 and 2013. Since 2013, the ice front is once again stable, yet
    the glacier surface continues to experience lowering at an alarming rate of
    ca. 6\,m/a.  Bowdoin Glacier velocities vary in response to tides and
    calves according to a relatively recurring pattern. 

    In summer 2014, three boreholes BH1, BH2, and BH3 were drilled about 2\,km
    upstream from the calving front. After the sucessful drilling of BH1 and
    BH2 it was planned that another two boreholes would be drilled some X\,km
    ustream in the thicker part of the glacier. But due to weather the
    hotwater drilling equipment could not be shipped upstream and the third
    and last hole was drilled some x\,m downstream of the first drilling site
    and equipped with all instruments left.


\subsection{Instrumentation}

    Fig.~\ref{fig:boreholes}b -- Borehole sensors map.

    The boreholes were equipped with thermistor strings throughout the ice
    column, piezometers at the glacier bed, and inclinometers at different
    depths. The piezometers and inclinometers were also equipped with
    thermistors so that temperature could be measured at multiple depths
    through the ice column. This manuscript focuses on the temperature data.
    The pressure and tilt data will be presented elsewhere.

    At the first (upper) drilling site one borehole (BH1) was equipped with
    the digital inclinometers and the other (BH2) with thermistor strings
    and basal piezometer. At the second (lower) drilling site the unique
    borehole (BH3) was equipped with digital inclinometers, a basal
    piezometer and thermistor strings.

    Data loggers were anchored on the ice surfaced and powered with batteries
    and solar panels to collect data throuhout the dark and cold Arctic winter.
    In the case of digital inclinometers, the observed resistance from the
    thermistors were converted to temperature values in-situ by the englacial
    digitizers, thus avoiding to record the resistance of the borehole cables
    and their potential variations due to cable deformation.


\subsection{Temperature calibration}

    Table X -- Temperature calibration?

    All thermistors were calibrated prior to fieldwork in the lab (Martin?).
    For digital inclinometres to conversion from resistance measurements to
    temperature values is performed in-situ by the englacial digitizers. For
    basal piezometers this conversion is performed by the data logger installed
    on the ice surface. For the thermistor strings the conversion is performed
    during post-processing following the formula:

    \begin{equation}
      T = 1 / (a_1 + a_2 \log(R) + a_3 \log(R)^3) - 273.15
    \end{equation}

    where $R$ is the measured resistance, $T$ the ice temperature and $a_1$,
    $a_2$, and $a_3$ coefficients calibrated individually for each sensor.

    Despite this initial calibration it was observed that temperatures
    observed initially in the borehole right after drilling were not at the
    pressure melting point. Thus, where possible a recalibration was applied in
    post-processing by applying a temperature offset when the initial
    temperature data was available. Unfortunately some of the initial
    temperature data has been lost so that this recalibration was possible for
    some of the sensors only.


% ----------------------------------------------------------------------
\section{Results}
% ----------------------------------------------------------------------

\subsection{Temperature time series}
    Fig.~\ref{fig:timeseries} -- Temperature time series.

\subsection{Temperature profiles}
    Fig.~\ref{fig:profiles}a -- Temperature profiles.

\subsection{Englacial warming}
    Fig.~\ref{fig:profiles}b -- Warming profiles.


% ----------------------------------------------------------------------
\section{Discussion}
% ----------------------------------------------------------------------

\subsection{Deep crevassing}
    Fig.~\ref{fig:arcticdem} -- Topography around boreholes.

\subsection{Tidal effects}
    \idea{Bowdoin surface topography is wavy showing that parts of the glacier
          are thinner. Could thin parts experience more longitudinal
          stretching than thick ones and be more prone to calving?}

% ----------------------------------------------------------------------
\section{Conclusions}
% ----------------------------------------------------------------------


% ----------------------------------------------------------------------
% Figures
\clearpage
% ----------------------------------------------------------------------

    \begin{figure}
      \centerline{\includegraphics{bowtem_boreholes}}
      \caption{\textbf{(a)} Bowdoin borehole locations from drilling in July
               2014 to dismantling in July 2017 and background satellite
               image from 8 Aug. 2016, 17:59:15 UTC. Contains modified
               Copernicus Sentinel data, processed with Sentinelflow.
               \textbf{(b)} Initial ice thickness and sensor depths.
               Inclinometers and piezometers are also equipped with
               thermistors.}
      \label{fig:boreholes}
    \end{figure}

    \begin{figure}
      \centerline{\includegraphics{bowtem_timeseries}}
      \caption{Ice temperature time series from drilling in July 2014 to
               dismantling in July 2017. Field campaigns (orange spans) and
               dates selected for temperature profiles (dashed lines;
               Fig.~\ref{fig:profiles}) are indicated.}
      \label{fig:timeseries}
    \end{figure}

    \begin{figure}
      \centerline{\includegraphics{bowtem_profiles}}
      \caption{\textbf{(a)} Daily mean temperature profiles for selected dates
               (Fig.~\ref{fig:timeseries}) three months after the drilling
               (solid lines) and long after the refreezing of the entire
               borehole (dashed lines).
               \textbf{(b)} Theoretical temperature diffusion (solid lines) and
               observed temperature changes (dashed lines) for the
               corresponding period.}
      \label{fig:profiles}
    \end{figure}

    \begin{figure}
      \centerline{\includegraphics{bowtem_arcticdem}}
      \caption{\textbf{(a)} Glacier surface topography in the vicinity of the
               boreholes sites in the early phase of the experiment (Arctic
               DEM, 6 Sep. 2014) and corresponding locations of the boreholes.
               The locations of BH2 and BH3 are estimated from their initial
               locations in July 2014 (color crosses) and the observed
               displacement of BH1 during the corresponding time intervals.
               \textbf{(b)} Glacier surface topographic profile along flow
               passing through BH1 and the estimated location BH3.}
      \label{fig:arcticdem}
    \end{figure}


% ----------------------------------------------------------------------
% Tables
\clearpage
% ----------------------------------------------------------------------

    \begin{table}[t]
      \caption{%
        Bowdoin Glacier drilling sites.}
      \label{tab:drilling}
      {\begin{tabular}{cccccc}
        \hline
        Hole & Depth & Date       & Latitude  & Longitude  & Elev. \\
        \hline
        BH1  & 272   & 2014-07-16 & 77.691244 & -68.555749 & 88.7 \\ % UI
        BH2  & 262   & 2014-07-17 & 77.691307 & -68.555685 & 87.7 \\ % UPT
        BH3  & 252   & 2014-07-23 & 77.689995 & -68.558857 & 83.4 \\ % LIPT
        \hline
      \end{tabular}}
    \end{table}

% ======================================================================
\end{document}
% ======================================================================

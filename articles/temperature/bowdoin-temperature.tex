\documentclass[utf8]{article}

\usepackage{doi}
\usepackage{authblk}
\usepackage[T1]{fontenc}
\usepackage[utf8]{inputenc}
\usepackage[pdftex]{xcolor}
\usepackage[pdftex]{graphicx}
\usepackage[authoryear,round]{natbib}

% review mode
\usepackage{geometry}
\usepackage{lineno}
\linenumbers
\linespread{1.5}

\graphicspath{{../../figures/}}

\definecolor{c0}{HTML}{1f77b4}
\definecolor{c1}{HTML}{ff7f0e}
\definecolor{c2}{HTML}{2ca02c}
\definecolor{c3}{HTML}{d62728}
\definecolor{c4}{HTML}{9467bd}
\definecolor{c5}{HTML}{8c564b}
\definecolor{c6}{HTML}{e377c2}
\definecolor{c7}{HTML}{7f7f7f}
\definecolor{c8}{HTML}{bcbd22}
\definecolor{c9}{HTML}{17becf}

\newcommand{\idea}[1]{\textcolor{c2}{\emph{[\textbf{IDEA:} #1]}}}
\newcommand{\note}[1]{\textcolor{c0}{\emph{[\textbf{NOTE:} #1]}}}
\newcommand{\todo}[1]{\textcolor{c3}{\emph{[\textbf{TODO:} #1]}}}

\hypersetup{colorlinks, citecolor=c0, linkcolor=c1, urlcolor=c6}

\title{Englacial warming indicates deep crevassing in Bowdoin tidewater
       glacier, northwest Greenland}

\author[1]{Julien Seguinot
           \thanks{Correspondence to seguinot@vaw.baug.ethz.ch}}
\author[1]{Martin Funk}
\author[1]{Andreas Bauder}
\author[1]{Thomas Wyder}
\author[2]{Cornelius Senn}
\author[3]{Shin Sugiyama}

\affil[1]{Laboratory of Hydraulics, Hydrology and Glaciology,
          ETH Zürich, Switzerland}
\affil[2]{Department of Civil, Environmental and Geomatic Engineering,
          ETH Zürich, Switzerland}
\affil[3]{Institute of Low Temperature Science, Hokkaido University,
          Sapporo, Japan}


% ======================================================================
\begin{document}
% ======================================================================

\maketitle

\begin{abstract}

    All around the margins of the Greenland Ice Sheet, ocean-terminating
    glaciers currently thin and accelerate. With basal friction forces nearly
    suppressed, only the internal viscous forces limit ice flow \todo{rework}.
    The viscosity
    of ice depends critically on its temperature, but temperature has rarely
    been measured on such fast flowing and heavily crevassed glaciers.
    %
    Here, we present a three-year record of englacial temperatures from the
    terminal tongue of Bowdoin Glacier, a tidewater glacier in northwest
    Greenland. Two boreholes separated by 165--197\,m show full-depth
    temperature differences up to ca.~2\,$^\circ$C. Englacial warming up to
    0.41$^\circ$C in 393 days indicates a deep and local heat source within the
    tidewater glacier.
    %
    We interpret the heat source as latent warming from meltwater refreezing in
    crevasses reaching to or near the bed of the glacier, whose localisation
    may be controlled by preferential mechanical damage and meltwater ponding
    in topographic dips between ogives.

\end{abstract}

% ----------------------------------------------------------------------
\section{Introduction}
% ----------------------------------------------------------------------

    In many parts of the world, glaciers are currently slowing down as they
    become thinner \citep{Heid.Kaab.2012, Dehecq.etal.2018}. Yet this behaviour
    is not ubiquitous. Around the margins of
    the Greenland Ice Sheet, ocean-terminating outlet glaciers also thin, but
    they accelerate and retreat faster than any other part of the ice sheet
    \citep[e.g.,][]{Krabill.etal.2000, Rignot.Kanagaratnam.2006,
    Pritchard.etal.2009, Hill.etal.2017}, significantly impacting its total
    mass loss \citep[e.g.,][]{Enderlin.etal.2014, Khan.etal.2015,
    McMillan.etal.2016}.

    These so-called tidewater glaciers are partly submerged in sea water, such
    that when they thin due to surface melt, more and more of the glaciers'
    weight is balanced by buoyancy forces from the ocean. Basal friction is
    reduced and the glaciers flow faster, thus thinning even more
    \citep{Meier.Post.1987}. When the glaciers come close to floatation, basal
    friction may be so low \citep{Shapero.etal.2016, Seddik.etal.2019} that the
    only factor limiting flow velocities are the internal viscous forces in
    the ice \todo{rework: lateral friction, buttressing by ice melange}.

    However, ice viscosity depends critically on temperature. Between
    $-15\,^\circ$C and the pressure-melting point, ice softness varies by as
    much as an order of magnitude \citep[p.~72]{Cuffey.Paterson.2010}.
    Numerical models show that such differences
    in ice viscosity propagate to influence the flow and shape of entire
    glaciers and ice sheets \citep[e.g.,][Figs.~2 and~7]{Seguinot.etal.2016}.

    Previous measurements show that glacier ice temperatures can be below
    the pressure-melting point (cold ice), at the pressure-melting point
    (temperate ice), or both (polythermal glacier)
    \citep[p.~399]{Ahlmann.1935, Cuffey.Paterson.2010}. Glacier temperatures
    are primarily controlled by air temperatures, geothermal heat flux, strain
    heating, and internal heat advection and diffusion \citep{Q-Robin.1955},
    but can also be affected by latent heating from meltwater refreezing in
    crevasses \citep{Phillips.etal.2010, Phillips.etal.2013, Luthi.etal.2015}.

    Much of this knowledge comes from measurements conducted on mountain
    glaciers and the interior of ice sheets. Ice temperature measurements from
    tidewater glaciers are presently available near the calving front for
    Svalbard \citep{Jania.etal.1996} but limited to upstream areas in Greenland
    \citep{Iken.etal.1993, Luthi.etal.2002, Luthi.etal.2015, Doyle.etal.2018}.
    Yet Greenlandic tidewater glaciers flow hundreds of metres to kilometres
    per year and experience strong longitudinal stresses where marine-based
    \citep[e.g.,][]{Fastook.etal.1995}. Heavy
    crevassing hinders accessibility and poses practical and technical
    challenges to instrumentation.

    Here, we present a new, three-year, continuous record of englacial
    temperature from Bowdoin Glacier, a relatively accessible tidewater outlet
    glacier of the northwestern Greenland Ice Sheet. Despite successive data
    losses due to harsh climate and ice deformation, the record reveals
    new mechanisms controlling temperature variations in tidewater glacier.


% ----------------------------------------------------------------------
\section{Methods}
% ----------------------------------------------------------------------

% -- -- -- -- -- -- -- -- -- -- -- -- -- -- -- -- -- -- -- -- -- -- -- -
\subsection{Bowdoin Glacier}
% -- -- -- -- -- -- -- -- -- -- -- -- -- -- -- -- -- -- -- -- -- -- -- -

    Bowdoin Glacier is a relatively accessible tidewater glacier located in
    northwest Greenland. It is a medium-size outlet glacier of the Greenland
    Ice Sheet with a catchment of about 60$\times$60\,km, and drains into
    Bowdoin Fjord across a 3\,km-wide calving front
    (Fig.~\ref{fig:boreholes}a). The glacier front is located 30\,km from the
    village of Qaanaaq and northernmost airport in Greenland
    \citep[Fig.~1]{Sugiyama.etal.2015}.

    The glacier was chosen for fieldwork due to its accessibility, and
    because at the time of field campaign planning it appeared that the mass
    loss of the Greenland Ice Sheet was propagating northwest
    \citep{Khan.etal.2010}. However, the more recent satellite gravimetry data
    show that virtually all margins of the Greenland Ice Sheet are now loosing
    mass \citep{Groh.Horwath.2016}. The surface of Bowdoin Glacier
    is heavily crevassed but less so than that of many other Greenlandic
    tidewater glaciers. Besides, a ca.~20\,m-wide medial moraine facilitate
    displacements on the glacier surface \citep[Figs.~68]{Chamberlin.1897}.
    Except preceding calving, the moraine is generally uninterrupted by
    crevasses and can be walked down to the glacier front
    (Fig.~\ref{fig:images}a).

    Bowdoin Glacier was first explored (and so named) by western explorers in
    the late-nineteenth century and described in the following terms:
    %
    \begin{quote}
        Beyond that, an isolated mountain of striking boldness and sharpness of
        outline jutted into the air apparently some two thousand feet, and
        then, from its base, the crystal wall of a great glacier stretched
        clear across the opposite side of the bay head. This glacier I named,
        in honour of my Alma Mater, Bowdoin Glacier, and the bay I called
        Bowdoin Bay \citep[p.~393--394]{Peary.1898}.
    \end{quote}
    %
    Peary reported a ``daily movement'' of 0.85\,m for July 1893 ``at the
    fastest point'' of the glacier \citep{Chamberlin.1894}.
    Photographs indicate that Bowdoin Glacier was thicker at that time, but its
    frontal position was only a few kilometers downstream from the present
    calving front (\citealp[p.~668]{Chamberlin.1895}, \citealp[Figs.~64
    and~65]{Chamberlin.1897}, \citealp[Fig.~1]{Podolskiy.etal.2016}) and has
    been relatively stable since then.

    However, a two-fold increase of surface velocity in the early 2000s was
    followed by a rapid frontal retreat of Bowdoin Glacier by ca.~2\,km between
    2007 and 2013 \citep[Fig.~2]{Sugiyama.etal.2015}, a behaviour synchronous
    to that of other tidewater glaciers in the area
    \citep{Sakakibara.Sugiyama.2018}. Since 2013, the ice front appears once
    again stable, yet the terminal tongue continues to experience surface
    lowering at an alarming rate of ca.~4.1\,m\,a$^{-1}$, which is primarily
    the expression of continued dynamic thinning \citep{Tsutaki.etal.2016}.

    Bowdoin Glacier longitudinal strain and seismicity vary in response to
    tides~\citep{Podolskiy.etal.2016, Podolskiy.etal.2017} indicating a low
    basal drag and near plug-flow conditions \citep{Seddik.etal.2019}. Major
    icebergs calve according to a recurring pattern in which fractures
    propagate nearly parallel to ice front \citep{Jouvet.etal.2017}. Subglacial
    meltwater exiting the glacier bed forms several submarine plumes whose flow
    is highly variable \citep{Jouvet.etal.2018} and entrain nutrients from the
    bottom to the surface of Bowdoin Fjord \citep{Kanna.etal.2018}.


% -- -- -- -- -- -- -- -- -- -- -- -- -- -- -- -- -- -- -- -- -- -- -- -
\subsection{Drilling sites}
% -- -- -- -- -- -- -- -- -- -- -- -- -- -- -- -- -- -- -- -- -- -- -- -

    During the field campaign in July 2014, three boreholes BH1, BH2, and BH3
    were drilled with hot water in the tidewater tongue of Bowdoin Glacier
    using meltwater from the crevasses. After the successful drilling of BH1
    and BH2 7\,m apart at the first (upper) drilling site ca.~2\,km from the
    calving front, it was planned that two other boreholes would be drilled
    upstream in the thicker part of the glacier. Due to unfavourable weather the
    hot water drilling equipment could not be shipped upstream. Thus, a third
    and last hole was drilled 158\,m downstream of the first borehole site
    to be equipped with all instruments left (Fig.~\ref{fig:boreholes}a,
    Table~\ref{tab:drilling}).

    Although the experiment was originally planned for one year, some of the
    instruments were let on the glacier for up to three years as more funding
    was obtained and additional field campaigns were planned for parallel
    experiments. From the drilling in July 2014 to the last field campaign in
    July 2017, the boreholes were displaced by 997 (BH1) to 1191\,m (BH3),
    their surface lowered from ca.~89 (BH1, 2014) to 54\,m~above sea level
    (BH3, 2017) and
    the distance between the lower (BH3) and uppermost (BH2) boreholes
    increased from 165 to 197\,m (Fig.~\ref{fig:boreholes}a). New crevasses
    appeared on the glacier surface, some causing damage to the instruments
    (Fig.~\ref{fig:images}b).


% -- -- -- -- -- -- -- -- -- -- -- -- -- -- -- -- -- -- -- -- -- -- -- -
\subsection{Instrumentation}
% -- -- -- -- -- -- -- -- -- -- -- -- -- -- -- -- -- -- -- -- -- -- -- -

    The boreholes were equipped with three types of sensors: strings of simple,
    regularly-spaced thermistors arranged to span the entire depth of the
    glacier, two piezometers near the base of the glacier, and digital
    inclinometer units at different depths. Besides their primary sensors, the
    piezometers and digital inclinometers were equipped with additional
    thermistors, so that ice temperature could be measured at multiple depths
    in the glacier. This manuscript focuses on the temperature data, and other
    data will be presented elsewhere.

    At the first (upper) drilling site, one borehole (BH1) was equipped with
    seven digital inclinometers (Fig.~\ref{fig:boreholes}b, blue triangles) and
    the other (BH2) with one basal piezometer (Fig.~\ref{fig:boreholes}b,
    orange square) and two thermistor strings (Fig.~\ref{fig:boreholes}b,
    orange circles). At the second (lower) drilling site, the unique
    borehole (BH3) was equipped with five digital inclinometers
    (Fig.~\ref{fig:boreholes}b, green triangles), one basal piezometer
    (Fig.~\ref{fig:boreholes}b, green square) and two thermistor strings
    (Fig.~\ref{fig:boreholes}b, green and grey circles).

    Due to the relocation of the second drilling site the thermistor strings
    were re-arranged to fit a smaller ice thickness. However, the deeper
    thermistor string depict temperatures incompatible with those recorded by
    digital inclinometers in the same hole. Because the depths of digital
    inclinometers were calibrated from independent pressure sensors, we
    interpret that sensors on the thermistor string were misplaced due to a
    manipulation error and mark their positions and data as erratic (ERR,
    Fig.~\ref{fig:boreholes}b, grey circles).

    The basal piezometers (GeoKon 4500) were connected to Campbell CR10X data
    loggers via Campbell AVW1 vibrating wire interfaces. The thermistor strings
    \citep[NTC~Fenwal 135-103FAG-J01,][]{Ryser.2014} were connected to Campbell
    CR1000 data loggers via Campbell AM416 relay multiplexers. The piezometers
    and thermistor strings data loggers were each powered by a 12\,V, 24\,Ah
    lead battery, and mounted in polyester cases on tetrapods
    (Fig.~\ref{fig:images}a). These batteries were temporarily replaced and
    carried out of the glacier during each field campaign to be recharged using
    a petrol generator at the camp. For the thermistor strings, in addition to
    automatic measurements, manual readings were performed in 2015, 2016 and
    2017 using a hand-held ohmmeter.

    The digital inclinometers \citep[DIBOSS,][]{Ryser.2014, Ryser.etal.2014,
    Ryser.etal.2014a} are equipped with iST~TSic~716 temperature sensors and
    were connected to CR1000 data loggers. Each data logger
    was set-up in a hard plastic case including three 12\,V, 65\,Ah lead
    batteries and a solar panel on the outside, and anchored to an aluminum
    poles drilled into the ice.
    In the case of digital inclinometers, the observed resistance from the
    thermistors were converted to temperature values in situ by the englacial
    digitizers, thus avoiding to record the resistance of the borehole cables
    and their potential variations due to cable deformation. The deployment and
    maintenance of the borehole installations was eased by the medial moraine
    (Fig.~\ref{fig:images}b).

    Finally, a Differential Global Positioning System (DGPS) receiver was
    installed near the first borehole (BH1). The DGPS receiver (\todo{model
    name?}) was mounted on an aluminium stake re-drilled every summer in the
    ice to accumodate melt. It was connected to a \todo{model name?} data
    logger set-up in a hard plastic case and powered by an external
    \todo{XX\,V, XX\,Ah} lead battery and a solar panel. The DGPS reference
    station was installed near the camp site (Fig.~\ref{fig:boreholes}).
    \note{Shin: could you help me to fill in these technical data?}


% -- -- -- -- -- -- -- -- -- -- -- -- -- -- -- -- -- -- -- -- -- -- -- -
\subsection{Temperature calibration}
% -- -- -- -- -- -- -- -- -- -- -- -- -- -- -- -- -- -- -- -- -- -- -- -


    For the thermistor strings, the conversion from resistance to temperature
    is performed during post-processing following the Steinhart-Hart equation:
    %
    \begin{equation}
      \frac{1}{T} = a_0 + a_1 \log(R) + a_3 \log(R)^3
    \end{equation}
    %
    where $R$ is the measured resistance, $T$ the ice temperature, and $a_0$,
    $a_1$, and $a_3$ are coefficients calibrated individually for each sensor
    prior to fieldwork in the lab for temperatures of -15, -12, -9, -6, -3 and
    0\,$^\circ$C (Table~\ref{tab:calibration}). The basal piezometers use
    temperature calibration coefficients published by the constructor
    (Table~\ref{tab:calibration}). The digital inclinometers use fully
    calibrated sensors delivering direct temperature measurements.

    Despite the pre-field calibration it was observed that initial temperatures
    observed in the borehole immediately after drilling were not at the
    pressure-melting point. Thus for sensors where initial data is available,
    a recalibration is applied in post-processing by correcting for the initial
    temperature offset, $\Delta T$ (Table~\ref{tab:calibration}). Unfortunately
    some of these initial
    temperature data have been lost so that the recalibration was possible for
    some sensors only. The pressure-melting point was computed as
    %
    \begin{equation}
      T_m = -\beta \rho g z
    \end{equation}
    %
    where $\beta$ is the empirical Clapeyron constant for glacier ice
    \citep{Luthi.etal.2002}, $\rho$ is the ice density, $g$ is the standard
    acceleration due to gravity, and $z$ is the depth below the ice surface
    (parameter values given in Table~\ref{tab:parameters}).


% ----------------------------------------------------------------------
\section{Results}
% ----------------------------------------------------------------------

% -- -- -- -- -- -- -- -- -- -- -- -- -- -- -- -- -- -- -- -- -- -- -- -
\subsection{Length of the record}
% -- -- -- -- -- -- -- -- -- -- -- -- -- -- -- -- -- -- -- -- -- -- -- -

    From the two drilling sites on the tidewater tongue of Bowdoin Glacier,
    continuous ice temperature time series could be obtained with a length of
    up to three years. Although the experiment was initially planned for one
    year, following initial success some of the instruments were let installed
    for up to three years. However, sensors were successively lost due to
    technical failures and ice deformation (Table~\ref{tab:timeline}).

    Of the total 44~sensors installed, 41~worked after the installation, 38
    recorded data for one year, 35 were still functional after two years (of
    which 16 were not connected to a data logger any longer but were used for a
    manual reading), and only two were left
    working after three years. Contact to sensors was progressively lost to a
    battery issue (2014 July~28 and again 2015 July~26, one sensor) cable
    damage during the installation (2014 July~16 and 23, three sensors) cable
    damage at depth (2014 Oct.~25, 2017 Jan.~28, Feb.~03, and June~21, five
    sensors), and most importantly, cable damage at the surface (2015 Nov.~12
    and unknown dates, 33~sensors, Table~\ref{tab:timeline},
    Fig.~\ref{fig:images}b).


% -- -- -- -- -- -- -- -- -- -- -- -- -- -- -- -- -- -- -- -- -- -- -- -
\subsection{Temperature time series}
% -- -- -- -- -- -- -- -- -- -- -- -- -- -- -- -- -- -- -- -- -- -- -- -

    After the drillings, most sensors are immersed and temperatures are
    measured at or near the pressure-melting point. In BH2 some sensors record
    strong temperature variations as they are temporarily emerged. Temperature
    then drop off the pressure-melting point and follow an S-curve before
    stabilizing below freezing. This initial phase lasts for hours to months
    depending on the sensor and relating to the equilibrium temperature. While
    most sensors reach a thermal equilibrium within three months, some refreeze
    only after more than a year, and stabilize at temperatures below but near
    the pressure-melting point (Fig.~\ref{fig:timeseries}).

    The sensors nearest to the ice surface exhibit a seasonal temperature
    cycle, which is out-of-phase with the atmospheric temperature cycle. The
    amplitude increases with time as the thermistor strings progressively
    melt-out of the glacier. Temperature records for sensors installed deeper
    down in the ice do not show a seasonal cycle, but many exhibit a slow
    warming trend (Fig.~\ref{fig:timeseries}).

    Manual temperature readings are generally compatible with the automatic
    records (Fig.~\ref{fig:timeseries}, full circles), but some of these
    measurement are off by a few degrees, perhaps due to wet connectors
    (Fig.~\ref{fig:timeseries}, empty circles). Manual readings were also
    performed in 2017 but yielded values well-off the expected range, which
    is certainly due surface cable damage visibly caused by opening crevasses
    (Fig.~\ref{fig:images}b).


% -- -- -- -- -- -- -- -- -- -- -- -- -- -- -- -- -- -- -- -- -- -- -- -
\subsection{Temperature profiles}
% -- -- -- -- -- -- -- -- -- -- -- -- -- -- -- -- -- -- -- -- -- -- -- -

    After the refreezing of the entire boreholes, vertical profiles depict
    temperatures below the pressure-melting point except for the base of the
    glacier where temperatures are at the pressure-melting point. The coldest
    temperatures are reached in the middle part of the ice column, above which
    is found a layer of relatively high temperatures, and finally a layer with
    seasonal temperature variations (Fig.~\ref{fig:profiles}a).

    The upper boreholes, BH1 and BH2, which are only separated by seven metres,
    show compatible temperature profiles (Fig.~\ref{fig:profiles}a, blue and
    orange lines). For the lower borehole, BH3, the deepest thermistor string
    (Fig.~\ref{fig:profiles}a, grey lines) depict temperatures incompatible
    with those recorded by digital inclinometers in the same hole
    (Fig.~\ref{fig:profiles}a, green lines), whose depths were calibrated
    using independent pressure sensors.

    Finally, the upper (BH1, BH2) and lower (BH3, ERR) boreholes show
    significant temperature differences despite being located on the same
    flow line and separated by only 165 (2014) to 197\,m (2017). Remarkably,
    temperature differences
    around 2$^\circ$C prevail over the entire glacier depth.


% -- -- -- -- -- -- -- -- -- -- -- -- -- -- -- -- -- -- -- -- -- -- -- -
\subsection{Englacial warming}
% -- -- -- -- -- -- -- -- -- -- -- -- -- -- -- -- -- -- -- -- -- -- -- -

    All three temperature profiles show a general warming trend, except for
    their base where some sensors take several months to refreeze after the
    boreholes were drilled (Figs.~\ref{fig:profiles}a and~\ref{fig:profiles}b,
    dashed lines). However, the warming trend is much more pronounced for the
    lower (BH3, ERR; Fig.~\ref{fig:profiles}, green and grey lines) than for
    the upper (BH1, BH2; Fig.~\ref{fig:profiles}, blue and orange lines)
    boreholes. Part of this
    englacial warming can be explained by heat diffusion from the base and
    near-surface of the glacier to its coldest depths.

    Theoretical warming due to heat diffusion was approximated by central
    differences on the one-dimensional diffusion equation,
    %
    \begin{equation}
      \frac{\partial T}{\partial t} =
        \frac{k}{\rho c} \frac{\partial^2 T}{\partial z^2}
    \end{equation}
    %
    where $k$ is the thermal conductivity of ice, and $c$ its specific heat
    capacity (Table~\ref{tab:parameters}). However, the observed warming trend
    (Fig.~\ref{fig:profiles}b,
    dashed lines) is up to an order of magnitude higher than the theoretical
    warming by diffusion (Fig.~\ref{fig:profiles}b, solid lines).


% ----------------------------------------------------------------------
\section{Discussion}
% ----------------------------------------------------------------------

% -- -- -- -- -- -- -- -- -- -- -- -- -- -- -- -- -- -- -- -- -- -- -- -
\subsection{Latent heating}
% -- -- -- -- -- -- -- -- -- -- -- -- -- -- -- -- -- -- -- -- -- -- -- -

    Local temperature variations have previously been observed in Arctic
    tidewater glaciers. At 10\,m depth on Hansbreen (northern Spitsbergen),
    ice temperatures 2 to 3\,$^\circ$C higher than the mean annual air
    temperature \citep{Jania.etal.1996} were recorded. In Sermeq Avannarleq
    (West Greenland)
    temperature differences up to 5\,$^\circ$C were measured between two
    boreholes only 86\,m apart down to a depth of ca.~300\,m
    \citep{Luthi.etal.2015}.

    Such temperature differences have been explained by the release of latent
    heat from meltwater refreezing in crevasses, a process sometimes called
    cryo-hydrologic warming \citep{Phillips.etal.2010}. Year after year,
    surface meltwater would penetrate in crevasses during summer and refreeze
    in winter, generating latent heat that diffuses in the
    glacier, potentially reducing ice viscosity and fastening flow
    \citep{Phillips.etal.2013}.  However, such temperature variations have so
    far only been observed in the upper ice layers near the glacier surface.

    Nevertheless, the temperature profiles at Bowdoin Glacier show significant,
    longitudinal differences in temperature over the entire depth of the
    glacier (Fig.~\ref{fig:profiles}). These differences could be a relict
    advected from upstream areas
    of Bowdoin Glacier with ice flow. However in this case, longitudinal
    temperature diffusion should cause both profiles to evolve towards more
    similar temperatures.

    On the opposite, the length of the temperature record allows to evidence an
    englacial warming trend (Fig.~\ref{fig:timeseries}), which can be observed
    at most depths and is up to an order of magnitude stronger than the
    theoretical warming due to vertical heat diffusion
    (Fig.~\ref{fig:profiles}). The warming trend is strongest for the warmest
    temperature profile, thus further enhancing the temperature differences
    between the two profiles (Fig.~\ref{fig:profiles}). The different profiles
    and warming trend can only be explained by involving an additional, local
    heat source, extending over the entire or nearly entire depth of the
    glacier.


% -- -- -- -- -- -- -- -- -- -- -- -- -- -- -- -- -- -- -- -- -- -- -- -
\subsection{Deep crevassing}
% -- -- -- -- -- -- -- -- -- -- -- -- -- -- -- -- -- -- -- -- -- -- -- -

    Although such local variations of temperatures could be explained by the
    proximity of a moulin, no large moulins have been observed in the vicinity
    of the borehole sites or elsewhere on Bowdoin Glacier during the field
    campaigns. Smaller moulins with apparent diameters of ca.~1\,m have
    occasionally been observed on the glacier. However, these moulins would
    most likely refreeze over the winter and thus could not explain the
    observed continuous warming trend nor the entirety of the temperature
    differences between the two drilling sites.

    As a tidewater glacier, Bowdoin Glacier is subject to low basal friction
    \citep{Seddik.etal.2019} and
    thus continuous longitudinal extension yielding to the formation of
    numerous surface transverse crevasses (Figs.~\ref{fig:boreholes} and
    \ref{fig:images}). Surface GPS records indicate that the longitudinal
    extension is most important at lowering tide and correlated with an intense
    seismic activity most likely a symptom of crevasse opening
    \citep{Podolskiy.etal.2016, Podolskiy.etal.2017}.

    The Bowdoin Glacier boreholes were drilled in a highly crevassed area in
    2014 and newly opened crevasses could be observed as the drilling sites
    were
    advected downstream and drifted further apart over the subsequent field
    seasons (Fig.~\ref{fig:images}b). After crevasse initiation, fracture
    propagation to the bed of the glacier is theoretically possible if enough
    water is supplied \citep{Veen.2007}. In fact full-depth crevasse
    propagation through kilometre-thick ice has already been observed in
    association with the drainage of a subglacial lake \citep{Das.etal.2008}.

    If such crevasses form in a tidewater glacier, it is to be expected that
    they would remain filled at least up to sea level even after the drainage.
    If surrounded by cold ice, meltwater filling the crevasses would eventually
    refreeze generating latent heat. Therefore, we interpret the latent heating
    inferred at Bowdoin Glacier to relate to meltwater refreezing in deep
    crevasses reaching at or near the glacier bed.


% -- -- -- -- -- -- -- -- -- -- -- -- -- -- -- -- -- -- -- -- -- -- -- -
\subsection{Ogive banding}
% -- -- -- -- -- -- -- -- -- -- -- -- -- -- -- -- -- -- -- -- -- -- -- -

    Overprinted on the heavy crevassing pattern (Fig.~\ref{fig:ogives}a), the
    topography of Bowdoin Glacier exhibits surface undulations transverse to
    the flow direction. This banding is best visualized on low-light satellite
    images (Fig.~\ref{fig:boreholes}a) or shaded relief images
    (Fig.~\ref{fig:ogives}b). The undulations are advected
    by the movement of the glacier, and thus appear to be related to changes in
    ice thickness rather than reflecting a pattern in the bed topography
    \citep[Fig.~\ref{fig:ogives}c;][Fig.~3]{Tsutaki.etal.2016}. They have a
    wavelength of ca.~350\,m and an amplitude of ca.~10\,m
    (Fig.~\ref{fig:ogives}d). The undulations seem to originate from a
    steeper part of the glacier ca.~8\,km upstream from the calving front
    immediately above the confluence zone of Bowdoin and Obelisk Glaciers
    \citep[Fig.~\ref{fig:images}a;][Fig.~3]{Tsutaki.etal.2016}, and
    could thus be considered as some kind of ogives.

    The upper borehole site (BH1, BH2) is located on a topographic high and the
    lower borehole site (BH3) is located in a topographic low
    (Fig.~\ref{fig:ogives}a). Although this
    was not at all foreseen, the distance between the two boreholes is roughly
    equal to half the ogive wavelength. Because the surface topography of
    Bowdoin Glacier is overprinted by a dense crevasse pattern and abundant
    penitents (Figs.~\ref{fig:images} and~\ref{fig:ogives}a), the undulations
    are not obvious in the field. But in fact it
    could be observed during field seasons that the upper drilling site offered
    a more extensive view than the lower drilling site.

    The lower borehole (BH3), located in a topographic dip, exhibits a warmer
    and faster-warming temperature profile than the upper boreholes (BH1, BH2),
    located on a topographic high. In light of the above observations we
    speculate that areas of lower topography may tend to localise deep
    crevassing. This localisation of crevassing
    could be the result of mechanical damage due to thinner ice, increased
    meltwater infiltration ponding between ogives, or a combination of these
    two processes.


% ----------------------------------------------------------------------
\section{Conclusions}
% ----------------------------------------------------------------------

    Following hot water drilling of the highly crevasse terminal tongue of
    Bowdoin Glacier, northwest Greenland, we present the first continuous,
    multi-year temperature record from one of many ocean-terminating outlet
    glaciers of the Greenland Ice Sheet, which currently experience a
    generalised thinning and accelerated retreat. From this record we identify
    potential new mechanisms that govern the englacial temperature of tidewater
    glaciers and thereby their viscosity.

    \begin{itemize}

      \item Temperature profiles from two drilling sites separated by 165
        (2014) to 197\,m (2017) differ by up to ca.~2\,$^\circ$C indicating
        strong, full-depth longitudinal temperature variations in the glacier.

      \item Englacial warming up to 0.6$^\circ$C in two years, an order of
        magnitude above the theoretical vertical heat diffusion, indicates a
        deep and local heat source within the tidewater glacier.

      \item In absence of visible moulins on the glacier surface, we interpret
        these results as the expression of latent warming from meltwater
        refreezing in crevasses reaching to or near the base of the glacier.

      \item We speculate that the localisation of such deep crevasses may be
        controlled by preferential mechanical damage and meltwater infiltration
        in thinner parts of the glacier associated with ogive banding.

    \end{itemize}

    We would like to point out that these results were not the expected outcome
    of the drilling experiment on Bowdoin Glacier, which was meant to sample
    both the terminal and upper, thicker parts of the glacier for a comparison.
    In fact, they are a somewhat lucky conclusion of a last-minute relocation
    of the second drilling site due to unfavourable weather. Our results are
    therefore limited by the number of sampling points (two), and especially
    the last above conclusion should be validated by a more systematical
    experiment.


% ----------------------------------------------------------------------
% Acknowledgements
% ----------------------------------------------------------------------

\paragraph{Acknowledgements}

    We would like to thank Toku Oshima and Kim Petersen for their warm welcome
    in Qaanaaq, for the shooting lessons and for assistance with field
    preparations. Many thanks to Evgeniy Podolskiy and Lukas Preiswerk for
    their precious help retrieving data from instruments and instruments from
    crevasses. Martin Lüthi provided constructive comments which greatly helped
    to interpret the data. Finally we thank Eef van Dongen for insightful
    discussions and help proofreading this manuscript.

    The current work was supported by the Swiss National Science Foundation
    grants no.~200020-169558 and 200021-153179/1 to M.~Funk.
    % Shin, Conny and Thomy @ ETH

\paragraph{Author contributions}

    MF, SS and AB designed the initial Bowdoin Glacier project.
    TW and CS assembled and calibrated the borehole instruments.
    MF, SS, AB and TW organized the first fieldwork and drilled the boreholes.
    SS processed the DGPS data.
    JS maintained the stations, analyzed the borehole data and wrote most of
    the manuscript.

\paragraph{Conflict of interest}

    The authors declare that they have no conflict of interest.

\paragraph{Contribution to the field}

    \note{Frontiers: ``When you submit your manuscript, you will be required to
          briefly summarize in 200 words your manuscript’s contribution to, and
          position in, the existing literature of your field. This should be
          written avoiding any technical language or non-standard acronyms''.
          Currently 187 words.}
    All around Greenland, glaciers flowing into the ocean react more strongly
    to climate change than the rest of the ice sheet. Because ice is lighter
    than sea water, when these glaciers thin, more and more of their weight is
    taken up by the ocean, allowing them to flow faster. A critical factor in
    the physics of fast glacier flow is the ice temperature. Similarly to honey
    or melting plastic, the viscosity of ice depends strongly on its
    temperature. Warm ice flows faster than cold ice.
    For the first time, we present continuous measurements of ice
    temperature from the fast-flowing and highly crevassed terminal part of a
    Greenlandic ocean-terminating glacier. The measurements indicate the
    presence of a deep, local heat source within the glacier. Our
    interpretation is that every summer, meltwater penetrates deep into the
    glacier through crevasses that extend from the surface to the bottom or
    nearly the bottom of the glacier. The zero-degree water then refreezes in
    winter, warming up the surrounding ice and thus making it softer.
    Our results shed light on a new mechanism contributing to the observed
    acceleration of ocean-terminating glaciers in Greenland.


\paragraph{Data availability}

    The Bowdoin Glacier temperature data will be made available in a public
    repository upon publication of this manuscript.


% ----------------------------------------------------------------------
% References
% ----------------------------------------------------------------------

\bibliographystyle{frontiersinSCNS_ENG_HUMS}
\bibliography{../../references/references}


% ----------------------------------------------------------------------
% Figures
\clearpage
% ----------------------------------------------------------------------

    \begin{figure}
      \centerline{\includegraphics{bowtem_boreholes}}
      \caption{%
        \textbf{(a)} Bowdoin borehole locations from drilling in July 2014 to
          dismantling in July 2017 and background satellite image from 2017
          Mar. 10, 17:41:29 UTC. Contains modified Copernicus Sentinel data,
          processed with Sentinelflow.
        \textbf{(b)} Initial ice thickness and sensor depths. Inclinometers
          and piezometers are also equipped with thermistors.}
      \label{fig:boreholes}
    \end{figure}

    \begin{figure}
      \centerline{\includegraphics{bowtem_images}}
      \caption{%
        \textbf{(a)} Bowdoin Glacier panoramic view on 2015 July 17. The
          calving front is ca.~3\,km wide and 50\,m high. Note the steeper
          section of the glacier above the confluence of Obelisk Glacier
        \textbf{(b)} Bowdoin Glacier lower drilling site and BH3 data loggers
          for thermistor strings and the basal piezometers on 2016 July~19. The
          data logger for the digital inclinometers was removed after cable
          damage related to the opening of the crevasse pictured.
        \textbf{(c)} Aerial view of the drilling sites on 2016 July~21. The
          location of BH1 and BH3 could be inferred from visible field
          installations.}
      \label{fig:images}
    \end{figure}

    \begin{figure}
      \centerline{\includegraphics{bowtem_timeseries}}
      \caption{%
        Ice temperature time series from drilling in July 2014 to dismantling
        in July 2017 (solid lines), and manual temperature readings (circles,
        with outliers marked as empty circles, and 2017 values off the graph).
        Field campaigns (orange spans) and dates selected for
        temperature profiles (dashed lines; Fig.~\ref{fig:profiles}) are
        indicated.}
      \label{fig:timeseries}
    \end{figure}

    \begin{figure}
      \centerline{\includegraphics{bowtem_profiles}}
      \caption{%
        \textbf{(a)} Daily mean temperature profiles for selected dates
          (Fig.~\ref{fig:timeseries}) ca.~three months after the drilling
          (solid lines) and towards the end of the record (dashed lines).
          Minimum observed non-seasonal temperatures are indicated.
        \textbf{(b)} Theoretical temperature diffusion (solid lines) and
          observed temperature changes (dashed lines) for the corresponding
          time period. The maximum observed warming is indicated. All curves
          are interpolated using cubic splines.}
      \label{fig:profiles}
    \end{figure}

    \begin{figure}
      \centerline{\includegraphics{bowtem_ogives}}
      \caption{%
        \textbf{(a)} Glacier surface topography in the vicinity of the
          boreholes sites in the early phase of the experiment (Arctic DEM,
          2014 Sep.~5) and corresponding locations of the boreholes. The
          locations of BH2 and BH3 are estimated from their initial locations
          in July 2014 (color crosses) and the displacement of BH1 measured by
          continuous GPS during the corresponding time intervals.
        \textbf{(b)} Multiple illumination shaded relief map of the terminal
          tongue of Bowdoin Glacier from the same data.
        \textbf{(c)} Elevation change between 2014 Sep.~5 and 2016 Apr.~24
          (Arctic DEM) showing generalised thinning, local thinning near the
          calving front, and the advection of glacier surface undulations.
        \textbf{(d)} Glacier surface topographic profile along a flow line
          passing near BH1 and the estimated location of BH3.}
      \label{fig:ogives}
    \end{figure}


% ----------------------------------------------------------------------
% Tables
\clearpage
% ----------------------------------------------------------------------

    \begin{table}[t]
      \caption{%
        Drilling locations, drilling dates and initial depths of the Bowdoin
        Glacier boreholes.}
      \label{tab:drilling}
      \centerline{\begin{tabular}{cccccc}
        \hline
        Hole & Depth & Date       & Latitude  & Longitude  & Elev. \\
        \hline
        BH1  & 272   & 2014-07-16 & 77.691244 & -68.555749 & 88.7 \\ % UI
        BH2  & 262   & 2014-07-17 & 77.691307 & -68.555685 & 87.7 \\ % UPT
        BH3  & 252   & 2014-07-23 & 77.689995 & -68.558857 & 83.4 \\ % LIPT
        \hline
      \end{tabular}}
    \end{table}

    \begin{table}
      \centerline{\begin{minipage}{85mm}
        \caption{%
          Temperature calibration coefficients and melt offset corrections.}
        \label{tab:calibration}
        \begin{tabular}{ccccc} \\\\\\
          \hline
          Sensor & $a_0\times10^{3}$ & $a_1\times10^{4}$
                 & $a_3\times10^{7}$ & $\Delta T$ (K) \\
          \hline
          LI03 &   ---   &   ---   &   ---   & -0.365214 \\
          LI04 &   ---   &   ---   &   ---   &  0.152383 \\
          LI05 &   ---   &   ---   &   ---   &  0.080331 \\
          LP   & 1.4051  & 2.369   & 1.019   &  0.011500 \\
          LT01 & 2.68849 & 2.91312 & 2.90792 & -0.146755 \\
          LT02 & 2.72120 & 2.77064 & 6.36107 & -0.113682 \\
          LT03 & 2.72538 & 2.75376 & 6.65425 & -0.155104 \\
          LT04 & 2.72203 & 2.75468 & 6.66331 & -0.157252 \\
          LT05 & 2.72629 & 2.74648 & 6.70740 & -0.099604 \\
          LT06 & 2.71837 & 2.77063 & 6.77034 & -0.088374 \\
          LT07 & 2.71239 & 2.80048 & 5.56011 & -0.113634 \\
          LT08 & 2.71464 & 2.79042 & 5.78066 & -0.106662 \\
          LT09 & 2.71831 & 2.78497 & 5.80214 & -0.116135 \\
          LT10 & 2.73324 & 2.72923 & 7.50070 & -0.120984 \\
          LT11 & 2.72182 & 2.77915 & 5.88847 & -0.078037 \\
          LT12 & 2.71306 & 2.80286 & 5.33636 & -0.085440 \\
          LT13 & 2.75088 & 2.66886 & 8.65715 & -0.098275 \\
          LT14 & 2.71922 & 2.79321 & 5.78862 & -0.126088 \\
          \hline
        \end{tabular}
      \end{minipage}
      \begin{minipage}{85mm}
        \begin{tabular}{ccccc}
          \hline
          Sensor & $a_1\times10^{3}$ & $a_2\times10^{4}$
                 & $a_3\times10^{7}$ & $\Delta T$ (K) \\
          \hline
          UI02 &   ---   &   ---   &   ---   & -0.336260 \\
          UI03 &   ---   &   ---   &   ---   &     ---   \\
          UI04 &   ---   &   ---   &   ---   & -0.159012 \\
          UI05 &   ---   &   ---   &   ---   &     ---   \\
          UI06 &   ---   &   ---   &   ---   &     ---   \\
          UI07 &   ---   &   ---   &   ---   &     ---   \\
          UP   & 1.4051  & 2.369   & 1.019   &  0.059716 \\
          UT01 & 2.62948 & 3.15882 &-2.11729 &  0.014610 \\
          UT02 & 2.70630 & 2.83255 & 4.85636 &  0.066282 \\
          UT03 & 2.71054 & 2.85092 & 4.36997 &  0.018222 \\
          UT04 & 2.70327 & 2.85200 & 4.38303 &     ---   \\
          UT05 & 2.70416 & 2.83808 & 4.40717 &     ---   \\
          UT06 & 2.67961 & 2.96215 & 1.95166 &     ---   \\
          UT07 & 2.71620 & 2.79459 & 5.60670 &     ---   \\
          UT08 & 2.70382 & 2.83626 & 4.52513 &     ---   \\
          UT09 & 2.72032 & 2.78078 & 5.86472 &     ---   \\
          UT10 & 2.71264 & 2.82098 & 4.98119 &     ---   \\
          UT11 & 2.70714 & 2.83545 & 4.61922 &     ---   \\
          UT12 & 2.69855 & 2.87227 & 3.95361 &     ---   \\
          UT13 & 2.70676 & 2.84831 & 4.31297 &     ---   \\
          UT14 & 2.73315 & 2.73929 & 7.08468 &     ---   \\
          UT15 & 2.70659 & 2.83499 & 4.69136 &     ---   \\
          UT16 & 2.71143 & 2.82174 & 5.01209 &     ---   \\
          \hline
        \end{tabular}
      \end{minipage}}
    \end{table}

    \begin{table}
      \caption{%
        Parameter values used to compute the pressure melting-point and
        theoretical temperature diffusion.}
      \label{tab:parameters}
      \centerline{\begin{tabular}{llrll}
        \hline
        Not.    & Name & Value & Unit \\
        \hline
        $\beta$ & Clapeyron constant
                & $7.9\times10^{-8}$    & K\,Pa$^{-1}$                      \\
        $c$     & Ice specific heat capacity
                & 2009                  & J\,kg$^{-1}$\,K$^{-1}$            \\
        $g$     & Standard gravity
                & 9.80665               & m\,s$^{-2}$                       \\
        $k$     & Ice thermal conductivity
                & 2.1                   & J\,m$^{-1}$\,K$^{-1}$\,s$^{-1}$   \\
        %$L$    & Water latent heat of fusion
        %       & $3.34\times10^5$      & J\,kg$^{-1}$\,K$^{-1}$            \\
        $\rho$  & Ice density
                & 910                   & kg\,m$^{-3}$                      \\
        \hline
      \end{tabular}}
    \end{table}

    \begin{table}
      \caption{%
        Timeline of events including borehole drilling, instrumental failures,
        and final removal.}
      \label{tab:timeline}
      \centerline{\begin{tabular}{lccp{95mm}}
        \hline
        Date       & Hole & Type      & Event \\
        \hline
        2014-07-15 & ---  & Campaign & \emph{2014 summer fieldwork begins} \\
        2014-07-16 & BH1  & Drilling & Contact to the deepest inclinometer
                                       (UI01) is lost. \\
        2014-07-17 & BH2  & Drilling & -- \\
        2014-07-20 & BH2  & Incident & Earlier upper thermistor (UT*) data are
                                       lost accidentally. \\
        2014-07-23 & BH3  & Drilling & Contact to the deepest inclinometers
                                       (LI01, LI02) is lost. \\
        2014-07-28 & BH2  & Battery  & The upper piezometer data logger battery
                                       drains prematurely without notice from
                                       the fieldwork participants. \\
        2014-07-29 & ---  & Campaign & \emph{2014 summer fieldwork ends} \\
        \hline
        2014-10-25 & BH1  & Incident & Contact to the deepest inclinometers
                                       (UI02, UI03) is lost. \\
        \hline
        2015-07-06 & ---  & Campaign & \emph{2015 summer fieldwork begins} \\
        2015-07-09 & BH2  & Upgrade  & The upper piezometer (UP) data logger
                                       gets a new battery. \\
        2015-07-14 & BH2  & Removal  & The upper thermistor strings (UT*) data
                                       logger is removed. \\
        2015-07-20 & ---  & Campaign & \emph{2015 summer fieldwork ends} \\
        \hline
        2015-07-26 & BH2  & Battery  & The new battery for the upper piezometer
                                       (UP) drains again prematurely, perhaps
                                       indicating faulty instruments. \\
        2015-11-12 & BH3  & Incident & Contact to all remaining inclinometers
                                       (LI03, LI04, LI05) is lost due to
                                       a crevasse opening on the surface. \\
        \hline
        2016-07-04 & ---  & Campaign & \emph{2016 summer fieldwork begins} \\
        2016-07-19 & BH3  & Removal  & The lower thermistor strings (LT*) data
                                       logger is removed. \\
        2016-07-21 & ---  & Campaign & \emph{2016 summer fieldwork ends} \\
        \hline
        2017-01-28 & BH1  & Incident & Contact to the deepest inclinometer
                                       (UI06) is lost. \\
        2017-02-03 & BH3  & Incident & The lower piezometer (LP) starts
                                       recording nonsense. \\
        2017-06-21 & BH1  & Incident & Contact to the deepest inclinometer
                                       (UI07) is lost. \\
        ?          & BH2  & Incident & Contact to the thermistor strings (UT*)
                                       is lost. \\
        ?          & BH3  & Incident & Contact to the thermistor strings (LT*)
                                       is lost. \\
        \hline
        2017-07-04 & ---  & Campaign & \emph{2017 summer fieldwork begins} \\
        2017-07-12 & BH1  & Removal  & The upper inclinometer (UI*) data logger
                                       is removed. \\
        2017-07-17 & ---  & Campaign & \emph{2017 summer fieldwork ends} \\
        \hline
      \end{tabular}}
    \end{table}


% ======================================================================
\end{document}
% ======================================================================

\documentclass[utf8]{article}

\usepackage{doi}
\usepackage{authblk}
\usepackage[T1]{fontenc}
\usepackage[utf8]{inputenc}
\usepackage[pdftex]{xcolor}
\usepackage[pdftex]{graphicx}
\usepackage[authoryear,round]{natbib}

% review mode
\usepackage{geometry}
\usepackage{lineno}
\linenumbers
\linespread{1.5}

\graphicspath{{../../figures/}}

\definecolor{c0}{HTML}{1f77b4}
\definecolor{c1}{HTML}{ff7f0e}
\definecolor{c2}{HTML}{2ca02c}
\definecolor{c3}{HTML}{d62728}
\definecolor{c4}{HTML}{9467bd}
\definecolor{c5}{HTML}{8c564b}
\definecolor{c6}{HTML}{e377c2}
\definecolor{c7}{HTML}{7f7f7f}
\definecolor{c8}{HTML}{bcbd22}
\definecolor{c9}{HTML}{17becf}

\newcommand{\idea}[1]{\textcolor{c2}{\emph{[\textbf{IDEA:} #1]}}}
\newcommand{\note}[1]{\textcolor{c0}{\emph{[\textbf{NOTE:} #1]}}}
\newcommand{\todo}[1]{\textcolor{c3}{\emph{[\textbf{TODO:} #1]}}}

\hypersetup{colorlinks, citecolor=c0, linkcolor=c1, urlcolor=c6}

\title{Bowdoin deformation strain heating computation}

\author[1]{Julien Seguinot
           \thanks{Correspondence to seguinot@vaw.baug.ethz.ch}}

\affil[1]{Laboratory of Hydraulics, Hydrology and Glaciology,
          ETH Zürich, Switzerland}


% ======================================================================
\begin{document}
% ======================================================================

%\maketitle

%\begin{abstract}
%\end{abstract}

% ----------------------------------------------------------------------
\section{Results}
% ----------------------------------------------------------------------

% -- -- -- -- -- -- -- -- -- -- -- -- -- -- -- -- -- -- -- -- -- -- -- -
\subsection{Temporal variations}
% -- -- -- -- -- -- -- -- -- -- -- -- -- -- -- -- -- -- -- -- -- -- -- -

    The previous approach has strong limitations.  First, the shear strain is
    negligible over much of the depth profile but may be important near the
    base. However, there are not enough sensors near the base to simply
    interpolate a strain rate profile through the available measurements.
    Instead, one could to fit a SIA velocity profile through to the
    measurements and try to extrapolate shear strain near the base where it
    primarily occurs but no measurements are available.

    Second, both longitudinal and shear strain rates vary a lot through time.
    This is most visible in the continuous, tilt measurements time series
    (Fig.~\ref{fig:strainrates}), but also true for surface velocities
    measured by surface DGPS, which need to be extrapolated to winter using
    other data sources. The nonlinearity in the dissipation equation will
    amplify fluctuations in effective strain, resulting in a higher value for
    strain heating than if one just considers the mean strain rate.

    However, if these time varitions were significant. one could expect to see
    their effects in the ice temperature time-series. The temperature record
    instead seems to show a continuous warming throughout the seasons.


% ----------------------------------------------------------------------
% Acknowledgements
% ----------------------------------------------------------------------

%\paragraph{Acknowledgements}
%\paragraph{Author contributions}
%\paragraph{Conflict of interest}
%\paragraph{Contribution to the field}
%\paragraph{Data availability}


% ----------------------------------------------------------------------
% References
% ----------------------------------------------------------------------

\bibliographystyle{frontiersinSCNS_ENG_HUMS}
\bibliography{../../references/references}


% ----------------------------------------------------------------------
% Figures
\clearpage
% ----------------------------------------------------------------------

    \begin{figure}
      \centerline{\includegraphics{bowdef_strainrates}}
      \caption{%
          Strain rates time series from daily average tilt measurements.}
      \label{fig:strainrates}
    \end{figure}

% ----------------------------------------------------------------------
% Tables
%\clearpage
% ----------------------------------------------------------------------


% ======================================================================
\end{document}
% ======================================================================

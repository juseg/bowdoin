\documentclass[utf8]{article}

\usepackage{doi}
\usepackage{authblk}
\usepackage[T1]{fontenc}
\usepackage[utf8]{inputenc}
\usepackage[pdftex]{xcolor}
\usepackage[pdftex]{graphicx}
\usepackage[authoryear,round]{natbib}

% review mode
\usepackage{geometry}
\usepackage{lineno}
\linenumbers
\linespread{1.5}

\graphicspath{{../../figures/}}

\definecolor{c0}{HTML}{1f77b4}
\definecolor{c1}{HTML}{ff7f0e}
\definecolor{c2}{HTML}{2ca02c}
\definecolor{c3}{HTML}{d62728}
\definecolor{c4}{HTML}{9467bd}
\definecolor{c5}{HTML}{8c564b}
\definecolor{c6}{HTML}{e377c2}
\definecolor{c7}{HTML}{7f7f7f}
\definecolor{c8}{HTML}{bcbd22}
\definecolor{c9}{HTML}{17becf}

\newcommand{\idea}[1]{\textcolor{c2}{\emph{[\textbf{IDEA:} #1]}}}
\newcommand{\note}[1]{\textcolor{c0}{\emph{[\textbf{NOTE:} #1]}}}
\newcommand{\todo}[1]{\textcolor{c3}{\emph{[\textbf{TODO:} #1]}}}

\hypersetup{colorlinks, citecolor=c0, linkcolor=c1, urlcolor=c6}

\title{Bowdoin deformation strain heating computation}

\author[1]{Julien Seguinot
           \thanks{Correspondence to seguinot@vaw.baug.ethz.ch}}

\affil[1]{Laboratory of Hydraulics, Hydrology and Glaciology,
          ETH Zürich, Switzerland}


% ======================================================================
\begin{document}
% ======================================================================

%\maketitle

%\begin{abstract}
%\end{abstract}

% ----------------------------------------------------------------------
\section{Methods}
% ----------------------------------------------------------------------

% -- -- -- -- -- -- -- -- -- -- -- -- -- -- -- -- -- -- -- -- -- -- -- -
\subsection{Theoretical englacial warming}
% -- -- -- -- -- -- -- -- -- -- -- -- -- -- -- -- -- -- -- -- -- -- -- -

    The theoretical englacial warming due to heat diffusion and viscous
    dissipation can be expressed by the temperature evolution equation,
    %
    \begin{equation}
      \rho c \frac{\partial T}{\partial t}
        = k \frac{\partial^2 T}{\partial z^2} + H,
    \end{equation}
    %
    where $T$ is the ice temperature $\rho$ is the ice density, $k$ is the
    thermal conductivity of ice, and $c$ its specific heat capacity
    (Table~\ref{tab:parameters}). The source term, $H$, corresponds to the
    energy dissipation due to strain heating, which can be expressed as
    %
    \begin{equation}
      H = \mathrm{tr}(\tau\dot\epsilon),
    \end{equation}
    %
    where $\tau$ is the deviatoric stress tensor, and $\dot\epsilon$ the
    strain-rate tensor \citep[p.~417]{Clarke.etal.1977, Cuffey.Paterson.2010}.
    Stresses and strains can be related by the constitutive law for ice
    \citep{Glen.1952, Nye.1953},
    %
    \begin{equation}
        \dot{\epsilon} = A\,\tau_e^{n-1}\,\tau \,,
    \end{equation}
    %
    where the equivalent stress, $\tau_e$, is defined by ${\tau_e}^2 =
    \frac{1}{2} \mathrm{tr}(\tau^2)$. The ice softness coefficient, $A$,
    depends on the pressure-adjusted ice temperature, $T$, and pressure, $p$,
    through an Arrhenius-type law, $A = A_0 \,e^\frac{-Q}{RT_{pa}}$, where
    $T_{pa}$ is the pressure-adjusted ice temperature calculated using
    the Clapeyron relation, ${T_{pa} = T - \beta p}$
    \citep[p.~72]{Cuffey.Paterson.2010}.
    The strain heating can then be rewritten as a function of the deviatoric
    stresses or the strain rates only,
    %
    \begin{equation}
        H = 2 A \tau_e^{n+1} = 2 A^{-1/n} \dot\epsilon_e^{1+1/n}.
    \end{equation}

    Assuming two-dimensional, vertical planar deformation, the effective strain
    rate, $\dot\epsilon_e^2 = \frac{1}{2}\mathrm{tr}(\epsilon^2)$, can be
    expressed in terms of its cartesian components, $\dot\epsilon_e^2 =
    \frac{1}{2}(\dot\epsilon_{xx}^2 + \dot\epsilon_{zz}^2) +
    \dot\epsilon_{xz}^2$. The consevation of volume yields
    $\mathrm{div}(\dot\epsilon) = \dot\epsilon_{xx} + \dot\epsilon_{zz} = 0$ so
    that the effective strain can be simplified to a function of its
    longitudinal and shear components,
    %
    \begin{equation}
        \dot\epsilon_e^2 = \dot\epsilon_{xx}^2 + \dot\epsilon_{xz}^2.
    \end{equation}


% ----------------------------------------------------------------------
\section{Results}
% ----------------------------------------------------------------------

% -- -- -- -- -- -- -- -- -- -- -- -- -- -- -- -- -- -- -- -- -- -- -- -
\subsection{Quick estimation}
% -- -- -- -- -- -- -- -- -- -- -- -- -- -- -- -- -- -- -- -- -- -- -- -

    The longitudinal strain can be estimated from the observed borehole
    positions. The distance distance between the lower (BH3) and uppermost
    (BH2) boreholes increased from 165 to 197\,m between 2014 July 20 (midpoint
    between the two observation dates 17 and 23) and 2017 July 17. This
    corresponds to a longitudinal strain rate, $\dot\epsilon_{xx}$, of
    $2.01\times10^9$\,s$^{-1}$. Measured horizontal shear strain rates,
    $\dot\epsilon_{xy}$, are on the order of $3\times10^9$\,s$^{-1}$, yielding
    an effective strain rate, $\dot\epsilon_e$, of $3.61\times10^9$\,s$^{-1}$.

    Assuming a constant ice hardness coefficient, $A$, corresponding to the
    pressure-melting point ice hardness, $A_0$ (Table~\ref{tab:parameters}),
    yields to a slight overestimate of the heat dissipation due to strain
    heating, $H$, at $1.57\times10^3$\,Pa\,s$^{-1}$. The corresponding
    rate of temperature change is $2.60\times10^{-2}$\,$^{\circ}$C\,a$^{-1}$,
    which is negligible as compared to the observed temperature changes.


% ----------------------------------------------------------------------
% Acknowledgements
% ----------------------------------------------------------------------

%\paragraph{Acknowledgements}
%\paragraph{Author contributions}
%\paragraph{Conflict of interest}
%\paragraph{Contribution to the field}
%\paragraph{Data availability}


% ----------------------------------------------------------------------
% References
% ----------------------------------------------------------------------

\bibliographystyle{frontiersinSCNS_ENG_HUMS}
\bibliography{../../references/references}


% ----------------------------------------------------------------------
% Figures
%\clearpage
% ----------------------------------------------------------------------

% ----------------------------------------------------------------------
% Tables
%\clearpage
% ----------------------------------------------------------------------

    \begin{table}
      \caption{%
        Parameter values used to compute the theoretical englacial warming.}
      \label{tab:parameters}
      \centerline{\begin{tabular}{llrlll}
        \hline
        Not.    & Name & Value & Unit & Source \\
        \hline
        $A_0$   & Ice hardness coefficient
                & $3.5\times10^{-25}$
                & Pa$^{-3}$\,s$^{-1}$
                & \citet[p.~74]{Cuffey.Paterson.2010} \\
        $\beta$ & Clapeyron constant
                & $7.9\times10^{-8}$
                & K\,Pa$^{-1}$
                & \citet{Luthi.etal.2002} \\
        $c$     & Ice specific heat capacity
                & 2097
                & J\,kg$^{-1}$\,K$^{-1}$
                & \citet[p.~400]{Cuffey.Paterson.2010} \\
        $g$     & Standard gravity
                & 9.80665
                & m\,s$^{-2}$
                & -- \\
        $k$     & Ice thermal conductivity
                & 2.10
                & J\,m$^{-1}$\,K$^{-1}$\,s$^{-1}$
                & \citet[p.~400]{Cuffey.Paterson.2010} \\
        %$L$     & Water latent heat of fusion
        %        & $3.35\times10^5$      & J\,kg$^{-1}$\,K$^{-1}$
        %        & \citet[p.~400]{Cuffey.Paterson.2010} \\
        $Q$     & Flow law activation energy
                & $116\times10^3$
                & J\,mol$^{-1}$
                & \citet[p.~74]{Cuffey.Paterson.2010} \\
        $R$     & Ideal gas constant
                & 8.31441
                & J\,mol$^{-1}$\,K$^{-1}$
                & -- \\
        $\rho$  & Ice density
                & 917
                & kg\,m$^{-3}$
                & \citet[p.~12]{Cuffey.Paterson.2010} \\
        \hline
      \end{tabular}}
    \end{table}


% ======================================================================
\end{document}
% ======================================================================

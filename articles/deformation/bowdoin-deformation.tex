\documentclass[utf8]{article}

\usepackage{doi}
\usepackage{authblk}
\usepackage[T1]{fontenc}
\usepackage[utf8]{inputenc}
\usepackage[pdftex]{xcolor}
\usepackage[pdftex]{graphicx}
\usepackage[authoryear,round]{natbib}

% review mode
\usepackage{geometry}
\usepackage{lineno}
\linenumbers
\linespread{1.5}

\graphicspath{{../../figures/}}

\definecolor{c0}{HTML}{1f77b4}
\definecolor{c1}{HTML}{ff7f0e}
\definecolor{c2}{HTML}{2ca02c}
\definecolor{c3}{HTML}{d62728}
\definecolor{c4}{HTML}{9467bd}
\definecolor{c5}{HTML}{8c564b}
\definecolor{c6}{HTML}{e377c2}
\definecolor{c7}{HTML}{7f7f7f}
\definecolor{c8}{HTML}{bcbd22}
\definecolor{c9}{HTML}{17becf}

\newcommand{\idea}[1]{\textcolor{c2}{\emph{[\textbf{IDEA:} #1]}}}
\newcommand{\note}[1]{\textcolor{c0}{\emph{[\textbf{NOTE:} #1]}}}
\newcommand{\todo}[1]{\textcolor{c3}{\emph{[\textbf{TODO:} #1]}}}

\hypersetup{colorlinks, citecolor=c0, linkcolor=c1, urlcolor=c6}

\title{Bowdoin deformation strain heating computation}

\author[1]{Julien Seguinot
           \thanks{Correspondence to seguinot@vaw.baug.ethz.ch}}

\affil[1]{Laboratory of Hydraulics, Hydrology and Glaciology,
          ETH Zürich, Switzerland}


% ======================================================================
\begin{document}
% ======================================================================

%\maketitle

%\begin{abstract}
%\end{abstract}

% ----------------------------------------------------------------------
%\section{Methods}
% ----------------------------------------------------------------------

% -- -- -- -- -- -- -- -- -- -- -- -- -- -- -- -- -- -- -- -- -- -- -- -
\subsection{Strain heating computation}
% -- -- -- -- -- -- -- -- -- -- -- -- -- -- -- -- -- -- -- -- -- -- -- -

    The energy dissipation due to strain heating can be expressed by 
    %
    \begin{equation}
        H = \mathrm{tr}(\tau\dot\epsilon).
    \end{equation}

    Let us applying the constitutive law for ice \citep{Glen.1952, Nye.1953},
    %
    \begin{equation}
        \dot{\epsilon} = A\,\tau_e^{n-1}\,\tau \,,
    \end{equation}
    %
    where the equivalent stress, $\tau_e$, is defined by ${\tau_e}^2 =
    \frac{1}{2} \mathrm{tr}(\tau^2)$, and the ice softness coefficient, $A$,
    depends on the pressure-adjusted ice temperature, $T_{pa}$, through an
    Arrhenius-type law: $A = A_0 \,e^\frac{-Q}{RT_{pa}}$
    \citep[Eqs.~63--65]{Paterson.Budd.1982, Aschwanden.etal.2012}.
    The strain heating can be rewritten as a function of pure deviatoric
    stresses or strain rates,
    \begin{equation}
        H = 2 A \tau_e^{n+1} = 2 A^{-1/n} \dot\epsilon_e^{1+1/n}.
    \end{equation}


% ----------------------------------------------------------------------
% Acknowledgements
% ----------------------------------------------------------------------

%\paragraph{Acknowledgements}
%\paragraph{Author contributions}
%\paragraph{Conflict of interest}
%\paragraph{Contribution to the field}
%\paragraph{Data availability}


% ----------------------------------------------------------------------
% References
% ----------------------------------------------------------------------

%\bibliographystyle{frontiersinSCNS_ENG_HUMS}
%\bibliography{../../references/references}


% ----------------------------------------------------------------------
% Figures
%\clearpage
% ----------------------------------------------------------------------

% ----------------------------------------------------------------------
% Tables
%\clearpage
% ----------------------------------------------------------------------

% ======================================================================
\end{document}
% ======================================================================

\documentclass[utf8]{article}

\usepackage{doi}
\usepackage{authblk}
\usepackage[T1]{fontenc}
\usepackage[utf8]{inputenc}
\usepackage[pdftex]{xcolor}
\usepackage[pdftex]{graphicx}
%\usepackage[authoryear,round]{natbib}

% review mode
\usepackage{geometry}
\usepackage{lineno}
\linenumbers
\linespread{1.5}

\graphicspath{{../../figures/}}

\definecolor{c0}{HTML}{1f77b4}
\definecolor{c1}{HTML}{ff7f0e}
\definecolor{c2}{HTML}{2ca02c}
\definecolor{c3}{HTML}{d62728}
\definecolor{c4}{HTML}{9467bd}
\definecolor{c5}{HTML}{8c564b}
\definecolor{c6}{HTML}{e377c2}
\definecolor{c7}{HTML}{7f7f7f}
\definecolor{c8}{HTML}{bcbd22}
\definecolor{c9}{HTML}{17becf}

\newcommand{\idea}[1]{\textcolor{c2}{\emph{[\textbf{IDEA:} #1]}}}
\newcommand{\note}[1]{\textcolor{c0}{\emph{[\textbf{NOTE:} #1]}}}
\newcommand{\todo}[1]{\textcolor{c3}{\emph{[\textbf{TODO:} #1]}}}

\hypersetup{colorlinks, citecolor=c0, linkcolor=c1, urlcolor=c6}

\title{A coincidental measure of tidal stress propagation\\
       in Bowdoin Glacier, northwestern Greenland}

\author[1]{Julien Seguinot}
\author[1]{Martin Funk}
\author[2]{Evgeniy Podoslkiy}
\author[3]{Cornelius Senn}
\author[3]{Shin Sugiyama}

\affil[1]{Laboratory of Hydraulics, Hydrology and Glaciology,
          ETH Zürich, Switzerland}
\affil[2]{Arctic Research Center, Hokkaido University, Sapporo, Japan}
\affil[3]{Department of Civil, Environmental and Geomatic Engineering,
          ETH Zürich, Switzerland}
\affil[4]{Institute of Low Temperature Science, Hokkaido University,
          Sapporo, Japan}


% ======================================================================
\begin{document}
% ======================================================================

\maketitle

\begin{abstract}

    The observed acceleration, thinning and retreat of marine-terminating
    outlet glaciers of the Greenland ice sheet account for about half of its
    mass loss. This so-called dynamic thinning, which has now propagated along
    much of the ice margin, nevertheless involves feedback processes between
    ice thickness, basal sliding, subglacial water and oceanic tides, that are
    not yet fully understood.
    %
    Here, we present a three-year record of ice pressure measured en-glacially
    at Bowdoin Glacier in northwestern Greenland. Although Bowdoin Glacier
    appears to have been very stable since its frontal position was first
    documented by late 19th century explorers, its calving front has recently
    experienced a rapid retreat of ca. 2\,km between 2007 and 2013, followed by
    a continued surface lowering since then.
    %
    About 2\,km upstream from the calving front, two boreholes were drilled and
    equipped with thermistors, inclinometers and pressure sensors. Although
    pressure sensors were meant to locate instruments in the water-filled
    boreholes immediately after the drilling, they continued to record after
    the complete refreezing of the boreholes and the stabilisation of ice
    temperatures well below the pressure melting point. All sensors recorded
    in-phase pressure variations with clear 12-hour and 14-day periodicities,
    leaving no doubt on the tidal nature of the signal. Surprisingly, the
    amplitude of pressure variations recorded in ice is comparable to that of
    sea level tides measured 130 km away at Pituffik. This lets us conclude,
    that tidal stresses applying to marine calving fronts can propagate several
    kilometres upstream in subfreezing glacier ice.

\end{abstract}

% ----------------------------------------------------------------------
%\section{}
% ----------------------------------------------------------------------

% -- -- -- -- -- -- -- -- -- -- -- -- -- -- -- -- -- -- -- -- -- -- -- -
%\subsection{}
% -- -- -- -- -- -- -- -- -- -- -- -- -- -- -- -- -- -- -- -- -- -- -- -


% ----------------------------------------------------------------------
% Acknowledgements
% ----------------------------------------------------------------------

%\paragraph{Acknowledgements}
%\paragraph{Author contributions}
%\paragraph{Conflict of interest}
%\paragraph{Contribution to the field}
%\paragraph{Data availability}


% ----------------------------------------------------------------------
% References
% ----------------------------------------------------------------------

%\bibliographystyle{frontiersinSCNS_ENG_HUMS}
%\bibliography{../../references/references}


% ----------------------------------------------------------------------
% Figures
%\clearpage
% ----------------------------------------------------------------------

% ----------------------------------------------------------------------
% Tables
%\clearpage
% ----------------------------------------------------------------------


% ======================================================================
\end{document}
% ======================================================================

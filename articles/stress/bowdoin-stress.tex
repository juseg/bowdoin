\documentclass[utf8]{article}

\usepackage{doi}
\usepackage{authblk}
\usepackage[T1]{fontenc}
\usepackage[utf8]{inputenc}
\usepackage[pdftex]{xcolor}
\usepackage[pdftex]{graphicx}
%\usepackage[authoryear,round]{natbib}

% review mode
\usepackage{geometry}
\usepackage{lineno}
\linenumbers
\linespread{1.5}

\graphicspath{{../../figures/}}

\definecolor{c0}{HTML}{1f77b4}
\definecolor{c1}{HTML}{ff7f0e}
\definecolor{c2}{HTML}{2ca02c}
\definecolor{c3}{HTML}{d62728}
\definecolor{c4}{HTML}{9467bd}
\definecolor{c5}{HTML}{8c564b}
\definecolor{c6}{HTML}{e377c2}
\definecolor{c7}{HTML}{7f7f7f}
\definecolor{c8}{HTML}{bcbd22}
\definecolor{c9}{HTML}{17becf}

\newcommand{\idea}[1]{\textcolor{c2}{\emph{[\textbf{IDEA:} #1]}}}
\newcommand{\note}[1]{\textcolor{c0}{\emph{[\textbf{NOTE:} #1]}}}
\newcommand{\todo}[1]{\textcolor{c3}{\emph{[\textbf{TODO:} #1]}}}

\hypersetup{colorlinks, citecolor=c0, linkcolor=c1, urlcolor=c6}

\title{A coincidental measure of englacial stress \\
       in Bowdoin Glacier, Greenland}

\author[1]{Julien Seguinot}
\author[2]{Evgeniy Podoslkiy}
\author[ ]{others}
%\author[1]{Martin Funk}
%\author[3]{Cornelius Senn}
%\author[3]{Shin Sugiyama}

\affil[1]{Independent scholar, Anafi, Greece}
\affil[2]{Arctic Research Center, Hokkaido University, Sapporo, Japan}
%\affil[3]{Department of Civil, Environmental and Geomatic Engineering,
%          ETH Zürich, Switzerland}
%\affil[4]{Institute of Low Temperature Science, Hokkaido University,
%          Sapporo, Japan}


% ======================================================================
\begin{document}
% ======================================================================

\maketitle

\begin{abstract}

    \note{preliminary abstract.}
    %
    While compressive stresses are understood to be critical variable governing
    glacier motion, they have rarely been measured in nature. Instead, the
    empirical understanding of glacier stress typically relies on laboratory
    flow-law experiments and indirect measurements of glacier surface velocity
    and englacial tilt rates. Nevertheless, there is no lack of observational
    evidence that cristal orientation, ice water content and impurities may
    complicate laboratory-derived laws in nature.
    %
    Here, we present an accidental, three-year record of englacial stress
    from water-pressure sensors frozen into a Greenlandic tidewater glacier.
    While the sensors, meant to locate instruments in hotwater-drilled
    boreholes, were not calibrated for solid stress measurements, they
    continued to record pressure changes after the complete refreezing of the
    boreholes and the stabilisation of ice temperatures well below the pressure
    melting point.
    %
    All sensors recorded in-phase pressure changes with 12-hour, 24-hour and
    14-day periodicities, leaving no doubt on the tidal nature of the signal.
    Surprisingly, the amplitude of these pressure variations, recorded one to
    two kilometres upstream the glacier front, is only an order of magnitude
    lower than the tidal amplitude measured at sea. However, glacier stress is
    anticorrelated with the tides, and show a delay of one to two hours, so
    that maximum stress occurs a little after low tide.

\end{abstract}

% ----------------------------------------------------------------------
\section{Introduction}
% ----------------------------------------------------------------------

    Most glaciers lead lives of constant stress, so they relax by slowly
    gliding on their bed.

% ----------------------------------------------------------------------
\section{Methods}
% ----------------------------------------------------------------------

    Borehole set-up (Fig.~\ref{fig:boreholes}).

% ----------------------------------------------------------------------
\section{Results}
% ----------------------------------------------------------------------

% -- -- -- -- -- -- -- -- -- -- -- -- -- -- -- -- -- -- -- -- -- -- -- -
\subsection{Stress timeseries}
% -- -- -- -- -- -- -- -- -- -- -- -- -- -- -- -- -- -- -- -- -- -- -- -

    Stress timeseries (Fig.~\ref{fig:timeseries}).
    Filtered timeseries (Fig.~\ref{fig:highpass}).

% -- -- -- -- -- -- -- -- -- -- -- -- -- -- -- -- -- -- -- -- -- -- -- -
\subsection{Spectral content}
% -- -- -- -- -- -- -- -- -- -- -- -- -- -- -- -- -- -- -- -- -- -- -- -

    Fourier transforms (Fig.~\ref{fig:fourier}).
    Fourier spectrograms (Fig.~\ref{fig:specgrams}).
    Continuous wavelet transform (Fig.~\ref{fig:wavelets}).

% -- -- -- -- -- -- -- -- -- -- -- -- -- -- -- -- -- -- -- -- -- -- -- -
\subsection{Phase relationships}
% -- -- -- -- -- -- -- -- -- -- -- -- -- -- -- -- -- -- -- -- -- -- -- -

    Cross-correlation (Fig.~\ref{fig:correlate}).
    Rolling window cross-correlation (Fig.~\ref{fig:rollcorr}).

% ----------------------------------------------------------------------
\section{Discussion}
% ----------------------------------------------------------------------

% -- -- -- -- -- -- -- -- -- -- -- -- -- -- -- -- -- -- -- -- -- -- -- -
% \subsection{}
% -- -- -- -- -- -- -- -- -- -- -- -- -- -- -- -- -- -- -- -- -- -- -- -

% ----------------------------------------------------------------------
\section{Conclusions}
% ----------------------------------------------------------------------


% ----------------------------------------------------------------------
% Acknowledgements
% ----------------------------------------------------------------------

%\paragraph{Acknowledgements}
%\paragraph{Author contributions}
%\paragraph{Conflict of interest}
%\paragraph{Contribution to the field}
%\paragraph{Data availability}


% ----------------------------------------------------------------------
% References
% ----------------------------------------------------------------------

%\bibliographystyle{frontiersinSCNS_ENG_HUMS}
%\bibliography{../../references/references}


% ----------------------------------------------------------------------
% Figures
%\clearpage
% ----------------------------------------------------------------------

    \begin{figure}
      \centerline{\includegraphics{bowstr_boreholes}}
      \caption{%
        \textbf{(a)}
          Bowdoin borehole locations from drilling in July 2014 to dismantling
          in July 2017 and background satellite image from 2017 Mars 10,
          17:41:29 UTC. Contains modified Copernicus Sentinel data, processed
          with Sentinelflow.
        \textbf{(b)}
          Initially observed ice thickness and localization of the piezometers
          as deduced from initial water-pressure measurements.
        \todo{add model of tilt unit casing with location of the sensor}.}
      \label{fig:boreholes}
    \end{figure}

    \begin{figure}
      \centerline{\includegraphics{bowstr_timeseries}}
      \caption{%
        \textbf{(a)}
          Complete piezometer record, including the initial borehole water
          pressure measurements, and the transition to solid ice stress
          measurements. Sharp peaks in the early part of the record correspond
          to the freezing phase recorded by thermistors.
        \textbf{(b, c)}
          Insets displaying an example of semi-diurnal and 14-day periodicities
          observed in the stress record after refreezing.}
      \label{fig:timeseries}
    \end{figure}

    \begin{figure}
      \centerline{\includegraphics{bowstr_highpass}}
      \caption{%
        \textbf{(a)}
          Complete piezometer record, after applying a fourth-order Butterworth
          high-pass filter with a cut-off period of one day. The bottom curve
          shows (non-filtered) tidal measurements from Pituffik, converted to
          sea-water pressure and divided by ten.
        \textbf{(b)}
          Zoom on the later part of the 2014 melt season and transition into
          fall, before contact to UI03 and UI02 was lost, but also while LI03
          was not yet refrozen. Other sensors record a transition from
          high-amplitude, daily stress oscillations (correlated with glacier
          speed) to semi-diurnal oscillations anti-correlated with, and an
          order of magnitude lower than, the tide.}
      \label{fig:highpass}
    \end{figure}

    \begin{figure}
      \centerline{\includegraphics{bowstr_fourier}}
      \caption{%
        \textbf{(a--i)}
          Fast-fourier transforms of the (non-filtered) series of stress
          derivative over time (kPa$\,$s$^{-1}$) after refreezing. The
          Refreezing date was computed independently for each unit as the date
          when temperatures reach $75\%$ of their minumum value relative to
          0$\,$C$^\circ$. \todo{ensure this is consistent among plots.}
        \textbf{(j)}
          Fast-fourier transform of the time derivative of Pituffik tides,
          converted to sea-water pressure and divided by ten. Insets show a
          zoom on the diurnal and semi-diurnal tidal components which can also
          be found in some of the stress records.}
      \label{fig:fourier}
    \end{figure}

    \begin{figure}
      \centerline{\includegraphics{bowstr_specgrams}}
      \caption{%
        \textbf{(a--h)}
          Rolling-window spectrograms of glacier stress and tidal pressure
          temporal derivatives after refreezing of the individual units.
          Fourier transforms are computed on 14-day windows with a 12-day
          overlap. Some of the vertical bands result from interpolating the
          signal from solar-power-adaptive to a constant, ten-minute time step.
          Color curves indicate the relative ratio of spectral power in the
          10--14 and 22--26\,h bands. UI03 and UI02 are omitted due to the
          short length of the post-refreezing solid stress record.}
      \label{fig:specgrams}
    \end{figure}

    \begin{figure}
      \centerline{\includegraphics{bowstr_wavelets}}
      \caption{%
        \textbf{(a--h)}
          Continuous wavelet transforms of glacier stress and tidal pressure
          temporal derivatives after applying a fourth-order Butterworth
          high-pass filter with a cut-off period of one day \todo{maybe
          filtering is not needed here}, covering available records during the
          2015 melt season.}
      \label{fig:wavelets}
    \end{figure}

    \begin{figure}
      \centerline{\includegraphics{bowstr_correlate}}
      \caption{%
        \textbf{(a)}
          Extract of the piezometer record and tidal pressure for October 2014,
          after applying a fourth-order Butterworth band-pass filter with
          cut-off periods of 2 and 12\,h \todo{maybe use the same filter in all
          plots?}.
        \textbf{(b)}
          Cross-correlation of the ice stress (and water pressure) time series
          against tidal pressure. Coloured dots indicate the maximum absolute
          correlation, which corresponds to an anti-correlation with a phase
          delay of ca.~1--2\,h.
        \textbf{(c)}
          Optimal phase delay as a function of sensor depth.
        \todo{
          I need to check that the clocks are on-time (at least UI* should be
          OK, but I'm not sure about LI*).}}
      \label{fig:correlate}
    \end{figure}

    \begin{figure}
      \centerline{\includegraphics{bowstr_rollcorr}}
      \caption{%
        \textbf{(a-g)}
          Rolling-window cross-correlation between the piezometer and tidal
          pressure record when the latter is available, after applying a
          fourth-order Butterworth band-pass filter with cut-off periods of 2
          and 12\,h. Colour curves show the optimal phase delay of stress
          records relative to tides when absolute maximum (anti)correlation is
          above 0.75.}
      \label{fig:rollcorr}
    \end{figure}
% ----------------------------------------------------------------------
% Tables
%\clearpage
% ----------------------------------------------------------------------


% ======================================================================
\end{document}
% ======================================================================

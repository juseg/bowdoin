\documentclass[utf8]{article}

\usepackage{doi}
\usepackage{authblk}
\usepackage[T1]{fontenc}
\usepackage[utf8]{inputenc}
\usepackage[pdftex]{xcolor}
\usepackage[pdftex]{graphicx}
\usepackage[authoryear,round]{natbib}

% review mode
\usepackage{geometry}
\usepackage{lineno}
\linenumbers
\linespread{1.5}

\graphicspath{{../../figures/}}

\definecolor{c0}{HTML}{1f77b4}
\definecolor{c1}{HTML}{ff7f0e}
\definecolor{c2}{HTML}{2ca02c}
\definecolor{c3}{HTML}{d62728}
\definecolor{c4}{HTML}{9467bd}
\definecolor{c5}{HTML}{8c564b}
\definecolor{c6}{HTML}{e377c2}
\definecolor{c7}{HTML}{7f7f7f}
\definecolor{c8}{HTML}{bcbd22}
\definecolor{c9}{HTML}{17becf}

\newcommand{\idea}[1]{\textcolor{c2}{\emph{[\textbf{IDEA:} #1]}}}
\newcommand{\note}[1]{\textcolor{c0}{\emph{[\textbf{NOTE:} #1]}}}
\newcommand{\todo}[1]{\textcolor{c3}{\emph{[\textbf{TODO:} #1]}}}

\hypersetup{colorlinks, citecolor=c0, linkcolor=c1, urlcolor=c6}

\title{Tide-modulated englacial stress measured at Bowdoin~Glacier, Greenland}

\author[1]{Julien Seguinot}
\author[2]{Evgeny A. Podolskiy}
\author[2, 3]{Shin Sugiyama}
\author[4]{Martin Funk}
\author[4]{Andreas Bauder}
\author[5]{Cornelius Senn}
\author[3]{Ralf Greve}
\author[1]{Harry Zekollari}

%\author[1]{Katarina Henning}
%\author[6]{Silvan Leinss}
%\author[7]{Daiki Sakakibara}

\affil[1]{Department of Water and Climate, Vrije Universiteit Brussel,
          Brussels, Belgium}
\affil[2]{Arctic Research Center, Hokkaido University, Sapporo, Japan}
\affil[3]{Institute of Low Temperature Science, Hokkaido University,
          Sapporo, Japan}
\affil[4]{Laboratory of Hydraulics, Hydrology and Glaciology, ETH Zurich, Zurich, Switzerland}
\affil[5]{Department of Civil, Environmental and Geomatic Engineering, ETH Zurich, Zurich, Switzerland}

% \affil[6]{GAMMA Remote Sensing, Gümligen, Switzerland}
% \affil[7]{Guide Office ROUTE, Sapporo, Japan}


% ======================================================================
\begin{document}
% ======================================================================

\maketitle

\begin{abstract}

    \note{abstract from ISAR-8.}
    %
    Englacial stress, the elusive variable governing glacier motion, has rarely
    been measured in situ. Instead, our empirical understanding of ice dynamics
    largely relies on laboratory flow-law experiments, but field measurements
    of stress-induced glacier surface velocity and englacial tilt indicate that
    crystal orientation, molten ice fraction and impurities may complicate the
    application of laboratory-derived laws in nature.
    %
    Here, we present a three-year stress record from piezometers frozen 123 to
    265 metres deep into the Bowdoin tidewater glacier in Greenland. These
    sensors were not initially intended to measure solid stress, but instead to
    locate instruments in hotwater-drilled boreholes. However, they continued
    to record pressure changes after the complete refreezing of the boreholes
    and the stabilisation of ice temperatures well below the melting point.
    %
    All sensors recorded in-phase stress variations with 12-hour, 24-hour and
    14-day periodicities, revealing a tidal signal in winter, disturbed during
    independently documented speed-up events in summer. The signal shows
    amplitudes of one to four kilopascals, only an order of magnitude weaker
    than the two metres tidal amplitude measured at sea. However, stress
    measurements are anticorrelated with the tide, and show a delay of one to
    two hours, so that maximum stresses occur a little after low tide. While
    detailed interpretations are hampered by to the lack of calibration, our
    data indicate that direct stress measurements in glaciers are feasible.

\end{abstract}


% ----------------------------------------------------------------------
\section{Introduction}
% ----------------------------------------------------------------------

    Most glaciers lead lives of constant stress, so they relax by slowly
    gliding on their bed.

% ----------------------------------------------------------------------
\section{Methods}
% ----------------------------------------------------------------------

    Borehole set-up (Fig.~\ref{fig:bores}).

% ----------------------------------------------------------------------
\section{Results}
% ----------------------------------------------------------------------

% -- -- -- -- -- -- -- -- -- -- -- -- -- -- -- -- -- -- -- -- -- -- -- -
\subsection{Stress timeseries}
% -- -- -- -- -- -- -- -- -- -- -- -- -- -- -- -- -- -- -- -- -- -- -- -

    Stress timeseries (Fig.~\ref{fig:nofil}).

    Ice-cased sensors recorded stress oscillations.
    The amplitude is an order of magnitude below Pituffik tides.

% -- -- -- -- -- -- -- -- -- -- -- -- -- -- -- -- -- -- -- -- -- -- -- -
\subsection{Spectral content}
% -- -- -- -- -- -- -- -- -- -- -- -- -- -- -- -- -- -- -- -- -- -- -- -

    Fourier transforms (Fig.~\ref{fig:pgram}).
    Fourier spectrograms (Fig.~\ref{fig:sgram}).
    Continuous wavelet transform (Fig.~\ref{fig:wlets}).

    Spectral content is very similar to that of Pituffik tides.
    Tidal signal is overprinted by daily oscillations in summer.
    Untested: especially during speed-up events, I think.

% -- -- -- -- -- -- -- -- -- -- -- -- -- -- -- -- -- -- -- -- -- -- -- -
\subsection{Phase relationships}
% -- -- -- -- -- -- -- -- -- -- -- -- -- -- -- -- -- -- -- -- -- -- -- -

    Filtered timeseries (Fig.~\ref{fig:lines}).
    Cross-correlation (Fig.~\ref{fig:ccorr}).
    Rolling window cross-correlation (Fig.~\ref{fig:mcorr}).

    Stress signal is anti-correlated with the tide.
    Maximum stress occur 1 to 2 hours after low tide.
    This phase delay appears to peak around 200 meter depth.
    The delay is rather stable in 2014-16 but gets messy in 16-17.


% ----------------------------------------------------------------------
\section{Discussion}
% ----------------------------------------------------------------------

% -- -- -- -- -- -- -- -- -- -- -- -- -- -- -- -- -- -- -- -- -- -- -- -
% \subsection{}
% -- -- -- -- -- -- -- -- -- -- -- -- -- -- -- -- -- -- -- -- -- -- -- -

% ----------------------------------------------------------------------
\section{Conclusions}
% ----------------------------------------------------------------------


% ----------------------------------------------------------------------
% Acknowledgements
% ----------------------------------------------------------------------

%\paragraph{Acknowledgements}
%\paragraph{Author contributions}
%\paragraph{Conflict of interest}
%\paragraph{Contribution to the field}
%\paragraph{Data availability}


% ----------------------------------------------------------------------
% References
% ----------------------------------------------------------------------

\bibliographystyle{abbrvnat}
\bibliography{../../../references/references}


% ----------------------------------------------------------------------
% Figures
\clearpage
% ----------------------------------------------------------------------

    \begin{figure}
      \centerline{\includegraphics{bowstr_bores}}
      \caption{%
        \textbf{(a)}
          Bowdoin borehole locations from drilling in July 2014 to dismantling
          in July 2017 and background satellite image from 2017 March 10,
          17:41:29 UTC. Contains modified Copernicus Sentinel data, processed
          with Sentinelflow.
        \textbf{(b)}
          Initially observed ice thickness and localization of the piezometers
          as deduced from initial water-pressure measurements.
        \textbf{(c)}
          Three-dimensional rendering of digital sensor units
          \citep[DIBOSS,][]{Ryser.2014, Ryser.etal.2014, Ryser.etal.2014a},
          including the top cylindrical slots with six pins where pressure
          sensors were installed under a silicon membrane.}
      \label{fig:bores}
    \end{figure}

    \begin{figure}
      \centerline{\includegraphics{bowstr_nofil}}
      \caption{%
        \textbf{(a)}
          Complete unfiltered piezometer record, including the initial borehole
          water pressure measurements, peak pressure during borehole closure
          by refreezing, and the transition to solid ice stress records.
        \textbf{(b, c)}
          Insets displaying an example of semi-diurnal periodic oscillations
          observed in the stress record after borehole closure.
        \textbf{(d)}
          Corresponding temperature record, estimated closure dates when
          temperatures reach $80\%$ of their minumum value relative to
          0$\,$C$^\circ$, and long-term ice warming trend due to latent heat
          \citep[cf.][]{Seguinot.etal.2020}.}
      \label{fig:nofil}
    \end{figure}

    \begin{figure}
      \centerline{\includegraphics{bowstr_pgram_stfft}}
      \caption{%
        \textbf{(a--i)}
          Fast-fourier transforms of the time derivative of hourly-resampled
          stress series, computed from the part of the record that follows
          estimated borehole closure (Fig.~\ref{fig:nofil}).
        \textbf{(j)}
          Fast-fourier transform of the time derivative of Pituffik tides,
          converted to sea-water pressure and divided by ten. Insets show a
          zoom on the diurnal and semi-diurnal tidal components which can also
          be found in some of the stress records.}
      \label{fig:pgram}
    \end{figure}

    \begin{figure}
      \centerline{\includegraphics{bowstr_sgram_stfft}}
      \caption{%
        \textbf{(a--h)}
          Rolling-window spectrograms of the post-closure time derivative of
          stress and the time derivative of tidal pressure divided by ten.
          Fast fourier transforms are computed on 14-day windows with a 12-day
          overlap. Colour curves visualize the relative ratio of power spectral
          density between the 10--14 and 22--26\,h bands. Semi-diurnal
          oscillations predominate the record outside the melt season.
          Near-basal units UI03 and UI02 are omitted due to the short length of
          their stress record after borehole closure.}
      \label{fig:sgram}
    \end{figure}

    \begin{figure}
      \centerline{\includegraphics{bowstr_sgram_stcwt}}
      \caption{%
        \textbf{(a--h)}
          Continuous Morlet wavelet transforms of post-closure stress and tidal
          pressure divided by ten, showing a predominant semi-diurnal signal
          outside the melt seasons and the emergence of a superimposed diurnal
          signal during the melt seasons.
          \note{%
            I find that this plot is bringing little new information on top of
            the Fourier spectrograms (\ref{fig:sgram}) and that either figure
            would probably be sufficient in itself. What do you think?}}
      \label{fig:wlets}
    \end{figure}

    \begin{figure}
      \centerline{\includegraphics{bowstr_lines_12hbp}}
      \caption{%
        \textbf{(a)}
          Complete piezometer record, after applying hourly resampling and a
          second-order band-pass Butterworth filter once forward and once
          backward (resulting in a zero-phase fourth-order filter) with cutoff
          frequencies of 2 and 6 per day (or cutoff periods of 4 and 12\,h).
          The bottom curve shows non-filtered tidal pressure from Pituffik,
          converted to sea-water pressure and divided by ten.
        \textbf{(b)}
          Zoom on the later part of the 2014 melt season and transition into
          fall, before signal to UI03 and UI02 was lost, but while LI03
          remained in liquid water. All sensors except LI03 record a transition
          from a complex signal to semi-diurnal oscillations anti-correlated
          with the tide, and roughly an order of magnitude lower in amplitude.}
      \label{fig:lines}
    \end{figure}

    \begin{figure}
      \centerline{\includegraphics{bowstr_ccorr_12hbp}}
      \caption{%
        \textbf{(a)}
          Extract of 10-minute-resampled, 4 to 12\,h band-pass-filtered stress
          series and unfiltered tidal pressure in September and October 2014.
        \textbf{(b)}
          Cross-correlation between each band-pass-filtered stress series
          against tidal pressure. Coloured dots indicate the maximum absolute
          correlation, which corresponds to an anti-correlation with a phase
          delay of ca.~1--2\,h.
        \textbf{(c)}
          Corresponding phase delay (with a precision on 10 minutes) plotted
          against initial sensor depth before closure.}
      \label{fig:ccorr}
    \end{figure}

    \begin{figure}
      \centerline{\includegraphics{bowstr_mcorr_12hbp}}
      \caption{%
        \textbf{(a-g)}
          Moving-window cross-correlation between the each 10-minute-resampled,
          4 to 12\,h band-pass-filtered stress series against tidal pressure.
          Colour curves show the moving-window phase delay of filtered stress
          records relative to tides when anticorrelation is above 0.75.
          \todo{add colorbar and mask periods without tide data.}}
      \label{fig:mcorr}
    \end{figure}

% ----------------------------------------------------------------------
% Tables
%\clearpage
% ----------------------------------------------------------------------


% ======================================================================
\end{document}
% ======================================================================

\documentclass[utf8]{article}

\usepackage{doi}
\usepackage{authblk}
\usepackage[T1]{fontenc}
\usepackage[utf8]{inputenc}
\usepackage[pdftex]{xcolor}
\usepackage[pdftex]{graphicx}
%\usepackage[authoryear,round]{natbib}

% review mode
\usepackage{geometry}
\usepackage{lineno}
\linenumbers
\linespread{1.5}

\graphicspath{{../../figures/}}

\definecolor{c0}{HTML}{1f77b4}
\definecolor{c1}{HTML}{ff7f0e}
\definecolor{c2}{HTML}{2ca02c}
\definecolor{c3}{HTML}{d62728}
\definecolor{c4}{HTML}{9467bd}
\definecolor{c5}{HTML}{8c564b}
\definecolor{c6}{HTML}{e377c2}
\definecolor{c7}{HTML}{7f7f7f}
\definecolor{c8}{HTML}{bcbd22}
\definecolor{c9}{HTML}{17becf}

\newcommand{\idea}[1]{\textcolor{c2}{\emph{[\textbf{IDEA:} #1]}}}
\newcommand{\note}[1]{\textcolor{c0}{\emph{[\textbf{NOTE:} #1]}}}
\newcommand{\todo}[1]{\textcolor{c3}{\emph{[\textbf{TODO:} #1]}}}

\hypersetup{colorlinks, citecolor=c0, linkcolor=c1, urlcolor=c6}

\title{A coincidental measure of englacial stress \\
       in Bowdoin Glacier, Greenland}

\author[1]{Julien Seguinot}
\author[2]{Evgeniy Podoslkiy}
\author[ ]{others}
%\author[1]{Martin Funk}
%\author[3]{Cornelius Senn}
%\author[3]{Shin Sugiyama}

\affil[1]{Independent scholar, Anafi, Greece}
\affil[2]{Arctic Research Center, Hokkaido University, Sapporo, Japan}
%\affil[3]{Department of Civil, Environmental and Geomatic Engineering,
%          ETH Zürich, Switzerland}
%\affil[4]{Institute of Low Temperature Science, Hokkaido University,
%          Sapporo, Japan}


% ======================================================================
\begin{document}
% ======================================================================

\maketitle

\begin{abstract}

    \note{preliminary abstract.}
    %
    While compressive stresses are understood to be critical variable governing
    glacier motion, they have rarely been measured in nature. Instead, the
    empirical understanding of glacier stress typically relies on laboratory
    flow-law experiments and indirect measurements of glacier surface velocity
    and englacial tilt rates. Nevertheless, there is no lack of observational
    evidence that cristal orientation, ice water content and impurities may
    complicate laboratory-derived laws in nature.
    %
    Here, we present an accidental, three-year record of englacial stress
    from water-pressure sensors frozen into a Greenlandic tidewater glacier.
    While the sensors, meant to locate instruments in hotwater-drilled
    boreholes, were not calibrated for solid stress measurements, they
    continued to record pressure changes after the complete refreezing of the
    boreholes and the stabilisation of ice temperatures well below the pressure
    melting point.
    %
    All sensors recorded in-phase pressure changes with 12-hour, 24-hour and
    14-day periodicities, leaving no doubt on the tidal nature of the signal.
    Surprisingly, the amplitude of these pressure variations, recorded one to
    two kilometres upstream the glacier front, is only an order of magnitude
    lower than the tidal amplitude measured at sea. However, glacier stress is
    anticorrelated with the tides, and show a delay of one to two hours, so
    that maximum stress occurs a little after low tide.

\end{abstract}

% ----------------------------------------------------------------------
\section{Introduction}
% ----------------------------------------------------------------------

    Most glaciers lead lives of constant stress, so they relax by slowly
    gliding on their bed.

% ----------------------------------------------------------------------
\section{Methods}
% ----------------------------------------------------------------------

% ----------------------------------------------------------------------
\section{Results}
% ----------------------------------------------------------------------

% -- -- -- -- -- -- -- -- -- -- -- -- -- -- -- -- -- -- -- -- -- -- -- -
\subsection{Stress timeseries}
% -- -- -- -- -- -- -- -- -- -- -- -- -- -- -- -- -- -- -- -- -- -- -- -

% -- -- -- -- -- -- -- -- -- -- -- -- -- -- -- -- -- -- -- -- -- -- -- -
\subsection{Spectral content}
% -- -- -- -- -- -- -- -- -- -- -- -- -- -- -- -- -- -- -- -- -- -- -- -

% -- -- -- -- -- -- -- -- -- -- -- -- -- -- -- -- -- -- -- -- -- -- -- -
\subsection{Phase relationships}
% -- -- -- -- -- -- -- -- -- -- -- -- -- -- -- -- -- -- -- -- -- -- -- -

% ----------------------------------------------------------------------
\section{Discussion}
% ----------------------------------------------------------------------

% -- -- -- -- -- -- -- -- -- -- -- -- -- -- -- -- -- -- -- -- -- -- -- -
% \subsection{}
% -- -- -- -- -- -- -- -- -- -- -- -- -- -- -- -- -- -- -- -- -- -- -- -

% ----------------------------------------------------------------------
\section{Conclusions}
% ----------------------------------------------------------------------


% ----------------------------------------------------------------------
% Acknowledgements
% ----------------------------------------------------------------------

%\paragraph{Acknowledgements}
%\paragraph{Author contributions}
%\paragraph{Conflict of interest}
%\paragraph{Contribution to the field}
%\paragraph{Data availability}


% ----------------------------------------------------------------------
% References
% ----------------------------------------------------------------------

%\bibliographystyle{frontiersinSCNS_ENG_HUMS}
%\bibliography{../../references/references}


% ----------------------------------------------------------------------
% Figures
%\clearpage
% ----------------------------------------------------------------------

% ----------------------------------------------------------------------
% Tables
%\clearpage
% ----------------------------------------------------------------------


% ======================================================================
\end{document}
% ======================================================================

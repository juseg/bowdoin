\documentclass[utf8]{article}

\usepackage{doi}
\usepackage{authblk}
\usepackage[T1]{fontenc}
\usepackage[utf8]{inputenc}
\usepackage[pdftex]{xcolor}
\usepackage[pdftex]{graphicx}
\usepackage[authoryear,round]{natbib}

% review mode
\usepackage{geometry}
\usepackage{lineno}
\linenumbers
\linespread{1.5}

\graphicspath{{../../figures/}}

\definecolor{c0}{HTML}{1f77b4}
\definecolor{c1}{HTML}{ff7f0e}
\definecolor{c2}{HTML}{2ca02c}
\definecolor{c3}{HTML}{d62728}
\definecolor{c4}{HTML}{9467bd}
\definecolor{c5}{HTML}{8c564b}
\definecolor{c6}{HTML}{e377c2}
\definecolor{c7}{HTML}{7f7f7f}
\definecolor{c8}{HTML}{bcbd22}
\definecolor{c9}{HTML}{17becf}

\newcommand{\idea}[1]{\textcolor{c2}{\emph{[\textbf{IDEA:} #1]}}}
\newcommand{\note}[1]{\textcolor{c0}{\emph{[\textbf{NOTE:} #1]}}}
\newcommand{\todo}[1]{\textcolor{c3}{\emph{[\textbf{TODO:} #1]}}}

\hypersetup{colorlinks, citecolor=c0, linkcolor=c1, urlcolor=c6}

\title{
    Tide-modulated englacial stress observed \\
    at Bowdoin Glacier, Greenland}

\author[1]{Julien Seguinot}
\author[2]{Evgeny A. Podolskiy}
\author[ ]{others}
%\author[1]{Martin Funk}
%\author[3]{Cornelius Senn}
%\author[3]{Shin Sugiyama}

\affil[1]{Department of Biological Sciences, University of Bergen and
          Bjerknes Centre for Climate Research, Bergen, Norway}
\affil[2]{Arctic Research Center, Hokkaido University, Sapporo, Japan}
%\affil[3]{Department of Civil, Environmental and Geomatic Engineering,
%          ETH Zürich, Switzerland}
%\affil[4]{Institute of Low Temperature Science, Hokkaido University,
%          Sapporo, Japan}


% ======================================================================
\begin{document}
% ======================================================================

\maketitle

\begin{abstract}

    \note{preliminary abstract.}
    %
    While stresses are understood to be critical variable governing
    glacier motion, they have rarely been measured in situ. Instead, the
    empirical understanding of glacier stress typically relies on laboratory
    flow-law experiments and indirect measurements of glacier surface velocity
    and englacial tilt rates. Nevertheless, there is no lack of observational
    evidence that crystal orientation, ice water content and impurities may
    complicate laboratory-derived laws in nature.
    %
    Here, we present an accidental, three-year record of englacial stress
    from water-pressure sensors frozen into a Greenlandic tidewater glacier
    two-to-one kilometers upstream the calving front.
    While the sensors, meant to locate instruments in hotwater-drilled
    boreholes, were not intended for solid stress measurements, they
    continued to record pressure changes after the complete refreezing of the
    boreholes and the stabilisation of ice temperatures well below the pressure
    melting point.
    %
    All sensors recorded in-phase pressure changes with 12-hour, 24-hour and
    14-day periodicities, revealing the tidal nature of the signal
    \note{overprinted by speed-up events during the melt season, I think.}
    The amplitude of these pressure variations is only an order of magnitude
    lower than the tidal amplitude measured at sea. However, the pressure is
    anticorrelated with the tides, and shows a delay of one to two hours, so
    that maximum stress occurs a little after low tide.
    %
    \note{%
      I'm not sure what the 1--2-hour lag implies. A better look at GPS
      velocities and tilt rates may help.}
\end{abstract}


% ----------------------------------------------------------------------
\section{Introduction}
% ----------------------------------------------------------------------

    Most glaciers lead lives of constant stress, so they relax by slowly
    gliding on their bed.

% ----------------------------------------------------------------------
\section{Methods}
% ----------------------------------------------------------------------

    Borehole set-up (Fig.~\ref{fig:bores}).

% ----------------------------------------------------------------------
\section{Results}
% ----------------------------------------------------------------------

% -- -- -- -- -- -- -- -- -- -- -- -- -- -- -- -- -- -- -- -- -- -- -- -
\subsection{Stress timeseries}
% -- -- -- -- -- -- -- -- -- -- -- -- -- -- -- -- -- -- -- -- -- -- -- -

    Stress timeseries (Fig.~\ref{fig:nofil}).
    Filtered timeseries (Fig.~\ref{fig:lines}).

    Ice-cased sensors recorded stress oscillations.
    The amplitude is an order of magnitude below Pituffik tides.

% -- -- -- -- -- -- -- -- -- -- -- -- -- -- -- -- -- -- -- -- -- -- -- -
\subsection{Spectral content}
% -- -- -- -- -- -- -- -- -- -- -- -- -- -- -- -- -- -- -- -- -- -- -- -

    Fourier transforms (Fig.~\ref{fig:pgram}).
    Fourier spectrograms (Fig.~\ref{fig:sgram}).
    Continuous wavelet transform (Fig.~\ref{fig:wlets}).

    Spectral content is very similar to that of Pituffik tides.
    Tidal signal is overprinted by daily oscillations in summer.
    Untested: especially during speed-up events, I think.

% -- -- -- -- -- -- -- -- -- -- -- -- -- -- -- -- -- -- -- -- -- -- -- -
\subsection{Phase relationships}
% -- -- -- -- -- -- -- -- -- -- -- -- -- -- -- -- -- -- -- -- -- -- -- -

    Cross-correlation (Fig.~\ref{fig:ccorr}).
    Rolling window cross-correlation (Fig.~\ref{fig:mcorr}).

    Stress signal is anti-correlated with the tide.
    Maximum stress occur 1 to 2 hours after low tide.
    This phase delay appears to peak around 200 meter depth.
    The delay is rather stable in 2014-16 but gets messy in 16-17.


% ----------------------------------------------------------------------
\section{Discussion}
% ----------------------------------------------------------------------

% -- -- -- -- -- -- -- -- -- -- -- -- -- -- -- -- -- -- -- -- -- -- -- -
% \subsection{}
% -- -- -- -- -- -- -- -- -- -- -- -- -- -- -- -- -- -- -- -- -- -- -- -

% ----------------------------------------------------------------------
\section{Conclusions}
% ----------------------------------------------------------------------


% ----------------------------------------------------------------------
% Acknowledgements
% ----------------------------------------------------------------------

%\paragraph{Acknowledgements}
%\paragraph{Author contributions}
%\paragraph{Conflict of interest}
%\paragraph{Contribution to the field}
%\paragraph{Data availability}


% ----------------------------------------------------------------------
% References
% ----------------------------------------------------------------------

\bibliographystyle{abbrvnat}
\bibliography{../../../references/references}


% ----------------------------------------------------------------------
% Figures
%\clearpage
% ----------------------------------------------------------------------

    \begin{figure}
      \centerline{\includegraphics{bowstr_bores}}
      \caption{%
        \textbf{(a)}
          Bowdoin borehole locations from drilling in July 2014 to dismantling
          in July 2017 and background satellite image from 2017 March 10,
          17:41:29 UTC. Contains modified Copernicus Sentinel data, processed
          with Sentinelflow.
        \textbf{(b)}
          Initially observed ice thickness and localization of the piezometers
          as deduced from initial water-pressure measurements.
        \todo{add model of tilt unit casing with location of the sensor}.}
      \label{fig:bores}
    \end{figure}

    \begin{figure}
      % FIXME move this after spectral analysis.
      \centerline{\includegraphics{bowstr_nofil}}
      \caption{%
        \textbf{(a)}
          Complete piezometer record, including the initial borehole water
          pressure measurements, and the transition to solid ice stress
          measurements.
        \textbf{(b, c)}
          Insets displaying an example of semi-diurnal periodic oscillations
          observed in the stress record after refreezing.}
      \label{fig:nofil}
    \end{figure}

    \begin{figure}
      % FIXME this has been merged with timeseries
      % \centerline{\includegraphics{bowstr_freezedates}}
      \caption{%
        \textbf{(a)}
          Same as Fig.~\ref{fig:nofil} but with split axes. Sharp peaks in
          the early part of the record correspond to the hotwater-drilled
          borehole refreezing phase.
        \textbf{(b)}
          Corresponding temperature record, showing the initial refreezing
          phase and long-term warming \citep[cf.][]{Seguinot.etal.2020}.
          Full circles denote refreezing dates estimated
          Insets displaying an example of semi-diurnal periodic oscillations
          observed in the stress record after refreezing.
        \note{%
          Maybe this can be merged with Fig.~\ref{fig:nofil}.}}
      \label{fig:freezedates}
    \end{figure}

    \begin{figure}
      \centerline{\includegraphics{bowstr_lines_12hbp}}
      \caption{%
        \textbf{(a)}
          Complete piezometer record, after applying a fourth-order Butterworth
          high-pass filter with a cut-off period of one day. The bottom curve
          shows (non-filtered) tidal measurements from Pituffik, converted to
          sea-water pressure and divided by ten.
        \textbf{(b)}
          Zoom on the later part of the 2014 melt season and transition into
          fall, before contact to UI03 and UI02 was lost, but also while LI03
          was not yet refrozen. Other sensors record a transition from
          high-amplitude, daily stress oscillations (correlated with glacier
          speed) to semi-diurnal oscillations anti-correlated with, and an
          order of magnitude lower than, the tide.
        \note{
          maybe add a separate figure with borehole temp and stress during
          refreezing.}
        \todo{
          fix year label, add GPS and tilts, and highlight the freezing dates?}
        \todo{
          Move after the spectrogram as Fig. 5?}}
      \label{fig:lines}
    \end{figure}

    \begin{figure}
      \centerline{\includegraphics{bowstr_pgram_stfft}}
      \caption{%
        \textbf{(a--i)}
          Fast-fourier transforms of the (non-filtered) series of stress
          derivative over time (kPa$\,$s$^{-1}$) after refreezing. The
          refreezing date is computed independently for each unit as the date
          when temperatures reach $75\%$ of their minumum value relative to
          0$\,$C$^\circ$. \todo{ensure this is consistent among plots.}
        \textbf{(j)}
          Fast-fourier transform of the time derivative of Pituffik tides,
          converted to sea-water pressure and divided by ten. Insets show a
          zoom on the diurnal and semi-diurnal tidal components which can also
          be found in some of the stress records.}
      \label{fig:pgram}
    \end{figure}

    \begin{figure}
      \centerline{\includegraphics{bowstr_sgram_stfft}}
      \caption{%
        \textbf{(a--h)}
          Rolling-window spectrograms of glacier stress and tidal pressure
          temporal derivatives after refreezing of the individual units.
          Fourier transforms are computed on 14-day windows with a 12-day
          overlap. Some of the vertical bands result from interpolating the
          signal from solar-power-adaptive to a constant, ten-minute time step.
          Colour curves indicate the relative ratio of spectral power between
          the 10--14 and 22--26\,h bands, rescaled between 12 and 24\,h for
          visualization, showing that semi-diurnal oscillations dominate the
          record outside the melt season. UI03 and UI02 are omitted due to the
          short length of their solid stress record after refreezing.}
      \label{fig:sgram}
    \end{figure}

    \begin{figure}
      % FIXME not used in talks, is this fig necessary?
      \centerline{\includegraphics{bowstr_sgram_stcwt}}
      \caption{%
        \textbf{(a--h)}
          Continuous wavelet transforms of glacier stress and tidal pressure
          temporal derivatives after applying a fourth-order Butterworth
          high-pass filter with a cut-off period of one day \todo{maybe
          filtering is not needed here}, showing the emergence of a diurnal
          signal during the 2015 melt season, in addition to the persistent
          semi-diurnal signal.
          \todo{%
            try and extend upwards to 14-days periods.}
          \note{%
            I do not yet master this. I tried to plot the CWT for longer
            record, but the magnitude of 12 and 24-h oscillations appears to
            diminish as I extend the time period. On the other hand the CWT
            seems to be good at picking the changes in frequency at the
            beginning and the end of the melt season.}}
      \label{fig:wlets}
    \end{figure}

    \begin{figure}
      \centerline{\includegraphics{bowstr_ccorr_12hbp}}
      \caption{%
        \textbf{(a)}
          Extract of the piezometer record and tidal pressure for October 2014,
          after applying a fourth-order Butterworth band-pass filter with
          cut-off periods of 0.5 and 12\,h.
          \todo{try 15\,h, or maybe even use the same filter in all plots.}
        \textbf{(b)}
          Cross-correlation of the ice stress (and water pressure) time series
          against tidal pressure. Coloured dots indicate the maximum absolute
          correlation, which corresponds to an anti-correlation with a phase
          delay of ca.~1--2\,h.
        \textbf{(c)}
          Corresponding phase delay as a function of sensor depth.
        \todo{
          fix labels, check that the clocks are on-time (at least UI* should be
          OK, but I'm not sure about LI*).}}
      \label{fig:ccorr}
    \end{figure}

    \begin{figure}
      \centerline{\includegraphics{bowstr_mcorr_12hbp}}
      \caption{%
        \textbf{(a-g)}
          Rolling-window cross-correlation between the piezometer and tidal
          pressure record when the latter is available, after applying a
          fourth-order Butterworth band-pass filter with cut-off periods of 2
          and 12\,h. Colour curves show the optimal phase delay of stress
          records relative to tides when absolute maximum (anti)correlation is
          above 0.75.
          \note{no idea what happens after 2016.}
          \todo{add colorbar (grey=anticorrelation), try 1-bit normalization}.}
      \label{fig:mcorr}
    \end{figure}

% ----------------------------------------------------------------------
% Tables
%\clearpage
% ----------------------------------------------------------------------


% ======================================================================
\end{document}
% ======================================================================

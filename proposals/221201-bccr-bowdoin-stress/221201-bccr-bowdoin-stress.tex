% Postdoc development plan due 2022.07.01.

% # Fast Track Initiative
%
% Do you have a good idea that might lead to a new project?
%
% The Fast-track initiatives (FTI) support focused, short-term research
% activities with the aim of achieving concise result(s).
%
% The call goes out every autumn, and deadline for submission is usually
% December 1. Calls for FTI will be announced by email to BCCR members in due
% time before the deadline.
%
% The Fast Track Initiative is set up to support pilot studies, proposals,
% synergy works and other intitiatives. We have several large projects today
% that started out as a fast track initiative.
%
% Limits to application: Maximum 500 kNOK
%
% Who can apply: Anyone affiliated with the Bjerknes Centre for Climate
% Research, resident in Bergen
%
% ## Criteria
%
% - Must target at least one of Bjerknes strategic research areas
% - Promote the integration of the Bjerknes consortium
% - Must have clearly identifiable outcome
% - Must be short-lived and completed within 1 year of the grant´s commencement
% - Proposals should not exceed one A4 page including a description of the
%   activity, objective(s), budget, timeline and outcome(s). If salaries are
%   applied for, the person shall be identified and an estimate of the salary
%   costs must be given by an economy consultant at one of the partner
%   institutions and according to the agreement on cost per hour for the SKD
%   period 2021-2026.
%
% ## Some examples of eligible activities
%
% Note that this list is not exhaustive:
%
% - Blue-sky research: identify or conduct exploratory activities including
%   laboratory work, for new, cross-cutting research relevant to Bjerknes
% - Preparation of synthesis/integration research articles or large research
%   proposals
% - Up to 3 month´s salary for a recently graduated master student to complete
%   a manuscript for publication at salary rate 42
%
% Non-eligible activities or expenses:
%
% - Running costs such as travels, conference fees, publication costs, etc.
% - Salaries of non-Bjerknes personnel
% - Salaries for PhD students before thesis is submitted
% - Research visits to Bergen. These must be applied to the Bjerknes Visiting
%   Fellow programme

% vanilla article for latex2rtf
\newif\iflatextortf
\iflatextortf
  \documentclass{article}
  \renewcommand{\cvdoubleitem}[4]{#1: #2; #3: #4}
  \renewcommand{\href}[2]{\htmladdnormallink{\color{blue}\underline{#2}}{#1}}

% else moderncv just looks cool
\else
  \documentclass[11pt,a4paper,sans,colorlinks]{moderncv}
  \name{Julien}{Seguinot}  % only used in metadata
  \moderncvstyle[left]{casual}  % casual (default), classic, oldstyle or banking
  \moderncvcolor{blue} % blue (default), orange, green, red, purple, grey or black

\fi

% latex packages
\usepackage[T1]{fontenc}
\usepackage[utf8]{inputenc}
\usepackage[bottom=0.08\paperheight,top=0.08\paperheight]{geometry}

% moderncv section numbers
% \newcounter{secnumber}
% \renewcommand\sectionstyle[1]{{
%   \refstepcounter{secnumber}\sectionfont\textcolor{color1}{\thesecnumber.~#1}}}

% custom commands
\newcommand{\todo}[1]{{\color{red}\emph{Todo}: #1}}

% document properties
\title{Development plan for Postdoctoral Research Fellow}

% ======================================================================
\begin{document}
\setlength{\parskip}{0.5\baselineskip}
% ======================================================================

    \begin{center}
      \textbf{Development plan for Postdoctoral Research Fellow}\\
      Julien~Seguinot \\
      Department of Biological Sciences and
      Bjerknes Centre for Climate Research
    \end{center}

    Most glaciers lead lives of constant stress, so they relax by slowly
    gliding on their beds. In glacier mechanics, stress is typically
    unobservable. Much as civil engineers inspect buildings for cracks,
    glaciologist measure the stress relaxation: the gliding, creeping and
    crevassing of glaciers. While direct stress measurements in glacier may
    sound like a theoretician's dream, they could be a drastically accelerate
    our understanding of glacier hazards and ice-ocean interactions. I have
    results hinting at their feasibility.

    I hereby apply for funds to analyse and publish these data: an accidental,
    three-year record of englacial stress from water-pressure sensors frozen
    into a Greenlandic tidewater glacier.

% ----------------------------------------------------------------------
\section{Existing results}
% ----------------------------------------------------------------------

    Between summers 2014 and 2017, colleagues and I deployed and maintained
    ice-borehole instruments on Bowdoin Glacier, a tidewater glacier in
    north-western Greenland. These instruments included digital multi-sensor
    units, each equipped with a piezometer to measure overhead water depth and
    thus its location in the borehole. While these water-pressure sensors were
    not intended for solid stress measurements, they continued to record
    pressure changes well after the complete refreezing of the borehole.
    Unexpectedly, that record is not just noise.

    In 2021 (then unemployed), together with my field colleague and glacier
    seismologist Evgeniy Podolskiy, I analysed the stress timeseries via
    Fourier transform, wavelet transforms, and rolling window spectrograms and
    cross-correlation.
    All sensors recorded in-phase pressure changes with 12-hour, 24-hour and
    14-day periodicities, corresponding to the local tidal signature.
    The amplitude of these pressure variations is only an order of magnitude
    lower than the tidal amplitude measured at sea. However, the pressure is
    anti-correlated with the tides, and shows a delay of one to two hours, so
    that maximum stress occurs a little after low tide.

% ----------------------------------------------------------------------
\section{Proposed work}
% ----------------------------------------------------------------------

    While we have figures ready for a paper, we lack solid
    interpretation for the phase relationship between oceanic tides and stress
    oscillations. We aim to seek an explanation in Bowdoin Glacier's surface
    velocity record: GPS timeseries collected by Shin Sugiyama and colleagues
    for several parts of the glacier over an overlapping observation period. At
    least some of these records have a subtle tidal signature.

    I hereby apply for funds to travel to Hokkaido University to analyze the
    spectral content of Bowdoin GPS signals, interpret the englacial stress
    timeseries, and compile the results in a publication.

% ----------------------------------------------------------------------
\section{Budget}
% ----------------------------------------------------------------------

    \begin{itemize}
        \item Three months of my own salary (135000 NOK?)
        \item Return travels to Sapporo (20000 NOK)
        \item Funding for an open-access publication (20000 NOK)
    \end{itemize}

% ----------------------------------------------------------------------
\section{Related}
% ----------------------------------------------------------------------

    \begin{itemize}
      \item J. Seguinot et al, \emph{Front. Earth Sci.}, 8, 2020, doi:\href
          {https://doi.org/10.3389/feart.2020.00065}{10.3389/feart.2020.00065}.
      \item
        Could we measure stress in glaciers? \href
          {https://juseg.github.io/could-we-measure-stress-in-glaciers/}
          {Blog post}.
    \end{itemize}

% ======================================================================
\end{document}
% ======================================================================

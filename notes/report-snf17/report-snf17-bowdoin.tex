\documentclass{article}

% Packages
\usepackage{lineno}
\usepackage{geometry}
\usepackage{setspace}
\usepackage[T1]{fontenc}
\usepackage[utf8]{inputenc}
\usepackage[authoryear,round]{natbib}

% Options
\onehalfspacing
\linenumbers

% Document properties
\title{Dynamic changes of tidewater outlet glaciers:\\
       Bowdoin glacier, Northwest Greenland\\\bigskip
       \large Final report (1 May 2014 -- 28 Feb. 2017)}
\author{}
\date{}

% Common units
\newcommand{\e}[1]{\ensuremath{\times 10^{#1}}}
\newcommand{\chem}[1]{\ensuremath{\mathrm{#1}}}
\newcommand{\unit}[1]{\ensuremath{\mathrm{#1}}}
\newcommand{\degree}[0]{\ensuremath{^{\circ}}}
\newcommand{\degC}[0]{\unit{{\degree}C}}

% Bold emphasis
%\DeclareTextFontCommand{\emph}{\bfseries}

% ----------------------------------------------------------------------
% Instructions
% ----------------------------------------------------------------------

% The scientific report should consist of the following:
%
% 1. Summary (2-4 pages) of the research work and its results
%
%     Please describe the research work conducted during the entire funding
%     period in relation to the objectives, milestones and hypotheses mentioned
%     in the research plan.
%
%     Please present your main research results and explain their relevance in
%     relation to the published, submitted or planned research output (e.g.
%     publications, events, patents etc.)
%
%     Please point out any major deviations from the research plan. Such
%     deviations should be justified and will be examined by the SNSF.
%
%     Please describe the contributions made by the project staff.
%
%     Please mention any important events (e.g. change of personnel, delayed
%     start of project etc.).
%
%     For projects with co-applicants based abroad (e.g. in the framework of
%     the lead agency agreement):
%
%     Please describe the major contributions made by your partners abroad.
%
% In final reports, the condition and the further use of scientific equipment
% acquired under an SNSF grant (items costing more than CHF 20,000) must be
% mentioned (Art. 45 of the Funding Regulations).

% ======================================================================
\begin{document}
% ======================================================================

\maketitle

% ----------------------------------------------------------------------
\section{Work conducted}
% ----------------------------------------------------------------------

This project (SNF grant No. 200021-153179/1 to M.~Funk) aimed at better
understanding the processes of interaction between ice melt, ice motion and ice
discharge to the ocean (calving) involved in the current rapid retreat of
Greenland marine-terminating (so-called tidewater) glaciers. Its principle
endeavour consisted in collecting for the first time observational data on and
near the ice calving front of a tidewater glacier, Bowdoin Glacier, chosen for
its relative ease of access and little apparent crevassing. Given the costs and
logistics involved in planning fieldwork in remote Northwest Greenland, the
project was planned as a collaboration between ETH Zürich (M.~Funk,
J.~Seguinot, G.~Jouvet) and Hokkaido University (S.~Sugiyama, S.~Tatsuki,
D.~Sakakibara).

Three field campaigns, i.e. one more than originally planned for, were
conducted. Due to weather conditions and technical issues with electronics in
the cold and dark polar winter, not all experiments were successful. However, a
large set of unprecedented data were collected, whose analysis is still ongoing
under follow-up projects (SNF grant No.~SNF 200020-169558 to M.~Funk, ETH grant
No.~ETH-12 16-2 to G.~Jouvet, ETH-UZH Ph.D. studentship to A.~Walter).

During the \emph{summer 2014 field campaign} (15 -- 29 July 2014), two
boreholes were drilled with hot water and equipped with thermistor strings
throughout the ice column, pressure sensors at the glacier bed, and
tilt sensors at different depths. Two automated cameras were installed on
elevated hills and set-up to record through the polar winter.

During the \emph{summer 2015 field campaign} (6 -- 20 July 2015), the first
borehole data were collected. Automated camera and borehole installation
batteries were either recharged or replaced. In addition, experimental flights
were conducted with an Unmanned Aerial Vehicle (UAV) allowing detailed imagery
of the calving front. Seismometers were installed and maintained for two weeks
on the glacier surface next to the calving front. Following success of various
experiments, it was decided to leave working installations on site and to come
back in 2016.

During the \emph{summer 2016 field campaign} (4 -- 21 July 2016), more borehole
data were collected. Automated cameras were upgraded with solar panels and
second external batteries in order to collect more pictures in winter.
Systematic UAV flights were performed more than twice per day and marked
control points regularly surveyed on and near the glacier. Seismometers were
again installed and maintained for two weeks on the glacier surface next to the
calving front, and additional seismometers were installed in the ice and
equipped with external batteries and solar panels for one year. A second team
set up camp on top of a mountain facing the glacier and ran an interferometric
RADAR for two weeks in order to monitor movements of the calving front.

As part of the newly granted projects, researchers from ETH Zürich, University
of Zürich and Hokkaido University will visit Bowdoin Glacier again in 2017 in
order to perform a new interferometric RADAR survey including precisely
positioned reflectors on the ground, new long-range UAV flights collecting data
from several glaciers in the area, and to retrieve the englacial seismometers
and remaining borehole installations.


% ----------------------------------------------------------------------
\section{Main results}
% ----------------------------------------------------------------------

\subsection{Ice temperature}

Borehole temperature measurements were \emph{very successful} and yielded a
continuous, one-year (2014--2015) record of temperature profile evolution.
Although we expected to observe no intrinsic temperature variations at depth,
this temporally-resolved record was needed to ensure the complete dissipation
of the thermal anomaly due to the hot water drilling itself. Our record shows
that this initial phase took three months, after which an equilibrium profile
was attained. Ice temperature was recorded to be below the pressure melting
point over the entire column, except for the melting ice base. Although this
result could be expected from ice thermodynamic theories, ice temperature
measurements in tidewater glaciers are rare and our measured values are crucial
to interpret the other borehole observations.

More surprisingly, the temperature profiles we measured in the two boreholes,
distant only by 250\,m, although similar in shape, exhibit temperature
differences of up to 2\degC. Although such localized temperature differences
have already been observed in Greenland \citep{Luthi.etal.20xx}, this is to our
knowledge the first time they are found to extend over the entire glacier
depth. In addition, one of the temperature profiles exhibit a quick warming
trend which can not be explained by temperature advection and diffusion
processed known to occur in glaciers. This surprising observation lead us to
extend the observation for this profiles by a second year (2015--2016). We
interpret the spatial temperature difference and the warming trend as the
expression of latent heating due to meltwater refreezing in deep crevasses
\citep{Seguinot.etal.inprepa}. A punctual temperature measurement confirming
the warming trend is planned for the summer 2017.


\subsection{Basal water pressure}

Borehole subglacial water pressure measurements were only \emph{partly
successful} It has previously been observed when drilling to glacier beds, that
some borehole seem to hit an efficient subglacial drainage network (hot water
used for drilling evacuate the borehole in seconds after it reaches the bed)
while others remain disconnected from that system (hot water remains in the
borehole after it reaches the bed). During the borehole drilling on Bowdoin
Glacier in summer 2014, one borehole connected to the subglacial system, and
one remained hydrologically isolated.

The two boreholes were set-up for observation for two years (2014--2016), yet
unfortunately the connected borehole batteries drained twice very quickly,
perhaps due to a defect in the data logger, thus leaving data only covering two
short summer periods. The lower borehole water pressure was recorded
continuously for two years (2014--2016). An increase in water pressure during
summer 2016 shows that the sensor, moving along with the glacier flow, probably
eventually reached an area connected to the subglacial hydrological system,
which would allow for comparison on the ice deformation record. We hope to
collect a final year of data (2016--2017) from this sensor in summer 2017.


\subsection{Ice deformation}

Borehole ice deformation measurements, unplanned for in the original project
proposal, were conducted. The deepermost tilt sensors were lost during
installation, resulting in lower
precision in the extrapolated ice deformation velocity profile. Nevertheless, a
continuous deformation record was obtained for one year (Nov.~2014--Nov.~2015)
from one borehole (until the cable broke in a crevasse), and for almost two
years (Nov.~2014--July~2016) from the other. As far as we know this is both the
first ice deformation record from the terminal part of a tidewater glacier, as
well as one of the longest ice deformation time-series recorded.

The deformation velocity extrapolated from tilt measurements account for about
as little as 10\% of the observed glacier surface velocities, showing that
Bowdoin Glacier flow is dominated by sliding. Although this result could be
expected from ice flow dynamics theory, this is to our knowledge the first time
that ice internal deformation and sliding are quantified on a tidewater
glacier. Moreover, our continuous deformation record indicate distinct speed-up
events associated with periods of warm weather during summer, showing that
fluctuations of ice velocity related to subglacial meltwater intrusions are
reverberated onto internal deformation \citep{Seguinot.etal.inprepb}. We hope
to confirm this finding with a final year of data (2016--2017) to be collected
from the remaining tilt sensor chain in summer 2017.


\subsection{Unmanned aerial vehicle}

UAV photogrammetry, unplanned for in the original project proposal, allowed for
the collection of detailed orthoimages and digital elevation models of the
glacier surface in the vicinity of the calving front.

Experimental flights performed during the summer 2015 field campaign allowed to
reconstruct a high-resolution displacement field corresponding to the
initiation of a large fracture prior to calving. The results were compared with
numerical modelling, revealing that this deep, water-filled fracture originated
from a subglacial mound that likely contributes to the stability the Bowdoin
Glacier calving front \citep{Jouvet.etal.2016}.

Systematic UAV flights performed during the summer 2016 field campaign resulted
in 22 high-resolution orthoimages and 19 velocity fields of the calving front
for 12 days. The results revealed a correlation between fluctuations of surface
velocity, air temperature, and the area of meltwater plumes at the glacier
snout, showing that accelerations are triggered by an increase of buoyant
forces in response to a surplus of subglacial meltwater
\citep{Jouvet.etal.inprep}.


\subsection{Seismometers}

Seismic records, unplanned for in the original project proposal, were collected
by the array of four surface seismometers (three from ETH Zürich and one from
Hokkaido University) installed 250\,m from the calving front of Bowdoin Glacier
during the field campaigns of summers 2015 and 2016. The records, currently
being analysed at Hokkaido University (E.~Podolskiy) reveal an intense seismic
activity correlated to lowering tide in the fjord and longitudinal stretching
measured on the glacier surface \citep{Podoslkiy.etal.2016}.

Preliminary data from the array of four subsurface seismometers (all from ETH
Zürich) installed for a year-long experiment next to the borehole site located
2\,km upstream the calving front were retrieved at the end of the summer 2016
field campaign. They seem to indicate some correlation of seismic activity to
air temperature, potentially indicating of a different seismic source than that
recorded next to the calving front (E.~Podolskiy, personal communication).
These instruments and their records will be collected during the summer 2017
field campaign.


\subsection{Interferometric RADAR}

The interferometric RADAR, also unplanned for in the original project proposal,
scanned the glacier surface and the calving front of Bowdoin Glacier
uninterruptedly for two weeks. The RADAR data, collected with a frequency of
2\,minutes, are currently being analysed by PhD candidate jointly funded by
ETH Zürich and University of Zürich (A.~Walter). They will allow to detect
distributed glacier motion at very high spatial and temporal resolution,
resolving small fluctuations in flow velocity and individual fracture events.

However, atmospheric perturbations of radio wave travel times were found to be
a much bigger problem than anticipated. Therefore, new RADAR data will be
collected during the summer 2017 field campaign, using a higher frequency and
artificial ground reflectors in order to correct for these perturbations.


% ----------------------------------------------------------------------
\section{Major deviations}
% ----------------------------------------------------------------------

The treatment of calving with a numerical ice flow model including damage
mechanics, planned for in the original project proposal, has not been
performed. However, numerical modelling was used to analyse the high-resolution
velocity fields obtained from UAV photogrammetry \citep{Jouvet.etal.2016}.

Moreover, as we gradually became aware of the fantastic opportunities offered
by the Bowdoin Glacier field site, we expanded our monitoring to several
measurements unplanned for in the original project proposal, namely the
borehole ice deformation measurements, the UAV photogrametric surveys, the
surface and subsurface seismic records, and the interferometric RADAR surveys.
This resulted in the collection of a huge data set, some of which still remains
to be analysed in follow-up projects.


% ----------------------------------------------------------------------
\section{Contributions by the project staff}
% ----------------------------------------------------------------------

M.~Funk organized the three summer field campaigns (2014, 2015, 2016) on
Bowdoin Glacier. He supervised the borehole drilling in 2014 and later formed
J.~Seguinot for data collection and processing. Together with partners abroad
he installed and maintained surface seismometers in 2015, and the
interferometric RADAR in 2016.

J.~Seguinot collected and analysed borehole data from 2015 and 2016, and
assisted partners abroad to install and maintain surface and subsurface
seismometers in 2016. He is currently preparing two publications based on
analysed borehole data.

G.~Jouvet and Y.~Weidmann assembled UAVs, conducted UAV flights in 2015 and
2016 and analysed the resulting data. G.~Jouvet conducted numerical modelling,
published results from 2015 and is preparing a second publication.

\emph{More? Baudi, Tinu? Who else is "project staff"?}


% ----------------------------------------------------------------------
\section{Contributions by partners abroad}
% ----------------------------------------------------------------------

Colleagues from Hokkaido University conducted a first, reconnaissance field
work on Bowdoin Glacier (2013) before this project started. During joint field
campaigns (2014, 2015, 2016), they contributed to all tasks that demanded a
coordination between our two teams, such as the application for expedition
permits, the helicopter transportation to Bowdoin Glacier, the hot water
drilling of the two boreholes in 2014 and the installation and retrieval of
seismometers and borehole data loggers in 2015 and 2016.

Besides, colleagues from Hokkaido University performed numerous additional
studies on Bowdoin Glacier and surroundings, such as GPS and ice penetrating
RADAR surveys \citep{Sugiyama.etal.2015, Tatsuki.etal.2016}, ultrasonic echo
sounding of the fjord topography \citep{Sugiyama.etal.2015}, analysis of
satellite imagery \citep{Sugiyama.etal.2015, Sakakibara.etal.2016,
Tatsuki.etal.2016}, weather observations \citep{Sugiyama.etal.2015}, record of
tidal and tsunami sea level variations in the fjord, analysis of seismic
records \citep{Podoslkiy.etal.2016}, modelling of the glacier elastic response
to tides, and supra-glacial and oceanic water sampling.

\emph{Should I include Riccardo though not official partner?}


% ----------------------------------------------------------------------
\section{Important events}
% ----------------------------------------------------------------------

Colleagues from Hokkaido University also organized an international workshop
on the Greenland ice sheet in Sapporo, 22-24 March 2016, in which several of
our staff participated. Following our July 2016 campaign and as part of the
larger Japanese project on Arctic Research ArCS, our colleagues have also
organized a popular science workshop with the inhabitants of Qaanaaq, the
Greenlandic village near Bowdoin Glacier that makes fieldwork there possible.


% ======================================================================
\end{document}
% ======================================================================

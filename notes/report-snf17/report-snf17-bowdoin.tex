\documentclass{article}

\usepackage{lineno}
\usepackage{geometry}
\usepackage{setspace}
\usepackage[T1]{fontenc}
\usepackage[utf8]{inputenc}

\onehalfspacing
\linenumbers

\title{Dynamic changes of tidewater outlet glaciers:\\
       Bowdoin glacier, Northwest Greenland\\\bigskip
       \large Final report (1 May 2014 -- 28 Feb. 2017)}
\author{}
\date{}

% ----------------------------------------------------------------------
% Instructions
% ----------------------------------------------------------------------

% The scientific report should consist of the following:
%
% 1. Summary (2-4 pages) of the research work and its results
%
%     Please describe the research work conducted during the entire funding
%     period in relation to the objectives, milestones and hypotheses mentioned
%     in the research plan.
%
%     Please present your main research results and explain their relevance in
%     relation to the published, submitted or planned research output (e.g.
%     publications, events, patents etc.)
%
%     Please point out any major deviations from the research plan. Such
%     deviations should be justified and will be examined by the SNSF.
%
%     Please describe the contributions made by the project staff.
%
%     Please mention any important events (e.g. change of personnel, delayed
%     start of project etc.).
%
%     For projects with co-applicants based abroad (e.g. in the framework of
%     the lead agency agreement):
%
%     Please describe the major contributions made by your partners abroad.
%
% In final reports, the condition and the further use of scientific equipment
% acquired under an SNSF grant (items costing more than CHF 20,000) must be
% mentioned (Art. 45 of the Funding Regulations).

% ======================================================================
\begin{document}
% ======================================================================

\maketitle

% ----------------------------------------------------------------------
\section{Work conducted}
% ----------------------------------------------------------------------

This project (SNF grant No. 200021-153179/1 to M.~Funk) aimed at better
understanding the processes of interaction between ice melt, ice motion and ice
discharge to the ocean (calving) involved in the current rapid retreat of
Greenland marine-terminating (so-called tidewater) glaciers. Its principle
endeavour consisted in collecting for the first time observational data on and
near the ice calving front of a tidewater glacier, Bowdoin Glacier, chosen for
its relative ease of access and little apparent crevassing. Given the costs and
logistics involved in planning fieldwork in remote Northwest Greenland, the
project was planned as a collaboration between ETH Zürich (M.~Funk,
J.~Seguinot, G.~Jouvet) and Hokkaido University (S.~Sugiyama, S.~Tatsuki,
D.~Sakakibara).

Three field campaings, i.e. one more than origninally planned for, were
conducted. Due to weather conditions and technical issues with electronics in
the cold and dark polar winter, not all experiments were successful. However, a
large set of unprecedented data were collected, whose analysis is still ongoing
under new projects (SNF grant No.~SNF 200020-169558 to M.~Funk, ETH grant
No.~ETH-12 16-2 to G.~Jouvet, UZH grant No.~??? to ???).

During the \emph{summer 2014 field campaign} (15 -- 29 July 2014), two
boreholes were drilled and equipped with thermistor strings throughout the ice
column, pressure sensors at the glacier bed, and inclinometers at different
depths. Two automated cameras were installed on elevated hills and set-up to
record through the polar winter.

During the \emph{summer 2015 field campaign} (6 -- 20 July 2015), the first
borehole data were collected. Automated camera and borehole installation
batteries were either recharged or replaced. In addition, experimental flights
were conducted with an Unmanned Aerial Vehicle (UAV) allowing detailed imagery
of the calving front. Seismometers were installed and maintained for two weeks
on the glacier surface next to the calving front. Following success of various
experiments, it was decided to leave working installations on site and to come
back in 2016.

During the \emph{summer 2016 field campaign} (4 -- 21 July 2016), more borehole
data were collected. Automated cameras were upgraded with solar panels and
second external batteries in order to collect more pictures in winter.
Systematic UAV flights were performed more than twice per day and marked
control points regularly surveyed on and near the glacier. Seismometers were
again installed and maintained for two weeks on the glacier surface next to the
calving front, and additional seismometers were installed in the ice and
equipped with external batteries and solar panels for one year. A second team
set up camp on top of a mountain facing the glacier and ran an interferometric
RADAR for two weeks in order to monitor movements of the calving front.

As part of the newly granted projects, researchers from ETH Zürich, University
of Zürich and Hokkaido University will visit Bowdoin Glacier again in 2017 in
order to perform a new interferometric RADAR survey including precisely
positioned reflectors on the ground, new long-range UAV flights collecting data
from several glaciers in the area, and to retrieve the englacial seismometers
and remaining borehole installations.


% ----------------------------------------------------------------------
\section{Main results}
% ----------------------------------------------------------------------

\begin{itemize}
\item Borehole measurements (two publications planned, project extended).
\item Drone data (one publication submitted, new project started).
\item Seismometers (one publication, analysis ongoing).
\item Interferometric radar (analysis ongoing).
\end{itemize}

% ----------------------------------------------------------------------
\section{Major deviations}
% ----------------------------------------------------------------------

\begin{itemize}
\item No PhD-student, postdoc instead.
\item No damage modelling.
\item Several unplanned experiments: inclinometers, seismics, drone, radar.
\end{itemize}

% ----------------------------------------------------------------------
\section{Contributions by the project staff}
% ----------------------------------------------------------------------

\begin{itemize}
\item M.~Funk contributions.
\item J.~Seguinot contributions.
\item G.~Jouvet contributions.
\end{itemize}

% ----------------------------------------------------------------------
\section{Contributions by partners abroad}
% ----------------------------------------------------------------------

\begin{itemize}
\item List papers published till now (Shin, Shun, Daiki).
\item Unpublished: infrasound, water sampling, radar, modelling.
\end{itemize}

% ----------------------------------------------------------------------
\section{Important events}
% ----------------------------------------------------------------------

\emph{I am not sure what to include here...
      I will probably remove this section.}

% ======================================================================
\end{document}
% ======================================================================

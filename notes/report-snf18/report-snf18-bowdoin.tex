\documentclass{article}

% Packages
\usepackage{geometry}
\usepackage{graphicx}
\usepackage[T1]{fontenc}
\usepackage[utf8]{inputenc}
\usepackage[authoryear,round]{natbib}
\graphicspath{{../../figures/}}

% Spacing
%\renewcommand\familydefault{\sfdefault}
\setlength{\parskip}{2ex}
\setlength{\parindent}{0em}

% Review mode
%\usepackage{lineno}
%\linenumbers
%\linespread{1.5}

% Document properties
\title{\vspace{-2ex}
       Dynamic changes of tidewater outlet glaciers:\\
       Bowdoin glacier, Northwest Greenland\\\bigskip
       \large Final report (1 Mar. 2017 -- 28 Feb. 2018)}
\author{}
\date{}

% Common units
\newcommand{\e}[1]{\ensuremath{\times 10^{#1}}}
\newcommand{\chem}[1]{\ensuremath{\mathrm{#1}}}
\newcommand{\unit}[1]{\ensuremath{\mathrm{#1}}}
\newcommand{\degree}[0]{\ensuremath{^{\circ}}}
\newcommand{\degC}[0]{\unit{{\degree}C}}

% Bold emphasis
%\DeclareTextFontCommand{\emph}{\bfseries}

% ----------------------------------------------------------------------
% Instructions
% ----------------------------------------------------------------------

% The scientific report should consist of the following:
%
% 1. Summary (2-4 pages) of the research work and its results
%
%     Please describe the research work conducted during the entire funding
%     period in relation to the objectives, milestones and hypotheses mentioned
%     in the research plan.
%
%     Please present your main research results and explain their relevance in
%     relation to the published, submitted or planned research output (e.g.
%     publications, events, patents etc.)
%
%     Please point out any major deviations from the research plan. Such
%     deviations should be justified and will be examined by the SNSF.
%
%     Please describe the contributions made by the project staff.
%
%     Please mention any important events (e.g. change of personnel, delayed
%     start of project etc.).
%
%     For projects with co-applicants based abroad (e.g. in the framework of
%     the lead agency agreement):
%
%     Please describe the major contributions made by your partners abroad.
%
% In final reports, the condition and the further use of scientific equipment
% acquired under an SNSF grant (items costing more than CHF 20,000) must be
% mentioned (Art. 45 of the Funding Regulations).

% ======================================================================
\begin{document}
% ======================================================================

\maketitle

% ----------------------------------------------------------------------
\section{Work conducted}
% ----------------------------------------------------------------------

    This project (SNF grant No.~200020-169558 to M.~Funk) is a one-year
    extension of an earlier grant (SNF No.~200021-153179/1) that aimed at
    better understanding the processes of interaction between ice melt, ice
    motion and ice discharge to the ocean (calving) involved in the current
    rapid retreat of Greenland marine-terminating (so-called tidewater)
    glaciers. Its principle endeavour consisted in collecting for the first
    time observational data on and near the ice calving front of a tidewater
    glacier, Bowdoin Glacier, chosen for its relative ease of access and little
    apparent crevassing. Given the costs and logistics involved in conducing
    fieldwork in remote Northwest Greenland, the project was planned as a
    collaboration between ETH Zürich (M.~Funk, J.~Seguinot, G.~Jouvet,
    F.~Walter) and Hokkaido University (S.~Sugiyama, S.~Tsutaki, D.~Sakakibara,
    E.~Podolskiy).

    During the extension period, a fourth and last field campaign was
    conducted, allowing to gather new data on Bowdoin Glacier, and resulting in
    a three-year continuous data record for some of the instruments installed
    in 2014. During this summer 2017 field campaign (4 -- 17 July 2017), the
    last \emph{ice borehole} data and instruments, as well as \emph{automated
    cameras}, all installed in 2014, were recovered. \emph{Seismometers} were
    again installed and maintained for two weeks on the glacier surface next to
    the calving front, and additional seismometers installed in the ice in 2016
    for one year were retrieved. As in 2016, a second field team set up camp on
    top of a mountain facing the glacier and ran an \emph{interferometric
    RADAR} for two weeks in order to monitor short-term changes in ice
    velocity. This second team also performed daily \emph{Unmanned Aerial
    Vehicle (UAV)} flights to image changes in the calving front geometry, and
    changes in the surface expression of subglacial meltwater plumes exiting
    the glacier front. For the first time in four years, a major calving event
    occurred during the field campaign, which was recorded simultaneously by
    the automated cameras, the seismometers, the interferometric radar, and the
    UAV flights.

    Thanks to this grant extension and additional funding sources (M.~Funk
    department funds, ETH grant No.~ETH-12 16-2 to G.~Jouvet, ETH-UZH Ph.D.
    studentship to A.~Walter), data analysis from this and previous field
    campaigns is still ongoing, and is planned to result in several
    publications. Until June 2019, the postdoc J.~Seguinot will be supported by
    ETH Zürich D-BAUG department funds to complete and publish the analysis of
    borehole data with M.~Funk. From May to July 2018 (and perhaps longer)
    J.~Seguinot will work at Hokkaido University so that the final
    interpretation of borehole data will be performed in close collaboration
    with colleagues from Japan who are currently processing other kind of data
    collected on Bowdoin Glacier. A joint ETH-UZH Ph.D. studentship was opened
    on the use of interferometric RADAR to monitor calving glaciers in
    Greenland, including analysis of data collected at Bowdoin Glacier
    (A.~Walter, supervised by A.~Vieli and M.~Funk). An internship (M.~Kneib,
    supervised by G.~Jouvet and M.~Detert), and another Ph.D studentship
    (E.~van~Dongen, ETH grant No.~ETH-12 16-2 to G.~Jouvet) were opened to
    analyse UAV images and use them to validate a numerical model of glacier
    flow and iceberg calving.


% ----------------------------------------------------------------------
\section{Main results}
% ----------------------------------------------------------------------

\subsection{Ice temperature}

    Between 2014 and 2016, ice temperature measurements exhibited a
    surprisingly local englacial warming trend, most likely due to percolation
    and refreezing of meltwater into deep crevasses (see final report
    No.~200021-153179/1). The full-depth temperature profile measurement
    planned for the summer 2017 field campaign failed, which is probably due to
    damage of the temperature cables installed in 2014 as they advected to a
    more and more crevassed area of the glacier surface. However, ice
    temperature was continuously measured by four sensors (connected to a
    different cable) until Jan. 2017, when cable damage probably occurred near
    the glacier base. Three of these sensors continued to record temperature
    until final removal of data loggers in July 2017. The resulting continuous
    data series of up to three years confirm the local warming trend already
    observed between 2014 and 2016. During the last year it was also found that
    the depth of some sensors, which were inaccurately positioned during the
    drilling, could be retrieved using the borehole freezing times, recorded as
    dip in the temperature time series. Although this will delay the
    publication of temperature data \citep{Seguinot.Funk.Inprep}, the resulting
    data set will be more complete since the depths for all sensors could be
    accurately reconstructed (Fig.~\ref{fig:closure}).

    \begin{figure}
      \centerline{\includegraphics{bowtem_closure}}
      \caption{\textbf{(left)} Measured average temperature on 2 Sep. 2014 in
               relation to number of days between the borehole drilling and
               refreezing. Refreezing takes longer where the ice is warmer.
               \textbf{(right)} Reconstructed sensor depths based on the number
               of days to freezing.}
      \label{fig:closure}
    \end{figure}


\subsection{Basal water pressure}

    Of the two basal water pressure sensors (see final report
    No.~200021-153179/1), only one was left recording in July 2016. This sensor
    continued to record water pressure until removal of the data loggers in
    July 2017, resulting in a continuous data series of three years.
    Surprisingly, the data show a very different pressure response in 2017 as
    compared to previous years.

    This might indicate that the sensor was advected into a different part of
    the subglacial drainage system. However, the relationship between ice-flow
    velocity and borehole water pressure is generally unclear, and it is thus
    difficult to conclude on processes that governed the recorded pressure
    fluctuations \citep{Podolskiy.etal.Inprep}.


\subsection{Ice deformation}

    Similarly to ice temperature, ice deformation was continuously measured by
    four sensors until Jan. 2017, and four sensors until July 2017, once again
    resulting in up to three years of continuous measurements.

    Most interestingly, ice deformation was measured continuously every minute
    for two entire summer melt seasons June--Sep. 2015 and June--Sep. 2016.
    These two periods exhibit distinct, multi-day speed-up events associated
    with periods of warm weather, showing that fluctuations of ice velocity
    related to subglacial meltwater intrusions are reverberated onto the
    internal ice deformation \citep{Seguinot.etal.Inprep}.


\subsection{Tidal pressure fluctuations}

    All inclinometres were equipped with pressure sensors, so that the
    instruments could be accurately located in the water-filled during
    installation. Very unexpectedly, it was found during the last year that
    pressure sensors, normally designed to measure water pressure, continued to
    record after the complete refreezing of the boreholes and the stabilisation
    of ice temperatures well below the pressure melting point.

    All sensors recorded in-phase pressure variations with clear 12-hour and
    14-day periods, leaving no doubt on the tidal nature of the signal.
    Surprisingly, the amplitude of pressure variations recorded in the ice is
    comparable to that of sea level tides measured by our Japanese colleagues
    in front of the glacier, but their sign is opposite
    (Fig.~\ref{fig:highpass}).

    Although the results seem to show that tidal stresses applying to marine
    calving fronts can propagate several kilometres upstream in subfreezing
    glacier ice, further analysis is needed to understand the mechanism
    involved. This work will be conducted by J.~Seguinot in collaboration with
    E.~Podolskiy during his visit at Hokkaido University, and the results will
    be presented in a third publication based on the borehole data
    \citep{Seguinot.etal.Inprepa}.

    \begin{figure}
      \centerline{\includegraphics{bowtid_highpass}}
      \caption{\textbf{(left)} High-pass filtered pressure data series,
               expressed in metres of water equivalent (m~w.e.).
               \textbf{(right)} Zoom on the month of Sep. 2014, also including
               the tidal gauge record from Pituffik (black).}
      \label{fig:highpass}
    \end{figure}


\subsection{Unmanned aerial vehicle}

    Further analysis of 2016 data showed that it is possible to reconstruct
    surface water velocity fields from UAV images of the subglacial meltwater
    plumes. This was the topic for an internship (M.~Kneib, supervised by
    G.~Jouvet and M.~Detert), whose results have now been incorporated in a
    publication \citep{Jouvet.etal.inreview}.

    In turn, daily UAV flights performed during the summer 2017 field campaign
    not only resulted in new high-resolution orthoimages but also included a
    more systematic monitoring of the subglacial meltwater plumes. Data
    analysis is ongoing as part of a new PhD project (E.~van~Dongen, ETH grant
    No.~ETH-12 16-2 to G.~Jouvet).


\subsection{Seismometers}

    The installation of in-ice subsurface seismometers for one year on Bowdoin
    Glacier was very successful. Despite the lack of sunlight and the cold
    temperatures, the four seismometers recorded for the entire year except for
    a short period of time in winter. This resulted in a unique seismic data
    set which is currently being analysed (E.~Podolskiy and F.~Walter).

    Beside, further analysis of the 2015 and 2016 surface seismic record
    allowed to detect long-period (LP) events similar to those produced by
    movement of fluids under volcanoes. These events have a well-defined source
    area and certainly characterise either stick-slip glacier motion or
    water-filled cracks \citep{Podolskiy.etal.Inprep}.


\subsection{Interferometric RADAR}

    During the summer 2017 field campaign, the sampling interval of RADAR data
    collection was decreased from 2 to 1~minute. Although ice motion over such
    time scales is generally small, this change in sampling rate was needed to
    more effectively filter out atmospheric perturbations caused by air flowing
    between the RADAR and the glacier. The analysis of interferometric RADAR
    data from Bowdoin Glacier, and that from another Greenlandic glacier, form
    the topic of an ongoing joint ETH-UZH PhD studentship (A.~Walter,
    supervised by A.~Vieli and M.~Funk). On Bowdoin, the RADAR data will allow
    for very detailed mapping of surface velocity changes that preceded and
    followed the major calving event observed during the 2017 field campaign,
    as well as characterising the evolution of crevassing activity over the
    course of the 2016 and 2017 campaigns.


% ----------------------------------------------------------------------
\section{Major deviations}
% ----------------------------------------------------------------------

    There were no major deviations from the extension project proposal.


% ----------------------------------------------------------------------
\section{Contributions by the project staff}
% ----------------------------------------------------------------------

    During the summer 2017 field campaign, L.~Preiswerk and E.~Podolskiy
    retrieved the last borehole data and equipment, and the automated cameras.
    J.~Seguinot and M.~Funk are currently analysing ice temperature, ice
    deformation, and tidal pressure record from the borehole data. L.~Preiswerk
    and E.~Podolskiy also retrieved the subsurface seismometers and maintained
    the temporary surface seismometers for the duration of the fieldwork.
    Seismic data is currently being analysed by E.~Podolskiy and F.~Walter.

    M.~Kneib operated daily UAV flights over the calving front and meltwater
    plumes, and analysed 2016 UAV data. New UAV data are currently being
    analysed by E.~van~Dongen and G.~Jouvet. A.~Walter and M.~Funk operated the
    interferometric RADAR, and currently analyse RADAR data together with
    A.~Vieli.

    In the framework of a collaboration with other ETH scientists, J~Seguinot
    also recently submitted a manuscript on the palaeo-glaciation of the Alps
    where SNF funding was acknowledged \citep{Seguinot.etal.2018}, and
    participated in a subglacial hydrology modelling inter-comparison where SNF
    funding was also acknowledged \citep{Fleurian.etal.submitted}.


% ----------------------------------------------------------------------
\section{Contributions by partners abroad}
% ----------------------------------------------------------------------

    During the summer 2017 field campaign, colleagues from Hokkaido University
    conducted continued weather observations, GPS records of ice surface
    velocity, GPS surveys of ice surface lowering, glacier surface mass balance
    measurements, new supra-glacial and oceanic water sampling, fjord
    bathymetry echo sounding, and recovery of two moorings that measured
    temperature and salinity for a year. Besides, analysis of satellite imagery
    of the region also resulted in a new publication
    \citep{Sakakibara.Sugiyama.2018}.


% ----------------------------------------------------------------------
%\section{Important events}
% ----------------------------------------------------------------------


% ----------------------------------------------------------------------
% References
% ----------------------------------------------------------------------

\small
\bibliographystyle{abbrvnat}
\bibliography{../../references/references}

% ======================================================================
\end{document}
% ======================================================================
